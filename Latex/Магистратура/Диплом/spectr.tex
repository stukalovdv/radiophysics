\subsection{Спектральный анализ полученных данных}

Спектральная обработка сигналов -- как дискретных, так и аналоговых -- позволяет оценить вид сигнала (импульса), излучаемого системой. Для изучения получаемой волны имеет место построить спектры, нормированные на максимальную спектральную плотность мощности. Таким образом можно оценить ширину спектральной линии и основные частоты, но не мощность излучения относительно разных параметров. 

Построим такой рисунок при $R_{2}=1.64$ и $\delta=0.1$ в зависимости от $\nu/\omega_{p0}$ (рис.\ref{ris:spm_nu}).
Как видно из полученных данных, из существенных изменений при нахождении спектра в зоне максимального КПД можно выделить только увеличение ширины линии, когда как остальные параметры излучаемой волны и её общий вид не меняются.

Спектральная картина только уже для фиксированных $\nu/\omega_{p 0}=0.1$ и $\delta=0.1$ изображена на рисунке \ref{ris:spm_r2}. Тот факт, что все спектры легко различимы на одном графике без дополнительного масштабирования, а также что построение нормированных графиков в этом случае несколько исказило бы картину, спектры построены в ненормированном виде.


Как видно на рисунке \ref{ris:spm_r2}, при меньших $R_{2}$ видна картина, аналогичная картине на рисунке \ref{pic:spm_res} -- наблюдается геометрический резонанс, далее излучение становится низкодобротным и зона геометрического резонанса убывает и остается только излучение на центральной частоте $\omega_{p0}$, что можно объяснить снижением влияния геометрического резонанса. 

При существенном увеличении $R_{2}$ излучается волна на частоте $\omega=\omega_{p0}$, а ширина спектральной линии уменьшается. Однако большие значения $R_{2}$ не представляют существенный практический интерес ввиду низкого КПД получаемого излучения -- б\'ольшая часть запасённой энергии тратится либо на соударения, либо на резонансные потери, также цилиндр становится слишком большим, чтобы запасенная энергия эффективно тратилась на излучение.