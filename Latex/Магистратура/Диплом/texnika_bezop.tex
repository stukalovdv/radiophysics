\section*{Техника безопасности}

При пользовании средствами вычислительной техники и периферийным оборудованием каждый работник должен внимательно и осторожно обращаться с электропроводкой, приборами и аппаратами и всегда помнить, что пренебрежение правилами безопасности угрожает и здоровью, и жизни человека. Во избежание поражения электрическим током необходимо твердо знать и выполнять следующие правила безопасного пользования электроэнергией:

Во избежание повреждения изоляции проводов и возникновения коротких замыканий не разрешается:
\begin{itemize}
	\item вешать что-то на провода;
	\item закладывать провода и шнуры за газовые и водопроводные трубы, за батареи отопительной системы;
	\item выдергивать штепсельную вилку из розетки за шнур, усилие должно быть приложено к корпусу вилки.
\end{itemize}

Для исключения поражения электрическим током запрещается:
\begin{itemize}
	\item прикасаться к экрану и к тыльной стороне блоков компьютера; 
	\item работать на средствах вычислительной техники и периферийном оборудовании мокрыми руками; 
	\item класть на средства вычислительной техники и периферийном оборудовании посторонние предметы.
	\item ремонт электроаппаратуры производится только специалистами-техниками с соблюдением необходимых технических требований.
	\item недопустимо под напряжением проводить ремонт средств вычислительной техники и периферийного оборудования.	
\end{itemize}

Согласно медицинским нормам, работать за компьютером следует в течение двух часов, после чего необходимо делать перерыв.