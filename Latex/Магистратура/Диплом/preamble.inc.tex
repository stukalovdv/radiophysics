\usepackage{setspace}%Пакет с интервалами

%Работа с цитатами

\usepackage[font={small,it}]{caption}
\usepackage{float} % пакеты с графикой
\usepackage{tikz}
\usepackage{pgfplots}
\usepackage{matlab-prettifier,graphics,epstopdf}
\usepackage{pstool}
\usepackage{siunitx}
\pgfplotsset{compat=1.9}

\usepackage[english,russian]{babel} %языковые пакеты
\usepackage{mathtext,amsmath,unicode-math,empheq}

\usepackage[utf8]{inputenc}%кодировка шрифта utf-8
\usepackage{fontenc} %шрифты
\usepackage{fontspec}

\usepackage{multicol}%Мультиколонки
\setlength{\columnsep}{1cm}

%Шрифты в заголовках
\usepackage{titlesec}
\usepackage{secdot}
\titleformat{\section}
{\centering\normalfont\fontsize{14}{21}\bfseries}{\thesection.}{1em}{}
\titleformat{\subsection}
{\centering\normalfont\fontsize{14}{21}\bfseries}{\thesubsection.}{1em}{}
\titleformat{\subsubsection}
{\centering\normalfont\fontsize{14}{21}\bfseries}{\thesubsubsection.}{1em}{}
%\titlelabel{\thetitle .\quad}
%

\usepackage{cancel}%зачеркивания в формулах
\usepackage{icomma}%умная запятая
\usepackage[bookmarks=true, colorlinks=true,unicode=true,urlcolor=black,linkcolor=black, anchorcolor=black,citecolor=black,menucolor=black, filecolor=black]{hyperref}%гиперссылки

\usepackage[includeheadfoot=true]{geometry}%параметры страницы
\geometry{a4paper, total={170mm,257mm},left=30mm,right=15mm,top=20mm,bottom=20mm}
\usepackage{indentfirst} %делать отступ в начале параграфа
\setlength{\parindent}{7.4mm} %7,4мм

\numberwithin{equation}{section} %нумерация уравнений с номерами разделов
\renewcommand{\refname}{Список источников} %Вместо "Список литературы" будет "Список источников"
\usepackage{csquotes}
\bibliographystyle{gost-numeric}
%\usepackage[backend=biber,bibencoding=utf8,style=numeric-comp,bibstyle=gost-numeric]{biblatex}
\usepackage[parentracker=true,backend=biber,bibencoding=utf8,citestyle=gost-numeric,bibstyle=gost-numeric]{biblatex}
%\addbibresource{biblio.bib}
\addbibresource{Collection.bib}

%\singlespacing
%\onehalfspacing%полуторный интервал
%\doublespacing
\setmainfont{Times New Roman}%тут всё просто)

\usepackage{xparse}

\let\oldsection\section
\makeatletter
\newcounter{@secnumdepth}
\RenewDocumentCommand{\section}{s o m}{%
	\IfBooleanTF{#1}
	{\setcounter{@secnumdepth}{\value{secnumdepth}}% Store secnumdepth
		\setcounter{secnumdepth}{0}% Print only up to \chapter numbers
		\oldsection{#3}% \section*
		\setcounter{secnumdepth}{\value{@secnumdepth}}}% Restore secnumdepth
	{\IfValueTF{#2}% \section
		{\oldsection[#2]{#3}}% \section[.]{..}
		{\oldsection{#3}}}% \section{..}
}
\makeatother
