%!TEX TS-program = xelatex
\documentclass[a4paper]{article}  

\usepackage{setspace}%Пакет с интервалами

%Работа с цитатами

\usepackage[font={small,it}]{caption}
\usepackage{float} % пакеты с графикой
\usepackage{tikz}
\usepackage{pgfplots}
\usepackage{matlab-prettifier,graphics,epstopdf}
\usepackage{pstool}
\usepackage{siunitx}
\pgfplotsset{compat=1.9}

\usepackage[english,russian]{babel} %языковые пакеты
%\usepackage{mathtext,amsmath,unicode-math,empheq}
\usepackage{amsmath,unicode-math,empheq} %на линуксе нет mathtext

\usepackage[utf8]{inputenc}%кодировка шрифта utf-8
\usepackage{fontenc} %шрифты
\usepackage{fontspec}

\usepackage{multicol}%Мультиколонки
\setlength{\columnsep}{1cm}

%Шрифты в заголовках
\usepackage{titlesec}
\usepackage{secdot}
\titleformat{\section}
{\centering\normalfont\fontsize{14}{21}\bfseries}{\thesection.}{1em}{}
\titleformat{\subsection}
{\centering\normalfont\fontsize{14}{21}\bfseries}{\thesubsection.}{1em}{}
\titleformat{\subsubsection}
{\centering\normalfont\fontsize{14}{21}\bfseries}{\thesubsubsection.}{1em}{}
%\titlelabel{\thetitle .\quad}
%

\usepackage{cancel}%зачеркивания в формулах
%\usepackage{icomma}%умная запятая
\usepackage[bookmarks=true, colorlinks=true,unicode=true,urlcolor=black,linkcolor=black, anchorcolor=black,citecolor=black,menucolor=black, filecolor=black]{hyperref}%гиперссылки

\usepackage[includeheadfoot=true]{geometry}%параметры страницы
\geometry{a4paper, total={170mm,257mm},left=30mm,right=15mm,top=20mm,bottom=20mm}
\usepackage{indentfirst} %делать отступ в начале параграфа
\setlength{\parindent}{7.4mm} %7,4мм

\numberwithin{equation}{section} %нумерация уравнений с номерами разделов
\renewcommand{\refname}{Список источников} %Вместо "Список литературы" будет "Список источников"
\usepackage{csquotes}
\bibliographystyle{gost-numeric}
%\usepackage[backend=biber,bibencoding=utf8,style=numeric-comp,bibstyle=gost-numeric]{biblatex}
\usepackage[parentracker=true,backend=biber,bibencoding=utf8,citestyle=gost-numeric,bibstyle=gost-numeric]{biblatex}
%\addbibresource{biblio.bib}
\addbibresource{Collection.bib}

%\singlespacing
%\onehalfspacing%полуторный интервал
%\doublespacing
\setmainfont{Times New Roman}%тут всё просто)

\usepackage{xparse}

\let\oldsection\section
\makeatletter
\newcounter{@secnumdepth}
\RenewDocumentCommand{\section}{s o m}{%
	\IfBooleanTF{#1}
	{\setcounter{@secnumdepth}{\value{secnumdepth}}% Store secnumdepth
		\setcounter{secnumdepth}{0}% Print only up to \chapter numbers
		\oldsection{#3}% \section*
		\setcounter{secnumdepth}{\value{@secnumdepth}}}% Restore secnumdepth
	{\IfValueTF{#2}% \section
		{\oldsection[#2]{#3}}% \section[.]{..}
		{\oldsection{#3}}}% \section{..}
}
\makeatother
% подгрузить преамбулу
%\newcommand*\rfrac[2]{{}^{#1}\!/_{#2}}
\begin{document}
\fontsize{14}{21pt}\selectfont
\section*{Введение}
\paragraph{\Large{1. Цели и задачи дисциплины. Место дисциплины в учебном процессе.}}

 Дисциплина \textbf{<<Волны в нелинейных средах>>} относится к вариативной части профессионального цикла магистерского образования на радиофизическом факультете ННГУ по направлению подготовки \textbf{03.04.03 -- <<Радиофизика>>}. Она является частью \textbf{магистерской программы} кафедры электродинамики \textbf{<<Электромагнитные волны в средах>>}. В соответствии с идеологией обучение по \textbf{магистерской программе} дисциплина \textbf{<<Волны в нелинейных средах>>} обязательна для освоение в первом семестре образовательного цикла. Курс читается с целью познакомить выпускника с методами описания широкого круга нелинейных явлений в электродинамике (в диэлектриках, активных средах и плазме), газодинамике, гидродинамике и других областях физики. На основе колебательно-волновой аналогии и общности математического описания радиофизик - исследователь должен видеть общее в разнообразных явлениях, происходящих в различных средах.
 
 Курс предполагает познакомить выпускника с современными методами отыскания точных решений достаточно широкого класса нелинейных уравнений в частных производных. С помощью метода обратной задачи теории рассеяния, преобразований Бэклунда, Миуры, Хопфа-Хироты описываются самые разнообразные физические процессы и явления. Отыскиваемые с помощью этих новых быстро развивающихся методов  точные (т.н. многосолитонные) решения претендуют в настоящее время на такую же роль основных (или базисных) решений нелинейных уравнений в частных производных, какую играют экспоненциальные решения в теории линейных уравнений в частных производных.
	
\section*{Определение терминов}
\begin{itemize}
	\item[а)] \textbf{\textit{Волна}}  По Дж. Уизему [3] “не существует единого строгого определения волн. Можно дать различные частные определения, но чтобы охватить весь диапазон волновых процессов, предпочтительнее пользоваться интуитивным представлением о волне, как о любом различимом сигнале, передающемся от одной части Среды к другой с некоторой определенной скоростью. Такой сигнал может быть возмущением любого вида ... при условии, что это возмущение четко видимо и что в любой заданный момент времени можно определить его местонахождение. Этот сигнал может искажаться, изменять свою величину и скорость, но при этом оставаться различимым”. Итак, \textit{\textbf{волна - это убегающее (перемещающееся) различимое возмущение.}}
	\item[б)] \textbf{\textit{Среда или распределенная система}} 
	
	Распределенная система – система, состоящая из бесконечно большого числа связанных элементарных звеньев, ячеек или систем и обладающая поэтому бесконечно большим числом степеней свободы (собственных нормальных колебаний).\newline \textbf{Распределенная система} -- это  либо система с неограниченным объемом, либо система ограниченного объема с бесконечно большим числом степеней свободы. Еще один аспект распределенности состоит в том, что локально однородные участки составляют в совокупности неоднородные распределенные системы.
	
	Волны в распределенной системе описываются дифференциальными уравнениями в частных производных типа
	\begin{equation}
		A\frac{\partial u}{\partial t}+B\frac{\partial u}{\partial x}+uC=0  ,
	\end{equation}
	которые дополняются соответствующими граничными и начальными условиями. 
	
	\item[б)] \textbf{\textit{Линейные среды}}
	
	Процессы в \textbf{\textit{линейных средах}} описываются линейными по амплитуде поля $u$ уравнениями (1), в которых и коэффициенты 
	\begin{equation}
		A\equiv A(x,t);\quad B\equiv B(x,t);\quad C\equiv C(x,t);
	\end{equation}
	не зависят от амплитуды поля $u$. При возбуждении поля в линейной среде (распределенной системе) всегда проявляются два основных свойства. 
	\begin{itemize}
		\item[1)] Воздействие на частоте $\omega$ рождает возмущение (волну) только на частоте $\omega$.
		\item[2)] \textbf{Принцип суперпозиции,} который применительно к процессам в электродинамике имеет такую простую  формулировку:\newline
		\textit{Если источники $\rho_{1,2}$ создают в среде поля $u_{1,2}$, то источники  $\rho_{1}+\rho_{2}$  создадут поля $u_{1}+u_{2}$. }
	\end{itemize}
	\item[г)] \textbf{\textit{Нелинейные среды}}
	
	Волны (и другие процессы) в \textbf{\textit{нелинейных средах}} описываются нелинейными уравнениями в частных производных. В большинстве случаев эти уравнения являются \textbf{квазилинейными уравнениями} типа  (1), в которых производные $u'_{x}$ , $u'_{t}$  входят линейно и коэффициенты 
	\begin{equation}
		A\equiv A(x,t,u);\quad B\equiv B(x,t,u);\quad C\equiv C(x,t,u);
	\end{equation}
	зависят только от поля $u$. Примером таких уравнений являются уравнения электродинамики нелинейной среды (нелинейного диэлектрика или нелинейного магнетика) без источников 
	\begin{equation}\label{4}
		\begin{cases}
			rot\vec{E}=-\dfrac{1}{c}\dfrac{\partial}{\partial t}\vec{B};\\
			rot\vec{H}=\dfrac{4\pi}{c}\vec{j}+\dfrac{1}{c}\dfrac{\partial}{\partial t}\vec{D};\\
			div\vec{D}=0;\\
			div\vec{B}=0
		\end{cases}
	\end{equation}
	с материальными уравнениями
	\begin{equation}
		\begin{cases}
		\vec{D}=\vec{E}+4\pi\vec{P}^{L}+4\pi\vec{P}^{NL}=\hat\varepsilon\vec{E}+4\pi\vec{P}^{NL}(\vec{E},\vec{H});\\
		\vec{P}^{L}=\hat{\xi}\vec{E};\\
		\hat{\varepsilon}=1+4\pi\hat{\chi};\\
		\vec{B}=\vec{H}+4\pi\vec{M}^{L}+4\pi\vec{M}^{NL}=\hat{\mu}\vec{H}+4\pi\vec{M}^{NL}(\vec{E},\vec{H});\\
		\vec{M}^{L}=\hat{\beta}\vec{H};\qquad\hat{\mu}=1+4\pi\hat{\beta};\qquad\vec{j}=\hat{\sigma}\vec{E}+\vec{j}^{NL}\left(\vec{E},\vec{H}\right),	
		\end{cases}
	\end{equation}
в которых $\vec{M}^{NL}(\vec{E},\vec{H})$, $\vec{P}^{NL}(\vec{E},\vec{H})$, и $\vec{j}^{NL}\left(\vec{E},\vec{H}\right)$ определены в виде функций от переменных $\vec{H},\vec{E}$.
\end{itemize}
\newpage
В качестве наиболее типичной нелинейной среды можно привести диэлектрик с так называемой \textit{кубичной нелинейностью}. Для него материальные уравнения имеют вид 
\begin{equation}
	\vec{D}=\vec{E}+4\pi\chi^{L}\cdot\vec{E}+4\pi\left(\hat{\chi}^{NL}\cdot\vec{E}\cdot\vec{E}\cdot\vec{E}\right)\equiv\left(\varepsilon+\alpha\left|\vec{E}\right|^{2}\right)\vec{E}=\varepsilon\vec{E}+4\pi\vec{P}^{NL}\tag*{(5')}
\end{equation}
$$\vec{B}=\mu\vec{H} (\mu=const), \vec{j}=0 (\sigma\equiv0)$$

где $\varepsilon=1+4\pi\chi^{L},\quad\alpha=4\pi\chi^{NL}.$

\section*{Некоторые свойства нелинейных сред} 

При возбуждении поля в \textit{\textbf{нелинейной среде}} нарушаются оба приведённые выше свойства \textit{\textbf{линейной среды}}. 

Рассмотрим на примере среды с \textit{кубичной нелинейностью} некоторые свойства нелинейных сред, которые проявляются при возбуждении полей
\begin{itemize}
	\item[1.]\textbf{Изменение спектрального состава действующего излучения (рождение новых гармоник частоты и пространственных гармоник)}
	\item[1.1] Пусть на входе в среду имеется поле 
	
	\begin{minipage}{0.5\textwidth}
		\begin{flushleft}
			$$\vec{E_{1}}=Re\left\{\vec{\tilde{E_{1}}}exp\left[i\omega t\right]\right\}=\dfrac{1}{2}\vec{\tilde{E_{1}}}exp\left[i\omega t\right]+\text{\textit{к.с.}}$$
			Тогда внутри \textit{тонкого} слоя (рис.1В) на частоте $3\omega$ появляется поляризация $\vec{P}^{NL}$, и поэтому на другом конце тонкого слоя (на выходе) будем иметь сильное поле
			$\vec{E_{1}}=Re\left\{\vec{\tilde{E_{1}}}exp\left[i\omega t\right]\right\}$ и слабенькое поле $\vec{E_{3}}=Re\left\{\vec{\tilde{E_{3}}}exp\left[3i\omega t\right]\right\}$, возбуждаемое слабым $\vec{j}(3\omega)=\left(\partial\vec{P}^{NL}/\partial t\right).$
		\end{flushleft}
	\end{minipage}
	\begin{minipage}{0.5\textwidth}
		\begin{flushright}
			right text
		\end{flushright}
	\end{minipage}
	\newpage
	Как этот интуитивно ощущающийся результат увидеть из уравнений электродинамики? 
	
	Преобразуем уравнения Максвелла к виду 
	\begin{equation*}
		rotrot\vec{E}+\frac{\mu}{c^{2}}\frac{\partial^{2}\left(\varepsilon\vec{E}\right)}{\partial t^{2}}=-\frac{4\pi}{c^{2}}\mu\frac{\partial^{2}\vec{P}^{NL}}{\partial t^{2}}\tag{4'}
	\end{equation*}

	Рассмотрим одномерное приближение 
	\begin{equation*}
		rotrot\vec{E}=-\frac{d^{2}}{dz^{2}}\vec{E}\tag{4''}\label{4''}
	\end{equation*}
	(так как $div\vec{E}=0,\quad\Delta=\frac{d^{2}}{dz^{2}}$) и подставим в \eqref{4''} поле $\vec{\tilde{E}}$ в виде 
	$$\vec{E}=Re\left\{\vec{\tilde{E_{1}}}exp\left[i\omega t\right]+\vec{\tilde{E_{3}}exp\left[3i\omega t\right]}\right\} .$$
	
	Рассмотрим установившееся во времени решение внутри этого слоя. Тогда в силу ортогональности функций $\cos(\omega t)$ и $\cos(3\omega t)$ на больших временах из \eqref{4''} можно получить два уравнения  
	\begin{align}
		-\frac{d^{2}\vec{\tilde{E_{1}}}}{dz^{2}}-\frac{\mu\varepsilon(\omega)}{c^{2}}\omega^{2}\vec{\tilde{E_{1}}}\cong3\frac{4\pi}{4c^{2}}\omega^{2}\mu\left(\hat{\chi}^{NL}_{\omega}\cdot\vec{\tilde{E_{1}}}\cdot\vec{\tilde{E_{1}}}\cdot\vec{\tilde{E_{1}}}\right);\label{6}\\		-\frac{d^{2}\vec{\tilde{E_{3}}}}{dz^{2}}-\frac{\mu\varepsilon(3\omega)}{c^{2}}9\omega^{2}\vec{\tilde{E_{3}}}\cong\frac{4\pi}{4c^{2}}\left(3\omega\right)^{2}\mu\left(\hat{\chi}^{NL}_{3\omega}\cdot\vec{\tilde{E_{1}}}\cdot\vec{\tilde{E_{1}}}\cdot\vec{\tilde{E_{1}}}\right)\label{7}
	\end{align}
для каждой из монохроматических компонент.

Однородное нелинейное уравнение \eqref{6} для поля сильной волны имеет в своей правой части малый нелинейный член. Этим членом можно пренебречь, поскольку из-за его наличия $\vec{\hat{E}}_{1}$ меняется слабо. В этом линейном приближении, позволяющем пренебречь малой правой частью, \eqref{6} приобретает вид однородного (линейного) уравнения без источников. Решение уравнения \eqref{6} в таком приближении заданного поля сильной волны имеет вид 
\begin{equation}
	\vec{\hat{E}}_{1}=\vec{\hat{E}}_{10}\cdot exp\left(-ikz\right)\equiv\tilde{\varepsilon}_{1}exp\left(-ikz\right)\vec{e}_{1}^{0}
\end{equation}

Оно описывает распространение волны постоянной амплитуды, величина которой определяется из условия на границе $z=0$.

Поле слабой волны на частоте $3\omega$ описывается уравнением \eqref{7}, в правой части которого находится источник (сторонний ток) на частоте $3\omega$ в виде наведённой сильным полем $\vec{\hat{E}}_{1}$ нелинейной поляризации 
\begin{equation}\label{v.9}
	\vec{\hat{P}}_{3\omega}^{NL}=4\pi\hat{\varepsilon}^{3}_{1}\left(\hat{\chi}_{3\omega}^{NL}\cdot\vec{e}^{0}_{1}\cdot\vec{e}^{0}_{1}\cdot\vec{e}^{0}_{1}\right)exp\left\{i\left(3\omega t-3kz\right)\right\}.
\end{equation}

В результате  возникает вынужденное решение на частоте $3\omega$ в виде 
\begin{equation}\label{10}
	\vec{\hat{E}}_{3}=\vec{e}_{3}^{0}\hat{\varepsilon}_{3}(z)exp\left[i(3\omega t-k_{3}z)\right],
\end{equation}
которое описывает поле, возникающее на границе $z=0$ и распространяющееся в $+z$-направлении  с фазовой скоростью
\begin{equation}
	v_{f}(3\omega)\equiv v_{f3}=(3\omega/k_{3})\equiv\frac{c}{\sqrt{\varepsilon_{3}\mu}}
\end{equation}
Отметим, что фазовый фронт волны тока (источника) перемещается также в $+z$–направлении с фазовой скоростью 
\begin{equation}
	v_{f}(3\omega)=(3\omega/3k)\equiv\frac{c}{\sqrt{\varepsilon_{1}\mu}}
\end{equation}
Правая часть уравнения \eqref{7} мала по сравнению с левой, поэтому решение уравнения \eqref{7} можно искать методом Ван дер Поля. Подставляя \eqref{10} в уравнение для поля   $\vec{\hat{E}}_{3}$, получим уравнение первого порядка для амплитуды 
\begin{equation}\label{v.13}
	2ik_{3}(\hat{d\varepsilon_{3}}/dz)exp(-ik_{3}z)=\frac{\alpha_{3}}{4\varepsilon_{3}}k_{3}^{2}\hat{\varepsilon_{1}}^{3}exp(-i3kz),\tag{В.13}
\end{equation}
где $\alpha_{3}=3\cdot\left(\vec{e_{3}}^{0}\cdot\hat{\chi}_{3\omega}^{NL}\cdot\vec{e}^{0}_{1}\cdot\vec{e}^{0}_{1}\cdot\vec{e}^{0}_{1}\right)4\pi$. Источник в правой части \eqref{v.13} зависит от координаты как $exp\{i(k_{3}-3k)z\}\equiv exp\{i\Delta kz\}.$ Решение \eqref{v.13} в виде
\begin{equation}\label{14}
	\hat{\varepsilon_{3}}(z)=\frac{\alpha_{3}\hat{\varepsilon_{1}}^{3}}{4\varepsilon_{3}\cdot2i}k_{3}\int_{0}^{z}exp(i\Delta kz')dz'=\frac{\alpha_{3}\varepsilon_{1}^{3}k_{3}}{4i\Delta k\varepsilon_{3}}exp\left(i\frac{1}{2}\Delta kz\right)sin\frac{\Delta k}{2}z
\end{equation}
описывает поле, периодически изменяющееся в направлении распространения. Интенсивность этого поля, пропорциональная $\sin^{2}\frac{1}{2}\Delta kz=\frac{1}{2}-\frac{1}{2}\cos\Delta kz,$ является периодическо функцией аргумента с периодом
\begin{equation}\label{15}
	z_{\text{П}}=\frac{2\pi}{\left|\Delta k\right|}\equiv\frac{2\pi}{\left|k_{3}-3k_{1}\right|}
\end{equation}

Амплитуда пространственных колебаний интенсивности поля 
\begin{equation}
	\label{16}
	\left|\tilde{\varepsilon}_{3}(z)\right|^{2}_{max}=\left|\frac{\alpha_{3}\tilde{\varepsilon}_{1}^{3} k_{3}}{4\varepsilon_{3}\Delta k}\right|^{2}
\end{equation}
сильно зависит от согласования фазовых скоростей (от дисперсии среды). Если $v_{f}(\omega)$ близко к $v_{f}(3\omega)$, то поле $\tilde{\varepsilon}_{3}$ будет быстро нарастать в $+z$–направлении. В пределе при точном синхронизме $\Delta k\rightarrow0$  поле  $\tilde{\varepsilon}_{3}(z)$  переходит в поле 

\begin{equation}
	\lim_{\Delta k\rightarrow0}\tilde{\varepsilon}_{3}(z)=\frac{\alpha_{3}\tilde{\varepsilon}_{1}^{3}k_{3}}{8i\varepsilon_{3}}z
\end{equation}
нарастающее в слое среды по линейному закону.
\item[1.2.] Пусть на входе в среду имеется поле 
$$2\vec{E}=\vec{\tilde{E_{1}}}exp(i\omega_{1}t)+\vec{\tilde{E_{2}}}exp(i\omega_{2}t)+\text{\textit{к.с.}}$$
Тогда на выходе первого бесконечно тонкого слоя получим слабые поля на частотах $3\omega_{1}, 2\omega_{1}+\omega_{2}, 2\omega_{1}-\omega_{2},3\omega_{2}, 2\omega_{2}+\omega_{1}, 2\omega_{2}-\omega_{1}$. 

\begin{minipage}{0.5\textwidth}
	\begin{flushleft}
		Так происходит рождение новых временных гармоник и, следовательно, рождение пространственных гармоник, т.е. изменение структуры распространяющегося излучения. 
		\item[2.] В \textbf{\textit{нелинейной среде}} нарушается принцип суперпозиции. В качестве простого примера можно привести нелинейную интерференцию (Рис. 2В) двух распространяющихся под углом друг к другу плоских волн одной частоты  
		$$\vec{\tilde{E}}_{1,2}=\left(\vec{\tilde{E}}_{1,2}\right)_{0}exp\left\{i(\mp k_{x}x-k_{z}z)\right\}.$$
	\end{flushleft}
\end{minipage}
\begin{minipage}{0.5\textwidth}
	\begin{flushright}
		
	\end{flushright}
\end{minipage}

В результате появления решётки диэлектрической проницаемости волны начинают взаимодействовать друг с другом, обмениваясь энергией. Это явление но-сит название нелинейного рассеяния. Если угол между направлениями распространения волн невелик, то явление носит название \textbf{\textit{попутного двухволнового взаимодействия в кубичной среде}}. При определённых условиях можно организовать перекачку энергии из одной волны в другую. 
\end{itemize}

\section*{Природа нелинейности}

Какова природа нелинейности? Она различна в разных средах. В электродинамике природа нелинейности заключена в нелинейном характере взаимодействия электромагнитного поля с веществом. Нелинейность проявляется только в сильном поле, когда напряженность поля становится сопоставимой с внутриатомной напряжённостью поля  $E_{a}$. Поэтому все среды в электродинамике нелинейны, но все при полях, разных по своей величине. В то же самое время все среды линейны при слабых полях.

В качестве примера можно рассмотреть резонансную среду, в которой электромагнитные явления описываются системой уравнений \eqref{4} и материальными уравнениями 

\begin{align}
	\begin{cases}
	\vec{B}=\mu\vec{H};\qquad\vec{D}=\varepsilon\vec{E}=\vec{E}+4\pi(\vec{P}^{L}+\vec{P}^{NL})\equiv\vec{E}+4\pi N_{0}<Sp\left[\hat{\vec{d}}\cdot\hat{\rho}\right]>;\\
	ih\frac{\partial\hat{\rho}}{\partial t}=\left[\hat{H}\cdot\hat{\rho}\right];\qquad\hat{H}=\hat{H}_{0}-\left(\hat{\vec{d}}\cdot\vec{E}\right);\\
	\hat{\rho}=
	\begin{vmatrix}
		\tilde{\rho}_{11}& \tilde{\rho}_{12}\\
		\tilde{\rho}_{21}& \tilde{\rho}_{22}
	\end{vmatrix};\qquad
	\hat{H}_{0}=
	\begin{vmatrix}
		W_{1}& 0\\
		0& W_{2}
	\end{vmatrix};\qquad
	\hat{\vec{d}}=
	\begin{vmatrix}
		\tilde{\vec{d}}_{11}& \tilde{\vec{d}}_{12}\\
		\tilde{\vec{d}}_{21}& \tilde{\vec{d}}_{22}
	\end{vmatrix};
	\end{cases}
	\tag{5''}\label{5''}
\end{align}
В этой системе при больших полях нелинейны материальные уравнения, которые описывают связь индукции и поля.

В ферритах и в плазме также нелинейны материальные уравнения. 

В газодинамике и гидродинамике нелинейность создаётся из-за нелинейной связи плотности, скорости, температуры и энтропии между собой. В гидродинамике все четыре основных уравнения
\begin{eqnarray*}
	& \frac{\partial\rho}{\partial t}+div\rho\vec{V}=0;\\
	& \frac{\partial\vec{V}}{\partial t}+\left(\vec{V}\cdot\nabla\right)\vec{V}=-\frac{1}{\rho}\nabla p+\left[\frac{\eta}{\rho}\Delta\vec{V}+\frac{1}{\rho}\left( +\frac{\eta}{3}\right)\nabla div\vec{V}\right]+\vec{g}+\frac{1}{\rho}\vec{f}_{M};\\
	& \rho T\left[\frac{\partial S}{\partial t}+\left(\vec{V}\cdot\nabla S\right)\right]=\left(\hat{\sigma}\cdot\nabla\right)\vec{V}+div\left(k\nabla T\right)+\rho Q;\\
	& p=p\left(\rho,S\right)
\end{eqnarray*}
имеют нелинейные члены. Но во многих случаях (и всегда при малых воздей-ствиях) эти уравнения оказываются практически линейными (пример – акустика).

От чего зависит эффективность процессов нелинейного взаимодействия гармоник, эффективность нелинейных преобразований волн? Обратимся к обсуждавшемуся примеру образования третьей гармоники. 

Поле $\vec{\tilde{E}}_{3}$ будет эффективно расти, если $\varepsilon\left(3\omega\right)\cong\varepsilon(\omega)$. А если это условие не выполняется, то решение будет иметь вид функции, осциллирующей вдоль оси $z$. Амплитуда осцилляций будет зависеть от периода осцилляций $z_{\text{П}}$ . Чем меньше $\Delta k$ (и чем больше $\alpha_{3}\tilde{\varepsilon}_{1}^{3}$  ), тем больше будет максимальное значение поля $\vec{\tilde{E}}_{3}$. Т.о., дисперсия противоборствует росту гармоники $\vec{\tilde{E}}_{3}$. 

Дисперсия как название явления произошло от слова <<расползание>> (dispersion). <<Расползание>> чего? >> Имеется в виду расползание гармоник. Дисперсия – свойство линейных сред (линейных систем, линейных процессов). В электродинамике – это зависимость $v_{f}=c/\sqrt{\varepsilon(\omega)\mu(\omega)}$ от $(\omega)$. При наличии $\varepsilon(\omega)$ волновой пакет из набора гармоник по мере распространения расползается. Его отдельные части (куски спектра) перемещаются каждый со своей групповой скоростью $\textbf{V}_{\text{\textit{гр}}}=(d\omega/dk)$ и постепенно его энергия размазывается во все большем и большем объеме.
В этой связи удобно изменить определение дисперсии и считать, что среда является диспергирующей, если 
\begin{equation}
	(d\omega/dk)\neq const\qquad\text{или}\qquad(d^{2}\omega/dk^{2})\neq0.
\end{equation}

Итак, нелинейность порождает гармоники, формирует резкие фронты и пр. Дисперсия расталкивает гармоники, заставляет их расползаться. В их противоборстве протекают нелинейные процессы. Параметр нелинейность/дисперсия может служить основой для классификации нелинейных процессов и эффектов. Этим же параметром определяется и выбор методов решения задач. Термины <<сильная>> или <<слабая>> нелинейность возникли в результате сопоставления действия нелинейности и дисперсии. 

\section*{Природа дисперсии}
В основе дисперсии находятся протекающие в среде периодические процессы, имеющие характерные временные масштабы, и периодические пространственные дислокации среды. 
\begin{itemize}
	\item[а)] В электродинамике роль периодических процессов играют собственные колебания частиц среды в собственных (внутренних) полях или под действием каких-то \underline{постоянных} внешних полей. Это могут быть 1) собственные колебания атомов, молекул, ядер и т.д. в собственных полях; 2) колебания свободных электронов в плазме с магнитным (внешним) полем с гирочастотой и пр. 
	\item[б)] Проницаемость среды (например, кристалла на рентгеновских длинах волн) может иметь периодическую зависимость от длины волны излучения, соизмеримой с периодом пространственной структуры вещества. 
\end{itemize}
	
	\underline{В случае} а) говорят о временной дисперсии, а \underline{в случае} б) – о пространственной дисперсии. Это разделение природы дисперсии несколько условно, если иметь в виду, например, переход в движущуюся систему координат. 
	
	В качестве простого примера, поясняющего природу временной дисперсии, получим материальные уравнения для среды, которая состоит из совокупности (N штук в единице объёма) одинаковых линейных осцилляторов, находящихся под воздействием монохроматического поля. Тогда каждый электрон-осциллятор движется согласно уравнению 
	\begin{equation}
		m\left(\ddot{\tilde{\vec{r}}}_{\omega}+\omega_{0}^{2}\tilde{\vec{r}}_{\omega}+2\Gamma\dot{\tilde{\vec{r}}_{\omega}}\right)\cong e\tilde{\vec{E}}_{\omega} exp(i\omega t)
	\end{equation}
	(Пренебрегаем в силе Лоренца релятивистскими поправками типа $(V/c)$, где $V$ –скорость движения электрона.) Зная смещение отдельного электрона 
	\begin{equation}
		\tilde{\vec{r}}^{0}_{\omega}=\frac{(e/m)\tilde{\vec{E}}_{\omega}}{(-\omega^{2}+\omega^{2}_{0})+2i\Gamma\omega},
	\end{equation}
	нетрудно найти комплексную амплитуду $e	\tilde{\vec{r}}^{0}_{\omega}$  переменного во времени дипольного момента  и далее вычислить дипольный момент единицы объема среды 
	\begin{equation}
		\tilde{\vec{P}}_{\omega}=Ne	\tilde{\vec{r}}^{0}_{\omega}=\frac{\left(Ne^{2}/m\right)\tilde{\vec{E}}_{\omega}}{(-\omega^{2}+\omega^{2}_{0})+2i\Gamma\omega}=\tilde{\chi}(\omega)\tilde{\vec{E}}_{\omega}.
	\end{equation}
	Восприимчивость является комплексной функцией частоты
	\begin{equation}
		\tilde{\chi}(\omega)=\chi'(\omega)+i\chi''(\omega)=\frac{(\omega_{pe}^{2}/4\pi)}{(-\omega^{2}+\omega_{0}^{2})+2i\Gamma\omega}\equiv\frac{k(4\Gamma\omega_{0})}{(-\omega^{2}+\omega_{0}^{2})+2i\Gamma\omega}.\label{22}
	\end{equation}
	Параметр $\omega_{pe}^{2}=4\pi Ne^{2}/m$ можно назвать плазменной частотой.
	 
	\begin{figure}[h]
		\label{Рис.3В}
	\end{figure}
	Действительные функции 
	\begin{equation}
		\chi'(\omega)=\frac{(-\omega^{2}+\omega_{0}^{2})(\omega_{pe}^{2}/4\pi)}{(-\omega^{2}+\omega_{0}^{2})^{2}+4\Gamma^{2}\omega^{2}}\tag{22'}
	\end{equation}
	\begin{equation}
		\chi''(\omega)=\frac{-2\Gamma\omega(\omega_{pe}^{2}/4\pi)}{(-\omega^{2}+\omega_{0}^{2})^{2}+4\Gamma^{2}\omega^{2}}
		\tag{22''}
	\end{equation}
	представлены на \eqref{Рис.3В}

	Прежде всего, проанализируем зависимость $\chi'(\omega)$, поскольку \\$\varepsilon'(\omega)=1+4\pi\chi'(\omega)$. Из рисунка видно, что имеются две области нормальной дисперсии
	\begin{equation}
		\frac{d\varepsilon}{d\omega}=4\pi\frac{d\chi'}{d\omega}>0
	\end{equation}
	и область аномальной дисперсии в интервале  
	\begin{equation}
		\omega_{0}-\Gamma<\omega<\omega_{0}+\Gamma.
	\end{equation}
	В области аномальной дисперсии велико по абсолютной величине значение $\chi''(\omega)$, т.е. в среде имеется большое поглощение энергии.
	
	Кроме того, из предыдущего примера ясно, что дисперсия и диссипация всегда существуют совместно. Имеются, правда, области частот, где превалирует одно или другое.
	
	\section*{Соотношения Крамерса-Кронига}
	Покажем, что по известному  $\chi'(\omega)$  можно найти $\chi''(\omega)$  , и наоборот. Проинтегрируем $\tilde{\chi}(\tilde{\omega})/(\tilde{\omega}-\Omega)$  в комплексной плоскости $\tilde{\omega}$ по контуру, указанному на рис. 4В. Из  формулы \eqref{22}  видно, что  $\tilde{\chi}(\tilde{\omega})$ имеет два полюса в точках  $(\tilde{\omega})_{1,2}=i\Gamma\pm\sqrt{\omega_{0}^{2}-\Gamma^{2}}$  в верхней полуплоскости (ибо $\omega_{0}^{2}>>\Gamma^{2}$). Кроме того, функция  $\tilde{\chi}(\tilde{\omega})/(\tilde{\omega}-\Omega)$ имеет  полюс в точке $\tilde{\omega}=\Omega$ на  действительной оси. Поэтому в результате обхода по выбранному контуру получим 
	\begin{equation}
		\oint\frac{\tilde{\chi}(\tilde{\omega})}{\tilde{\omega}-\Omega}d\tilde{\omega}=\int\limits_{-R}^{\Omega-\nu}\frac{\chi(\tilde{\omega})d\omega}{\omega-\Omega}+\int\limits_{\Omega+\nu}^{R}\frac{\tilde{\chi}(\omega)d\omega}{-\Omega+\omega}+\int\limits_{L_{\infty}}\frac{\tilde{\chi}(\tilde{\omega})d\tilde{\omega}}{\tilde{\omega}-\Omega}+\int\limits_{L_{\nu}}\frac{\tilde{\chi}(\tilde{\omega})d\tilde{\omega}}{\tilde{\omega}-\Omega}=0
	\end{equation}
	
	Исследуя эти четыре интеграла, учтем, что $\lim\limits_{R=L_{\infty}\rightarrow\infty}\int\limits_{L_{\nu}}\dfrac{\tilde{\chi}(\tilde{\omega})d\tilde{\omega}}{\tilde{\omega}-\Omega}=0$ и что последний интеграл легко находится при $\nu\rightarrow0$:
	$$\lim\limits_{\nu\rightarrow0}\int\limits_{\pi}^{2\pi}\frac{\tilde{\chi}(\Omega+\nu e^{i\theta})}{\nu e^{i\theta}}i\nu e^{i\theta}d\theta=i\pi\tilde{\chi}(\Omega)$$
	\begin{figure}[H]
		содержимое...\caption{}		
	\end{figure}
	В этом предельном случае два других интеграла образуют один интеграл в смысле главного значения. В результате мы получим формулу 
	\begin{equation}
		\tilde{\chi}(\Omega)=\frac{i}{\pi}\int\limits_{-\infty}^{\infty}\frac{\tilde{\chi}(\omega)d\omega}{\omega-\Omega}.
	\end{equation}
	Подставляя $\tilde{\chi}(\omega)=\chi'(\omega)+i\chi''(\omega)$, получим из равенства реальных и мнимых частей соотношения  Крамерса-Кронига 
	\begin{equation}
		\chi'(\Omega)=\frac{-1}{\pi}\int\limits_{-\infty}^{\infty}\frac{\chi''(\omega)d\omega}{\omega-\Omega},\qquad
		\chi''(\Omega)=\frac{1}{\pi}\int\limits_{-\infty}^{\infty}\frac{\chi'(\omega)d\omega}{\omega-\Omega}
	\end{equation}
	Эти соотношения представляют собой мощный инструмент для исследования свойств различных сред.
	
	\textbf{Помимо дисперсии при распространении волн в нелинейной среде важную роль играет поглощение (диссипация) энергии. Диссипация, уменьшая интенсивность волны в среде, уменьшает тем самым воздействие нелинейности. Поэтому при достаточно сильной диссипации поле может затухнуть скорее, чем заметно изменится его спектральный состав или пространственная структура.}
	\section*{Типы волн в разных средах}
	Какие типы волны могут существовать в разных средах? 
	\begin{itemize}
		\item[I.]\textbf{Линейная среда без дисперсии}(в частности, вакуум).
		
		В ней все гармоники распространяются с одной скоростью  $c$. Если все волны всех частот  распространяются  в $+z$-направлении, то поле имеет вид $$\vec{E}(z,t)=\int\limits_{\infty}^{\infty}\vec{e}_{0}\tilde{E}(\omega)exp\left\{i\left(\omega t-\frac{\omega}{c}z\right)\right\}d\omega=\vec{e}_{0}E(z-ct),$$
		который называется \underline{простой стационарной волной}.
		\item[II.] В \textbf{линейной среде с дисперсией} ничего подобного не получится. Там никогда не будет решения в виде \underline{простой} волны. 
		\item[III.] В \textbf{нелинейной среде без дисперсии} может быть решение в виде \underline{простой} волны  
		$$\vec{E}(z,t)=\vec{e}_{0}E(z-V_{|E|}t).$$
		Но это решение не сохраняет по мере распространения свою форму, ибо $V=V_{|E|}$. Поэтому такое решение не является \underline{стационарной} волной. Подобные волны постепенно превращаются в ударные и меняют свою форму очень сильно. 
		\item[IV.] В \textbf{нелинейной среде с дисперсией} при некоторых условиях возможны волны, имеющие постоянную форму и перемещающиеся с постоянной скоростью. Эти стационарные волны 
		$$E(z,t)=E(z-\={V}\cdot t)$$
		называются \textbf{\textit{солитонами}} (уединенными волнами). У них всегда вполне определенная (для данной среды) форма и постоянная скорость $\={V}$, величина которой определяется заключенной в волне энергией и некоторыми другими характеристиками среды. Эти волны – <<продукт равновесия в борьбе нелинейности и дисперсии>>. 
	\end{itemize}
	В \textbf{нелинейной среде с дисперсией} могут существовать многосолитонные решения (или волны). Эта терминология зависит от постановки задачи (математической или физической). 
	
	\textbf{\Huge Конец введения.}
	\newpage
	\section*{Часть 1. Нелинейная оптика}
	\subsection*{Раздел 2. Трехчастотные взаимодействия в квадратичной среде}
	\subsubsection*{Введение}
	
	Решение всех задач в нелинейной оптике начинается с решения уравнений Максвелла, которые в наиболее распространённой форме для нелинейной среды без источников могут быть представлены в виде уравнения 
	\begin{equation*}
		rotrot\vec{E}+\frac{\mu}{c^{2}}\frac{\partial^{2}\left(\varepsilon\vec{E}\right)}{\partial t^{2}}=-\frac{4\pi}{c^{2}}\mu\frac{\partial^{2}\vec{P}^{NL}}{\partial t^{2}}\tag{В.4'}
	\end{equation*}
	\setcounter{equation}{0}
	Если в квадратичной  среде, т.е. среде с нелинейной поляризацией в виде 
	\begin{equation}
		\vec{P}^{NL}=\left(\hat{\chi}^{NL}\cdot\vec{E}\vec{E}\right)
		\label{1}
	\end{equation}
	задано поле в виде двух бегущих волн разных частот
	\begin{equation}
		\vec{E}=\frac{1}{2}\tilde{\vec{e}}_{1}^{0}\tilde{E}_{1}(\vec{r})exp\left\{i\left[\omega_{1}t-\left(\vec{k}_{1}\cdot\vec{r}\right)\right]\right\}+\frac{1}{2}\tilde{\vec{e}}_{2}^{0}\tilde{E}_{2}(\vec{r})exp\left\{i\left[\omega_{2}t-\left(\vec{k}_{2}\cdot\vec{r}\right)\right]\right\}+\text{\textit{к.с.}},
	\end{equation}
	то одна из  нелинейных поляризаций возникает на частоте 
	\begin{equation}
		\omega_{3}=\omega_{1}+\omega_{2}
	\end{equation}
	Её амплитуда определяется выражением
	\begin{equation}
		\vec{P}^{NL}_{\omega_{3}}=\left(\hat{\chi}^{NL}(\omega_{1},\omega_{2},\omega_{3})\cdot\hat{\vec{e}}_{1}^{0}\hat{\vec{e}}_{2}^{0}\right)\hat{E}_{1}\hat{E}_{2}exp\left\{i(\omega_{1}+\omega_{2})t-i\left(\left[\vec{k}_{1}+\vec{k}_{2}\right]\cdot\vec{r}\right)\right\} .
	\end{equation}
	Пусть направления $\vec{k}_{1,2}$ выбраны так, что скорость перемещения фазового фронта волны поляризации
	\begin{equation}
		V_{f}=\left(\omega_{3}/\left|\vec{k}_{1}+\vec{k}_{2}\right|\right)
	\end{equation}
	в направлении $\left\{\left(\vec{k}_{1}+\vec{k}_{2}\right)/\left|\vec{k}_{1}+\vec{k}_{2}\right|\right\}=\vec{k}^{0}_{P}$ близка к фазовой скорости волны на частоте $\omega_{3}$ в направлении $\vec{k}_{3}$. Тогда под влиянием наведённой поляризации $\vec{P}^{NL}_{\omega_{3}}$ (плотности тока $i\omega_{3}\vec{P}^{NL}_{3}$ на частоте $\omega_{3}$ )  в среде возникает поле 
	\begin{equation}
		\vec{E}_{3}=\frac{1}{2}\tilde{\vec{e}}_{3}^{0}\tilde{E}_{3}(\vec{r})exp\left\{i\left[\omega_{3}t-\left(\vec{k}_{3}\cdot\vec{r}\right)\right]\right\}+\text{\textit{к.с.}}
	\end{equation}

	Если $$\left|\Delta\vec{k}\right|=\left|\vec{k}_{3}-\vec{k}_{1}-\vec{k}_{2}\right|\rightarrow0,$$
	то оно будет расти, по крайней мере, на начальном участке (где поля $\vec{E}_{1,2}$ сильны, а оно слабо). Если считать, что направлением наилучшего синхронизма является  $z$-направление, то амплитуда поля $\tilde{E}_{3}(z)$ будет изменяться по закону 
	\begin{equation}
		\frac{d\tilde{E}_{3}}{dz}\cong\frac{1}{2ik_{3}}\cdot\frac{\alpha_{3}}{2\varepsilon_{3}}\cdot k_{3}^{2}\tilde{E}_{1}(z)\tilde{E}_{2}(z)exp(i\Delta k_{z}z),\label{7.2}
	\end{equation}
	где
	\begin{equation}
		\Delta k_{z}=\left(\left[\vec{k}_{3}-\vec{k}_{1}-\vec{k}_{2}\right]\cdot\vec{z}_{0}\right);\quad\alpha_{3}=\left(\tilde{\vec{e}}_{3}^{0*}\cdot\hat{\chi}^{NL}\cdot\tilde{\vec{e}}_{1}^{0}\cdot\tilde{\vec{e}}_{2}^{0}\right)\cdot4\pi;\quad k_{3}^{2}=\frac{\varepsilon_{3}\mu_{3}\omega_{3}^{2}}{c^{2}}.\label{7.1}
	\end{equation}
	Уравнение \eqref{7.1} аналогично уравнению \eqref{v.13}, описывающему изменение медленной амплитуды поля третьей гармоники в кубичной среде. Некоторые различия правых частей этих уравнений (в частности, на фактор $1/2$) обусловлены тем, что нелинейные поляризации определяются разными выражениями.  
	
	При точном синхронизме поле $\tilde{E}_{3}(z)$ описывается уравнением 
	\begin{equation}
		\frac{d\tilde{E}_{3}}{dz}=\frac{\alpha_{3}k_{3}}{4i\varepsilon_{3}}\tilde{E}_{1}(z)\tilde{E}_{2}(z).\label{9}
	\end{equation}
	В приближении заданных полей накачки $\tilde{E}_{1,2}\cong\tilde{E}_{1,2}(0)$ оно растет линейно с ростом расстояния от границы среды 
	\begin{equation}
		\tilde{E}_{3}(z)\cong-i\frac{\alpha_{3}k_{3}}{4\varepsilon_{3}}\tilde{E}_{1}(0)\tilde{E}_{2}(0)z.
	\end{equation}
 	Если поля $\tilde{E}_{1,2}(0)$ считать примерно равными $\left|\tilde{E}_{2}(0)\right|\cong\left|\tilde{E}_{1}(0)\right|\equiv\left|\tilde{E}(0)\right|$, то на длине
 	\begin{equation}
 		L_{0}=\left(\frac{\alpha_{3}k_{3}}{4\varepsilon_{3}}\left|\tilde{E}(0)\right|\right)^{-1}
 	\end{equation}
 	поле $\tilde{E}_{3}(z)$ сравнивается по величине с полем $\tilde{E}(0)$. Это расстояние называется характерной длиной \textbf{\textit{нелинейного взаимодействия}} в квадратичной среде.
 	
 	В отсутствие синхронизма и в приближении \textit{\textbf{не истощающейся}} накачки $\tilde{E}_{1,2}(z)\cong\tilde{E}_{1,2}(0)$ общее решение \eqref{7.2} в виде
 	\begin{equation}
 		\left|\tilde{E}_{3}(z)\right|^{2}=\left|-\frac{\alpha_{3}k_{3}\tilde{E}_{1}\tilde{E}_{2}}{2(\Delta k_{z})\varepsilon_{3}}exp\left\{i\frac{1}{2}\Delta k_{z}z\right\}\sin\left\{\frac{1}{2}\Delta k_{z}z\right\}\right|^{2}
 	\end{equation}
 	осциллирует с периодом $z_{\text{П}}=2l_{k}=2\pi/\Delta k_{z}$. Нарастание поля $|\tilde{E}_{3}(z)|$ происходит только на так называемой \textit{\textbf{когерентной длине}}  
 	\begin{equation}
 		l_{k}=\pi/\Delta k_{z}=z_{\text{П}}/2.
 	\end{equation}
 	На таком же следующем расстоянии поле гармоники убывает до нуля. Если $l_{k}$ существенно короче \textbf{\textit{длины нелинейного взаимодействия}}, то именно $l_{k}$ определяет максимальное значение поля гармоники внутри слоя среды. 
 	
 	Воздействие нелинейности и дисперсии на процесс рождения  $q$-той гармоники (на формирование искажений гармонической волны) в слабо поглощающей нелинейной среде всегда сопоставляется посредством сравнения \textit{\textbf{длины нелинейного взаимодействия}}  $(L_{0})_{q}$  с соответствующей \textit{\textbf{когерентной длиной}} $(l_{k})_{q}$. В нелинейной оптике, как правило, для всех гармоник (\underline{кроме одной}) выполняется соотношение  $(L_{0})_{q}>>(l_{k})_{q}$ сильной дисперсии. В этих случаях слабо поглощающая нелинейная среда с дисперсией определяется как \textbf{\textit{слабо нелинейная}}. 
 	
 	В нелинейной оптике диссипативные и нелинейные члены всегда должны быть малыми, а сами волны мало отличающимися от волн в линейной среде. В типовых задачах нелинейной оптики решения уравнений Максвелла следует искать в виде суммы  2-х – 4-х волн, частоты которых заранее определяются из дисперсионных характеристик среды. 
 	
 	В случае взаимодействии трех волн в квадратичной среде \eqref{1} поле $\tilde{E}_{3}(z)$ будет расти достаточно эффективно при условии  
 	\begin{equation}
 		l_{k}\equiv\frac{\pi}{\Delta k_{z}}>>L_{0}\equiv\frac{4\varepsilon_{3}}{\alpha_{3}k_{3}\left|\tilde{E}(0)\right|};\quad\left|\vec{k}_{1}+\vec{k}_{2}-\vec{k}_{3}\right|<<\pi/L_{0}.
 		\label{14.1}
 	\end{equation}
 	Другими словами, в квадратичной среде три волны с частотами $\omega_{3}=\omega_{1}+\omega_{2}$ будут взаимодействовать достаточно эффективно, если их волновые векторы удовлетворяют \textit{\textbf{условию <<синхронизма>>}} \eqref{14.1}. 
 	
 	 \subsection*{Пункт 1. Условия осуществления синхронизма.} 
 	Всегда ли имеет место условие \eqref{14.1}? В каких случаях оно выполняется? 
 		
		\begin{minipage}{0.5\textwidth}
			\begin{flushleft}
				Пусть для простоты частоты $\omega_{1,2}$ будут равными $\omega_{1,2}=\omega$. Тогда из $$\vec{k}_{1}+\vec{k}_{2}=\vec{k}_{3}$$ найдём $$\left|\vec{k}_{3}\right|=\frac{2\omega}{c}\sqrt{\varepsilon(2\omega)\mu}<|\vec{k}_{1}|+|\vec{k}_{2}|=\frac{2\omega}{c}\sqrt{\varepsilon(\omega)\mu}$$
				или 
				\begin{equation}
					\varepsilon(\omega)>\varepsilon(2\omega)
					\label{15.1}
				\end{equation}
			Условие \eqref{15.1} выполнимо в средах с \textit{\textbf{аномальной дисперсией}}. В изотропной среде аномальная дисперсия имеет место только вблизи линий поглощения. Но в этих областях велика диссипация, которая подавляет нелинейные процессы. Поэтому для осуществления процесса необходима анизотропная среда. Условие \eqref{15.1} может выполняться  вдали от линий поглощения (в области прозрачности) в кристаллах двух типов:  
			\begin{equation}
				n_{2}^{e}<n_{1}^{0}\quad\text{и}\quad n_{2}^{0}<n_{1}^{e}
				\label{16.1}
			\end{equation}
			\textit{Поверхности показателей преломления} в кристаллах первого типа, к которому принадлежит кристалл $KH_{2}PO_{4}$ (KDP), представлены на Рис. 1.2. 
			\end{flushleft}
		\end{minipage}
		\begin{minipage}{0.5\textwidth}
			\begin{flushright}
				Содержимое
			\end{flushright}
		\end{minipage}
		\newpage
		(Типичным представителем кристаллов второго типа является кальцит  $CaCO_{3}$.) В кристалле первого типа можно организовать накапливающееся взаимодействие двух видов, которые символически записываются как
		\begin{align}
			1^{o}+1^{0}=2^{e}\tag{17a}\\
			1^{0}+1^{e}=2^{e}.\tag{17б}			
		\end{align}	
		Какие  когерентные взаимодействия осуществлены в квадратичной среде?  
 	\begin{itemize}
 		\item[1)]В KDP осуществлены режимы генерации суммарной и разностной частот $\omega_{3}=\omega_{1}\pm\omega_{2}$ и, в частности, генерация второй гармоники $\omega_{2}=\omega+\omega$ с помощью двух типов синхронного взаимодействия: 
 		\begin{equation}
 			\text{а)}\vec{k}_{1}^{o}(\omega)+\vec{k}_{2}^{o}(\omega)=\vec{k}_{3}^{o}(2\omega)\quad\text{б)}\quad\vec{k}_{1}^{0e}(\omega)+\vec{2}_{1}^{o}(\omega)=\vec{k}_{3}^{e}(2\omega).
 		\end{equation}
 		В случае а) фактически взаимодействуют две плоские волны. В случае б) взаимодействуют три плоские волны с разными направлениями волновых векторов. 
 		\item[б)] В KDP получена также генерация третьей гармоники в результате двух  последовательных трехчастотных взаимодействий: $$\omega+\omega=2\omega;\quad\omega+2\omega=3\omega$$
 		Это – промежуточный случай между двумя предельными: образованием второй гармоники в результате одного трёхчастотного взаимодействия и образованием ударной волны в результате бесконечно большого числа таких взаимодействий
 	\end{itemize}
 	\subsection*{Пункт 2. Описание трехчастотных (трёхволновых) взаимодействий.} 
 	В уравнения Максвелла в виде
 	\begin{equation}
 		rotrot\vec{E}+\frac{\mu\hat{\varepsilon}_{0}}{c^{2}}\frac{\partial^{2}\vec{E}}{\partial t^{2}}+\frac{4\pi\hat{\sigma}\mu}{c^{2}}\frac{\partial\vec{E}}{\partial t}=-\frac{4\pi\mu}{c^{2}}\frac{\partial^{2}\vec{P}^{NL}}{\partial t^{2}}\label{18}
 	\end{equation}
 	подставляем поле в виде трех взаимодействующих волн
 	\begin{equation}
 		\vec{E}=\frac{1}{2}\sum_{q=1}^{3}\tilde{\vec{e}}^{0}_{q}\tilde{\vec{E}}_{q}(\vec{r})exp\left\{i\left[\omega_{q}t-\left(\vec{k}_{q}\cdot\vec{r}\right)\right]\right\}+\text{\textit{к.с.}}\label{19}
 	\end{equation}
 	с медленно меняющимися на длинах $\lambda_{q}=2\pi/|\vec{k}_{q}|$ в направлениях $\vec{k}_{q}$ комплексными амплитудами. Считаем, что \eqref{19} описывает плоские волны, а не пучки, так что $\tilde{E}_{q}(\vec{r})$ слабо изменяются в направлениях, перпендикулярных $\vec{k}_{q}$.
 	
 	Из \eqref{18} после упрощений и перехода к монохроматическим компонентам получим систему из трёх связанных дифференциальных комплексных уравнений 
 	\begin{equation}
 		\begin{cases}
 		 		\left(\left[\tilde{\vec{e}}_{3}^{0*}\times\left[\vec{k}_{3}\times\tilde{\vec{e}}_{3}^{0}\right]\right]\cdot\nabla\tilde{E}_{3}\right)+\bar{\gamma}_{3}\tilde{E}_{3}+i\beta\omega_{3}^{2}exp\left\{+i\left(\Delta\vec{k}\cdot\vec{r}\right)\right\}\tilde{E}_{1}\tilde{E}_{2}=0;\\
 				\left(\left[\tilde{\vec{e}}_{1,2}^{0*}\times\left[\vec{k}_{1,2}\times\tilde{\vec{e}}_{1,2}^{0}\right]\right]\cdot\nabla\tilde{E}_{1,2}\right)+\bar{\gamma}_{1,2}\tilde{E}_{1,2}+i\beta\omega_{1,2}^{2}exp\left\{-i\left(\Delta\vec{k}\cdot\vec{r}\right)\right\}\tilde{E}_{3}\tilde{E}_{2,1}^{*}=0;
 		\end{cases}\label{20}
 	\end{equation}
 	в которой введены коэффициент нелинейного взаимодействия  
 	\begin{equation}
 		\beta=\frac{\pi\mu}{c^{2}}(\tilde{\vec{e}}_{1,2}^{0*}\cdot\hat{\chi}^{NL}\cdot\tilde{\vec{e}}_{3}^{0}\cdot\tilde{\vec{e}}_{2,1}^{0*})=\frac{\pi\mu}{c^{2}}(\vec{e}_{3}^{0*}\cdot\hat{\chi}^{NL}\cdot\tilde{\vec{e}}_{1}^{0}\cdot\tilde{\vec{e}}_{2}^{0})
 		\label{20'}\tag{20'}
 	\end{equation}
 	и коэффициенты линейного поглощения
 	\begin{equation}
 		\bar{\gamma}_{q}=\left\{\left(\tilde{\vec{e}}_{q}^{0*}\cdot\hat{\varepsilon}''(\omega_{q})\cdot\tilde{\vec{e}}_{q}^{0*}\right)\equiv\left(\tilde{\vec{e}}_{q}^{0*}\cdot\frac{4\pi\hat{\sigma}}{\omega_{q}}\cdot\tilde{\vec{e}}_{q}^{0*}\right)\right\}\cdot\frac{\mu\omega_{q}^{2}}{2c^{2}}
 		\label{20''}\tag{20''}
 	\end{equation}
 	В анизотропной среде в общем случае $\vec{E}_{q}$ и $\vec{k}_{q}$ не полностью ортогональны друг другу (Рис. 1.3.). 
 	
 	\begin{minipage}{0.5\textwidth}
 		\begin{flushleft}
 			Если вектор 
 			$$\left(\vec{k}_{q}/\left|\vec{k}_{q}\right|\right)=\vec{z}_{0}\cos\theta_{q}+\vec{x}_{0}\sin\theta_{q}$$
 			необыкновенной волны направлен под углом $\theta_{q}$ к оси  $Oz$  и тензор $\hat{\varepsilon}_{0}$ в собственных кристаллографических осях  $x, y, z$ имеет вид 
 			$\hat{\varepsilon}_{0}=
 			\begin{vmatrix}
 				 \varepsilon_{\perp}& 0& 0\\
 				0& \varepsilon_{\perp}& 0\\
 				0& 0& \varepsilon_{\Uparrow}
 			\end{vmatrix}$, то величина
 		$$k_{q}=\frac{(\omega_{q}/c)\sqrt\mu\varepsilon_{\perp}\varepsilon_{\Uparrow}}{\sqrt{\varepsilon_{\Uparrow}\cos^{2}\theta_{q}+\varepsilon_{\perp}\sin^{2}\theta_{q}}}$$ и орт электрического поля $$\vec{e}_{q}^{0}=\frac{-\vec{x}_{0}\varepsilon_{\Uparrow}\cos\theta_{q}+\vec{z}_{0}\varepsilon_{\perp}\sin\theta_{q}}{\sqrt{\varepsilon_{\Uparrow}^{2}\cos^{2}\theta_{q}+\varepsilon_{\perp}^{2}\sin^{2}\theta_{q}}}$$ зависят от его направления.
 		\end{flushleft}
 	\end{minipage}
 	\begin{minipage}{0.5\textwidth}
 		\begin{flushright}
 			содержимое...
 		\end{flushright}
 	\end{minipage}
 	
 	Вектора $\tilde{\vec{e}^{0}_{q}},	 \vec{k}_{q},\tilde{\vec{D}}_{q},   \tilde{\vec{h}^{0}_{q}}=(\vec{\tilde{H}}_{q}/\tilde{H}_{q})$ удовлетворяют соотношениям 
 	$$\left(\vec{k}_{q}\cdot\tilde{\vec{h}^{0}_{q}}\right)=0,\quad\left(\vec{k}_{q}\cdot\tilde{\vec{D}}_{q}\right)=0;\quad\left(\tilde{\vec{h}^{0}_{q}}\cdot\tilde{\vec{D}}_{q}\right)=0;$$
 	
 	$$\left[\vec{k}_{q}\times\tilde{\vec{e}^{0}_{q}}\right]\tilde{E}_{q}=\frac{\omega_{q}}{c}\mu\tilde{H}_{q}\tilde{\vec{h}^{0}_{q}}=\-\tilde{y}^{0}\cdot\frac{\varepsilon_{\Uparrow}\cdot\cos^{2}\theta_{q}+\varepsilon_{\perp}\cdot\sin^{2}\theta_{q}}{\sqrt{\varepsilon_{\Uparrow}^{2}\cos^{2}\theta_{q}+\varepsilon_{\perp}^{2}\sin^{2}\theta_{q}}}k_{q}\tilde{E}_{q},$$
 	
 	$$\left[\vec{k}_{q}^{0}\times\tilde{\vec{e}^{0}_{q}}\right]=-\vec{y}^{0}\cdot\frac{\varepsilon_{\Uparrow}\cdot\cos^{2}\theta_{q}+\varepsilon_{\perp}\cdot\sin^{2}\theta_{q}}{\sqrt{\varepsilon_{\Uparrow}^{2}\cos^{2}\theta_{q}+\varepsilon_{\perp}^{2}\sin^{2}\theta_{q}}};$$
 	
 	$$\vec{s}_{q}^{0}\overline{S_{q}}^{T}\equiv\overline{\vec{S}_{q}}^{T}=\frac{c}{8\pi}\textbf{Re}\left[\vec{\tilde{E}}_{q}\times\vec{\tilde{H}}_{q}^{*}\right]=\frac{c\tilde{E}_{q}\tilde{H}_{q}^{*}}{8\pi}\left[\tilde{\vec{e}^{0}_{q}}\times\tilde{\vec{h}^{0*}_{q}}\right].$$
 	
 	В приближении плоской волны $(\partial\tilde{E}_{}/\partial x_{q})=(\partial\tilde{E}_{q}/\partial y_{q})=0$ которое не учитывает изменений $\nabla\tilde{E}_{q}$ в направлениях, перпендикулярных $\vec{k}_{q}$, получаем 
 	$$\left(\left[\tilde{\vec{e}^{0*}_{q}}\times\left[\vec{k}_{q}\times\tilde{\vec{e}^{0}_{q}}\right]\right]\cdot\nabla\tilde{E}_{q}\right)=k_{q}\cos\alpha_{q}\cdot\left(\vec{s}_{q}^{0}\cdot\nabla\tilde{E}_{q}\right)\equiv\frac{\omega_{q}}{c}n_{q}\cos^{2}\alpha_{q}\left(\frac{\partial\tilde{E}_{q}}{\partial z_{q}}\right),$$
 	где оси $Oz_{q}$  выбраны по направлениям $\vec{k}_{q}\left(\vec{k}_{q}\uparrow\uparrow\vec{z}_{q}^{0}\right)$
 	
 	Если справедливо условие $z_{1}=z_{2}=z_{3}\equiv z,$ то \eqref{20} преобразуется в систему уравнений в обыкновенных производных
 	\begin{equation} 	
 		\begin{aligned}
 			&\frac{\omega_{1,2}}{c}\hat{n}_{1,2}\frac{d\tilde{E}_{1,2}}{dz}+\bar{\gamma}_{1,2}\tilde{E}_{1,2}+i\beta\omega_{1,2}^{2}exp\left\{-i\psi-i\left(\Delta k_{z}\cdot z\right)\right\}\tilde{E}_{3}\tilde{E}_{2,1}^{*}=0;\\
 			&\frac{\omega_{3}}{c}\hat{n}_{3}\frac{d\tilde{E}_{3}}{dz}+\bar{\gamma}_{3}\tilde{E}_{3}+i\beta\omega_{3}^{2}exp\left\{i\psi+i\left(\Delta k_{z}\cdot z\right)\right\}\tilde{E}_{1}\tilde{E}_{2}=0,
 		\end{aligned}\tag{20'''}\label{20'''}
 	\end{equation}
 	где $\hat{n}_{q}=\sqrt{\varepsilon_{q}\mu_{q}}\cos\alpha_{q}\left(\vec{s}_{q}^{0}\cdot\vec{z}_{q}^{0}\right)=n_{q}\cos^{2}\alpha_{q},\quad\psi\equiv\psi(x,y)=x\cdot\Delta k_{x}+y\cdot\Delta k_{y}.$
 	
 	Введём $+Z$–составляющую вектора Пойнтинга всех трех волн 
 	$$S_{z}=S_{1z}+S-{2z}+S_{3z}=\frac{c}{8\pi\mu}\cdot\left\{\hat{n}_{1}\left|\hat{E}_{1}\right|^{2}+\hat{n}_{2}\left|\hat{E}_{2}\right|^{2}+\hat{n}_{3}\left|\hat{E}_{3}\right|^{2}\right\}.$$
 
 	Введем безразмерную интенсивность и число $m_{q}^{2}$ фотонов $q$-той волны 
 	\begin{equation}
 		\omega_{q}m_{q}^{2}=\frac{c}{8\pi\mu}(\hat{n}_{q}\left|\hat{E}_{q}\right|^{2}/S_{z})
 		\label{21.1}
 	\end{equation}
 	имеющее размерность времени, а также координату
 	\begin{equation}
 		\zeta=(z/L_{0})=z\beta\sqrt{8\pi\mu S_{z}c\frac{\omega_{1}\omega_{2}\omega_{3}}{\hat{n}_{1}\hat{n}_{2}\hat{n}_{3}}}
 		\tag{21'}\label{21'}
 	\end{equation}
 	нормированную на длину \textit{\textbf{нелинейного взаимодействия}} $L_{0}$. Тогда уравнения преобразуются в систему из 4-х  (а не 6-ти) действительных связанных уравнений 
 	\begin{equation}
 		\begin{aligned}
 			\frac{dm_{1,2}}{d\zeta}+\gamma_{1,2}m_{1,2}=-m_{2,1}m_{3}\sin\Phi;\\
 			\frac{dm_{3}}{d\zeta}+\gamma_{3}m_{3}=-m_{2}m_{1}\sin\Phi;\\
 			\frac{d\Phi}{d\zeta}=\delta+(ctg\Phi)\frac{d}{d\zeta}ln(m_{1}m_{2}m_{3})
 		\end{aligned}\label{22.1}
 	\end{equation}
 	где 
 	$$\Phi=arg\tilde{E}_{1}+\tilde{E}_{2}-arg\tilde{E}_{3}+\xi(x,y)+\Delta k_{z}\cdot z,$$
 	$$\delta=(\Delta k_{z}\cdot L_{0}),\qquad\gamma_{q}=\frac{\overline{\gamma}_{q}L_{0}}{\omega_{q}\hat{n}_{q}}.$$
 	\subsection*{Пункт 3. Законы сохранения в непоглощающей среде} 
 	
 	 Проанализируем систему \eqref{22.1} и ее решения в приближении
 		\begin{equation}
 			\gamma_{1}=\gamma_{2}=\gamma_{3}=0\label{23}
 		\end{equation} 		
 	отсутствия поглощения. Умножим каждое уравнение для $m_{q}$ на $2m_{q}$ и  заметим, что справедливы соотношения 
 	$$\frac{d}{d\zeta}(m_{1}^{2}+m_{3}^{2})=\frac{d}{d\zeta}(m_{2}^{2}+m_{3}^{2})=\frac{d}{d\zeta}(m_{1}^{2}-m_{2}^{2}),$$
 	которые позволяют получить законы Мэнли- Роу сохранения чисел квантов:
 	\begin{equation}
 		m_{1}^{2}+m_{3}^{2}=N_{1};\quad m_{2}^{2}+m_{3}^{2}=N_{2};\quad m_{1}^{2}-m_{2}^{2}=N_{1}-N_{2}.
 		\label{24}
 	\end{equation}
 	Видно, что из трех законов \eqref{24} независимы только два. Интерпретация такова: система \eqref{22.1} описывает процессы слияния и распада квантов. 
 	
 	Кроме \eqref{24}, имеются еще 2 независимых закона сохранения. Один из них – закон сохранения энергии (вектора Пойнтинга $S_{z}$). Для его определения умножим каждое уравнение на $\omega_{q}2m_{q}$ и сложим их. Тогда получим 
 	$$\frac{d}{d\zeta}(\omega_{1}m_{1}^{2}+\omega_{2}m_{2}^{2}+\omega_{3}m_{3}^{2})$$
 	и с учетом \eqref{21.1}
 	\begin{equation}
 		\omega_{1}m_{1}^{2}+\omega_{2}m_{2}^{2}+\omega_{3}m_{3}^{2}=const=1.
 	\end{equation}
 	Последнее уравнение \eqref{22.1} можно проинтегрировать в два этапа. Вначале находится первый интеграл для частного случая $\delta=0$ в виде 
 	\begin{equation}
 		m_{1}m_{2}m_{3}\cos\Phi=\Gamma(0)\equiv\Gamma_{0}.\tag{26'}\label{26'}
 	\end{equation}
 	Затем методом вариации произвольной постоянной $$m_{1}m_{2}m_{3}\cos\Phi+f(\zeta)=\Gamma$$ находится первый интеграл в виде
 	\begin{equation}
 		m_{1}m_{2}m_{3}\cos\Phi+\frac{1}{2}\delta m_{3}^{2}=\Gamma.\label{26}
 	\end{equation}
 	\subsection*{Пункт 4. Генерация второй гармоники по схеме $1^{o}+1^{o}=2^{e}$} 
 	В этом частном случае
 	\begin{equation}
 		m_{1}=m_{2}\equiv m,\quad\omega_{1}=\omega_{2}=\omega,\quad\omega_{3}=2\omega\label{27}
 	\end{equation}
 	уравнения \eqref{22.1} превращаются в систему
 	\begin{equation}
 		\begin{aligned}
 			&\frac{dm}{d\zeta}+\gamma m=-mv\sin\Phi;\\
 			&\frac{dv}{d\zeta}+\gamma_{3}v=m^{2}\sin\Phi;\\
 			&\frac{d\Phi}{d\zeta}=\delta+ctg\Phi\frac{d}{d\zeta}ln(m^{2}v),\label{28}
 		\end{aligned}
 	\end{equation}
 	в которой для удобства введено $m_{3}\equiv v$. В случае отсутствия диссипации 
 	\begin{equation}
 		\gamma=\gamma_{3}=0\tag{23'}\label{23'}
 	\end{equation}
 	система \eqref{28} имеет первые интегралы  
 	\begin{align}
 		m^{2}+v^{2}&=N;\tag{29а}\label{29a}\\
 		m^{2}v\cos\Phi+\frac{1}{2}\delta v^{2}&=\overline{\Gamma}+\frac{1}{2}\delta v_{0}^{2}\equiv\Gamma;\tag{29b}\label{29b}\\
 		2\omega(m^{2}+v^{2})&=1.\tag{29c}\label{29c}
 	\end{align}
 	\setcounter{equation}{29}
 	Рассмотрим различные решения этих уравнений.
 	\setcounter{section}{4}
 	\subsection{Случай точного синхронизма}
 		\begin{equation}
 			\delta=0\label{30}
 		\end{equation}
 	\subsubsection{Вторая гармоника отсутствует на входе слоя}
 		 Вначале найдём решение, которое учитывает влияние истощения первой гармоники\footnote{В приближении заданного поля первой гармоники $m\equiv m_{0}=const$ решение \eqref{28} которое описывает рост второй гармоники на начальном участке среды, находим в виде $v(\zeta)=m_{0}^{2}\cdot\zeta=(m_{0}^{2}/z_{0})z\equiv m_{0}(z/L_{0})$, где использована оценочная  длина нелинейного взаимодействия $L_{0}=(z_{0}/m_{0})=(\hat{n}_{1}/\sqrt{2}\beta c\omega\left|\tilde{E}_{1}(0)\right|)$.}, \textit{\textbf{при условии отсутствия второй гармоники на входе слоя }}
 		\begin{equation}
 			v(0)\equiv v_{0}=0\label{31}
 		\end{equation}
	Из \eqref{29b} в этом случае находим $\Gamma=0$ и далее из физического смысла задачи (роста второй гармоники) получаем 
	\begin{equation}
		\cos\Phi=0;\qquad\sin\Phi=1\label{32}
	\end{equation}
	Затем используем первый интеграл \eqref{29a}, и второе уравнение \eqref{28} приобретает вид
	
	\begin{equation}
		\frac{d}{d(\zeta\sqrt{N})}\left(\frac{v}{\sqrt{N}}\right)^{-1}=1-\left(\frac{v}{\sqrt{N}}\right)^{2}\equiv1-U.\label{33}
	\end{equation}
	С граничным условием \eqref{31} его решение 
	\begin{equation}
		(v/\sqrt{N})^{2}\equiv U(\zeta)=th^{2}(\zeta\sqrt{N})\equiv th^{2}(\xi),\tag{34v}\label{34v}
	\end{equation}
	а также полученное из \eqref{29a} решение 
	\begin{equation}
		\left[m(\zeta)/\sqrt{N}\right]^{2}=1-th^{2}(\xi)=1/ch^{2}(\xi)\tag{34m}\label{34m}
	\end{equation}
	для квадрата амплитуды первой гармоники изображены на Рис. 1.4.
	 
	\vspace{5cm}
	
	Решение \eqref{34v} свидетельствует о возможности полной перекачки энергии во вторую гармонику на бесконечно большой длине слоя среды. 
	\subsubsection{Вторая гармоника на границе присутствует}
	Уравнения \eqref{28} при условиях\eqref{23'} и \eqref{30} можно проинтегрировать и в случае \textit{\textbf{наличия на границе $\zeta=0$ второй гармоники }}
	\setcounter{equation}{34}
	\begin{equation}
		v(0)=v_{0},\qquad m(0)=m_{0}.\label{35}
	\end{equation}
	Результат интегрирования зависит от величины первого интеграла \eqref{29b}, т.е. от решения третьего уравнения \eqref{28}. 
	
	Если $\bar{\Gamma}=0$ (и соответственно $\cos\Phi=0$, а $\sin\Phi=\pm1$), то для $v_{0}<1$ возможны два решения \eqref{28}, одно из которых описывает нарастание $v$ по изображённому на рис. 1.4 закону, а второе – уменьшение $v$. Координата $\zeta_{0}$ определяется из условия $$v_{0}=\sqrt{N}th(\zeta_{0}\sqrt{N})\quad\iff\quad\sqrt{U_{0}}=th(\xi_{0}).$$
	
	Решение \eqref{28} в произвольном случае $m_{0}^{2}v_{0}\cos\Phi_{0}\equiv\Gamma(\delta=0)\equiv\Gamma$ также существует. В этом случае из первых интегралов \eqref{29a}-\eqref{29c} находятся 
	$$m=\sqrt{N-v^{2}}\qquad\text{и}\qquad\sin\Phi=\pm\sqrt{1-\Gamma^{2}/(N-v^{2})v^{2}},$$ и получается уравнение для второй гармоники 
	\begin{equation}
		\frac{dU}{d\xi}=\pm2\sqrt{(1-U)^{2}U-\alpha}\label{36}
	\end{equation}
	где величина $U=v^{2}/N$ положительна и меньше единицы и где положительна величина $\alpha=(\Gamma^{2}/N^{3})$. Знак указывает на направление, в котором идет процесс. Если знак положительный, то вторая гармоника растёт. 
	
	Дифференциальное уравнение \eqref{36} можно представить в интегральной форме в виде эллиптического интеграла 
	\begin{equation}
		2\xi=\int\limits_{U(0)}^{U}\frac{dU}{\sqrt{(1-U)^{2}U-\alpha}},\label{37}
	\end{equation}
	Если $\alpha<(\rfrac{4}{27})$, то кубическое уравнение
	\begin{equation}
		(1-U)^{2}U-\alpha=0\label{38}
	\end{equation}
	может иметь три положительных корня $0<U_{a}<U_{b}<U_{c}$. В этом случае решение \eqref{37} может меняться между двумя наименьшими положительными корнями $U_{a}<U_{b}$. Оно колеблется между ними с периодом 
	\begin{equation}
		\xi_{\text{П}}=\frac{\sqrt{N}L_{\text{П}}}{L_{0}}=\int\limits_{U_{a}}^{U_{B}}\frac{dU}{\sqrt{(1-U)^{2}U-\alpha}}.\label{39}
	\end{equation}
	Эллиптический интеграл \eqref{37} с помощью замены переменной 
	\begin{equation}
		t^{2}=\left\{\left(U-U_{a}\right)/\left(U_{b}-U_{a}\right)\right\}\label{40}
	\end{equation}
	и введение модуля
	\begin{equation}
		k^{2}=\frac{U_{b}^{2}-U_{a}^{2}}{U_{c}^{2}-U_{a}^{2}}\equiv\left\{\left(U_{b}-U_{a}\right)/\left(U_{c}-U_{a}\right)\right\}\label{41}
	\end{equation}
	приводится к явному виду эллиптического интеграла
	\begin{equation}
		\xi=\frac{1}{\sqrt{U_{c}-U_{a}}}\cdot\int\limits_{t(0)}^{t(\xi)}\frac{dt}{\sqrt{(1-t^{2})(1-k^{2}t^{2})}}\label{42}
	\end{equation}
	Решением интегрального уравнения \eqref{42} является \textit{\textbf{эллиптический синус}} 
	\begin{equation}
		t(\xi,k)=sn\left\{\left(\xi+\xi_{0}\right)\cdot\sqrt{U_{c}-U_{a}},k\right\}\equiv sn(\theta+\theta_{0},k)
	\end{equation}
	Нормированные интенсивности первой и второй гармоник (числа фотонов)  можно выразить через эту \textit{\textbf{эллиптическую функцию Якоби}} 
	\begin{equation}
		\begin{cases}
			U=U_{a}+(U_{b}-U_{a})sn^{2}\left[\sqrt{U_{c}-U_{a}}(\xi+\xi_{0}),k\right];\\
			(m^{2}/N)=1-U,\
		\end{cases}\label{44}
	\end{equation}
	Результаты проведенного анализа представлены на Рис. 1.5.  и Рис.1.6
	\vspace{4cm}
	
	Площадь, ограниченная кривыми $U_{max}$ и $U_{min}$ на Рис. 1.6, определяет интервалы изменения этих величин. Величина $\alpha$ не может быть более $\rfrac{4}{27}\cong0,148$. При $\alpha_{CR}= \rfrac{4}{27}$ величина $U$ перестаёт изменяться и становится равной $U=\overline{U}=\rfrac{1}{3}$, значения чисел фотонов $\overline{m}^{2}$, $\overline{v}^{2}$  оказываются неизменными. 
	
	Таким образом, \textbf{из-за наличия второй гармоники на границе среды уменьшается коэффициент преобразования во вторую гармонику}. Процесс становится периодическим, и максимального значения вторая гармоника достигает на конечной длине, равной половине пространственного периода \eqref{39}. 
	\subsection{Отсутствие синхронизма}
	В отсутствие синхронизма (при рассогласовании фазовых скоростей) при наличии расстройки волновых векторов
	\begin{equation}
		\delta\neq0\label{45}
	\end{equation}
	справедлив первый интеграл \eqref{26} в своем наиболее общем  виде\footnote{Решение последнего уравнения \eqref{22.1} ищем в виде $m^{2}v\cos\Phi+f=\Gamma$. Находим $\cos\Phi\frac{d}{d\zeta}(u^{2}\nu)-(m^{2}v\sin\Phi)(\rfrac{d\Phi}{d\zeta})+(\rfrac{df}{d\zeta})=0$ и $(\rfrac{df}{d\zeta})=\delta(m^{2}v)\sin\Phi.$ Отсюда получаем $f(\zeta)=\delta\int\limits_{0}^{\zeta}m^{2}v\sin\Phi d\zeta=\delta\int\limits_{0}^{\zeta}v(\rfrac{dv}{d\zeta})d\zeta=\frac{\delta}{2}(v^{2}-v_{0}^{2})$}. Обобщением  уравнения \eqref{37} в этом случае является уравнение 
	\begin{equation}
		2\xi=\int\limits_{U(0)}^{U}\frac{dU}{\sqrt{(1-U)^{2}U-\left[\sqrt{\alpha}-(\delta/2)\cdot(U-U_{0})\right]^{2}}}.\tag{37*}\label{37*}
	\end{equation}
	По прежнему уравнение 
	\begin{equation}
		(1-U)^{2}U-\left[\sqrt{\alpha}-(\delta/2)\cdot(U-U_{0})\right]^{2}=0\tag{38*}\label{38*}
	\end{equation}
	определяющее особенности подынтегрального выражения \eqref{37*}, имеет три положительных корня $0<U_{a}<U_{b}<U_{c}$ в области малых значений параметров $\alpha,\delta,U_{0}$. Заменой \eqref{40} интеграл \eqref{37*} приводится к виду \eqref{42}. Все выводы о характере изменения $m^{2}, v^{2}$, полученные в отсутствие отстройки, будут справедливы и при её наличии в случае $\delta\neq0$.
	\subsubsection{Отсутствие второй гармоники на границе слоя }
	
	В важном частном случае отсутствия второй гармоники на границе слоя 
	\begin{equation}
		v^{2}(0)\equiv U_{0}\cdot N=0\label{46}
	\end{equation}
	когда первый интеграл \eqref{29b} будет равен нулю $$\Gamma=0;\qquad\Rightarrow\qquad\alpha=0; U_{0}=0.$$
	корни уравнения \eqref{37*} (для параметра $\delta>0$) имеют вид 
	\begin{equation}
		U_{a}=0;\quad U_{b}=(\rfrac{1}{U_{c}})=\left(\sqrt{1+\frac{\delta^{2}}{16}}-\frac{\delta}{4}\right)^{2}\label{47}
	\end{equation}
	Из \eqref{47} видно, что при наличии отстройки волновых векторов не может быть полной перекачки энергии во вторую гармонику даже при прочих идеальных условиях. Перекачка будет происходить по закону \eqref{44}, в котором 
	\begin{equation}
		U_{a}=0;\quad\zeta_{0}=0;\quad k=U_{b};\quad U=U_{b}sn^{2}(\rfrac{\xi}{\sqrt{U_{b}}},U_{b}).\tag{44'}\label{44'}
	\end{equation}
	\newpage
	В параметрическом приближении решение может выглядеть ещё проще\footnote{4.2.2. Рассмотрим \textit{\textbf{параметрическое приближение роста второй гармоники}} для случая $\delta\neq0$. Воспользуемся условием $m^{2}=\overline{m}^{2}=const$  и представим второе уравнение \eqref{28} с учетом интеграла \eqref{29b} в виде $$\frac{dv^{2}}{d\zeta}=2\overline{m}^{2}v\sin\Phi\equiv\sqrt{4\overline{m}^{4}v^{2}-\left(2\Gamma-\delta\{N-\overline{m}^{2}\}\right)^{2}}\equiv2\overline{m}^{2}\sqrt{v^{2}-a^{2}}.$$
	Решение можно получить в виде $\sqrt{v^{2}-a^{2}}=\sqrt{v^{2}_{0}-a^{2}}+\overline{m}^{2}(\zeta-\zeta_{0})$ (*),  где  $v_{0}^{2}$ обозначает величину $v^{2}$ в точке $\zeta_{0}$ . Из выражения (*) находим, что поле второй гармоники растет неограниченно. \textbf{Интерпретация проста: резервуар фотонов $m^{2}$ безграничен, поэтому и $v^{2}$ растет безгранично независимо от начальных условий.}}. 
	\subsection{Влияние линейных потерь на генерацию второй гармоники }
	Это влияние можно оценить из уравнений \eqref{28} в приближении
	\begin{equation}
		\delta=0;\qquad\gamma_{1}=\gamma_{3}=\gamma.\label{48}
	\end{equation}
	В этом случае аналитическое решение (28) следует искать в виде \eqref{28}
	\begin{equation}
		\left\{m(\zeta),v(\zeta)\right\}=\left\{a_{1}(y),a_{2}(y)\right\}exp(-\gamma\zeta),\label{49}
	\end{equation}
	где роль новой координаты играет функция 
	\begin{equation}
		y=(1/\gamma)\left\{1-exp(-\gamma\zeta)\right\}.\label{50}
	\end{equation}
	Подставив \eqref{49} в \eqref{28}, мы получим уравнения
	\begin{equation}
		\frac{da_{1}}{dy}=-a_{1}a_{2}\sin\Phi;\qquad\frac{da_{2}}{dy}=a_{1}^{2}\sin\Phi;\qquad\frac{d\Phi}{dy}=ctg\Phi\frac{d}{dy}\textbf{ln}a_{1}^{2}a_{2},\label{51}
	\end{equation}
	которые совпадают по форме с уравнениями \eqref{28} при условии $\gamma=\gamma_{3}=0$. Важное отличие \eqref{51} от \eqref{28} в том, что $y$ меняется в конечных пределах $$0\equiv y(0)<y<y(\zeta=\infty)\equiv(1/\gamma).$$
	Пределы применимости \eqref{51} свидетельствуют о том, что характер нелинейного процесса в квадратичной среде определяется не абсолютной величиной длины затухания $(\rfrac{1}{\bar{\gamma}})$, а её относительным значением 
	\begin{equation}
		(\gamma)^{-1}=(\bar{\gamma}L_{0})^{-1}\label{52}
	\end{equation}
	нормированным на длину нелинейного взаимодействия. В случае <<малого затухания>> $\gamma<<1$ расстояние  $y_{\infty}=(\rfrac{1}{\gamma})$ может принимать достаточно большие значения. При этом картина изменения амплитуд $a_{1,2}$ волн первой и второй гармоник похожа на ту, которую имели изменения величин $m$ и $v$ в случае $\delta=0$. 
	
	Истинные значения $v$ или $m$  находятся по формуле \eqref{49} и при $\zeta\rightarrow\infty$ равны нулю. Поэтому $v(\zeta)$ всегда имеет экстремум в некоторой точке $\zeta_{\text{ор}}$ . На рис. 1.7. приведены зависимости нормированной амплитуды второй гармоники от нормированной координаты $\zeta$ для трёх значений постоянной затухания, нормированной на длину нелинейного взаимодействия. В случае отсутствия второй гармоники на границе слоя среды $(a_{2}^{2}(0)=0)$ точное аналитическое решение уравнений \eqref{51} (представлено на Рис. 1.7)  имеет вид  
	\vspace{3cm}
	$$a_{2}(y)=th(y)\Rightarrow\qquad v(\zeta)=th(y)\cdot \exp(-\gamma\zeta)$$
	В этом случае величину
	\begin{equation}
		v^{2}(\zeta)=\exp(-2\gamma\zeta)th^{2}\left\{(\rfrac{1}{\gamma}) [1-\exp(-\gamma\zeta)] \right\}\equiv\eta\label{53}
	\end{equation}
	можно назвать КПД преобразования первой гармоники во вторую гармонику. Используя \eqref{53}, можно оценить максимальное значение КПД по мощности волноводного удвоителя с потерями и его оптимальную длину. 
	
	В приближении малого затухания $\gamma=L_{0}\bar{\gamma}<<1$ и при условии 
	\begin{equation}
		(\gamma\zeta_{\text{ор}})<<1\label{54}
	\end{equation}
	получим $y_{\text{ор}}\cong\zeta_{\text{ор}}$ и 
	\begin{equation}
		(\gamma\zeta_{\text{ор}})\cong\left\{y/(2+\gamma)\right\}ln(4/\gamma)\cong(\gamma/2)lln(4\gamma)\label{55}
	\end{equation}
	что позволяет оценить $th^{2}\zeta_{\text{ор}}\cong1$ и максимальный КПД преобразования по формуле 
	\begin{equation}
		\eta_{max}\cong\exp(-2\gamma\zeta_{\text{ор}})th^{2}\zeta_{\text{ор}}\cong(\gamma/4)^{\gamma}\label{56}
	\end{equation}
	\underline{Выводы:}
	\begin{enumerate}
		\item Результаты преобразования во вторую гармонику ухудшаются по трём причинам: 
		\begin{enumerate}
			\item[а)] из-за рассогласования (расстройки) фазовых скоростей $\delta\neq0$; 
			\item[б)] из-за присутствия второй гармоники на границе слоя среды (при  $z = 0$); 
			\item[в)] из-за линейного поглощения.
		\end{enumerate}
		\item Наличие любой из этих трёх причин исключает возможность стопроцентного преобразования во вторую гармонику.
	\end{enumerate}
	
	\section{Трехволновые взаимодействия}
	В этот раздел попадают такие процессы, как
	\begin{enumerate}
		\item образование второй гармоники по схеме $1^{o}+1^{e}=2^{e}$;
		\item образование суммарной частоты $\omega_{3}=\omega_{2}+\omega_{1}$;
		\item образование разностной частоты $\omega_{2}=\omega_{3}-\omega_{1}$;
	\end{enumerate}
	\subsection{} В приближении \textbf{\textit{отсутствия поглощения}} \eqref{23} общее решение уравнений \eqref{22.1} с учетом первых интегралов \eqref{24}-\eqref{26} представляется как 
	\begin{equation}
		2\zeta=\int\limits_{W_{a}}^{W}\frac{dW}{\sqrt{(N_{1}-W)\cdot(N_{2}-W)\cdot W-(\Gamma-\frac{\delta}{2}W)^{2}}},\label{57}
	\end{equation}	
	где $W\equiv W(\zeta)=m^{2}_{3}(\zeta)$ см.\eqref{37*} для сравнения. Решение уравнения \eqref{57} находится точно по такой же схеме, как и решение уравнения \eqref{37*}. Вначале находятся три положительных корня $0<W_{a}<W_{b}<W_{c}$  уравнения 
	\begin{equation}
		(N_{1}-W)\cdot(N_{2}-W)\cdot W-\left(\Gamma-\frac{\delta}{2} W\right)^{2}=0,
		\label{58}
	\end{equation}
	определяющего все существенные свойства решения интегрального уравнения \eqref{57}. Заменой \eqref{40} и введением модуля эллиптического интеграла \eqref{41} интегральное уравнение \eqref{57} приводится к виду \eqref{42}. Это позволяет записать решение \eqref{57} в виде 
	\begin{equation}
		\begin{aligned}
			m_{3}^{2}(\zeta)=W_{a}+\left[W_{b}-W_{a}\right]sn^{2}\left\{\sqrt{W_{c}-W_{a}}(\zeta+\zeta_{0}),k\right\},\\
			m_{2}^{2}(\zeta)=N_{2}-m^{2}_{3}(\zeta),\qquad m_{1}^{2}(\zeta)=N_{1}-m_{3}^{2}(\zeta),
			\label{59}
		\end{aligned}
	\end{equation}
	где значение аргумента на границе слоя среды определяется из соотношения 
	\begin{equation}
		m_{2}^{2}(0)=W_{a}+\left[W_{b}-W_{a}\right]sn^{2}\left\{\sqrt{W_{c}-W_{a}}\zeta_{0},k\right\},
		\label{60}
	\end{equation}
	Процесс перекачки энергии из одной волны в другую является периодическим. Период процесса определяется (см. также \eqref{39}) как 
	\begin{equation}
		(L_{\text{П}}/L_{0})=\left[2/\sqrt{W_{c}-W_{a}}\right]K(k),\label{61}
	\end{equation}
	где 
	\begin{equation}
		K(k)=\int\limits_{0}^{1}\left\{dt/\sqrt{(1-t^{2}(1-k^{2}t^{2}))}\right\}=\int\limits_{0}^{\pi/2}\frac{d\psi}{\sqrt{1-k^{2}sin^{2}\psi}} \label{62}
	\end{equation}
	полный эллиптический интеграл 1-го рода.
	\subsubsection{Отсутствие гармоники на границе + точный синхронизм.}
	В частном случае отсутствия гармоники на границе и наличия точного синхронизма
	\begin{equation}
		m_{3}^{2}(0)=0;\qquad\delta=0
		\label{63}
	\end{equation}
	из \eqref{24} – \eqref{26} находим  $\Gamma=0, W_{a}=0, W_{b}=N_{2}, W_{c}=N_{1}, k^{2}=(N_{2}/N_{1}).$
	В результате решение \eqref{59} приобретает форму 
	\begin{equation}
		\begin{aligned}
			m_{3}^{2}(\zeta)=N_{1}sn^{2},\\
			m_{1}^{2}(\zeta)=N_{1}-m_{3}^{2}(\zeta).
		\end{aligned}\tag{59'}\label{59'}
	\end{equation}
	Процесс имеет период  
	\begin{equation}
		(L_{\text{П}}/L_{0})=[2/\sqrt{N_{1}}]K(k)
		\tag{61'}\label{61'}
	\end{equation}
	\subsection{Параметрические процессы при трёхволновом взаимодействии.}
	При 3-волновом взаимодействии понятие параметрического приближения становится более широким, чем при 2-волновом. Новая особенность взаимодействия появляется из-за того, что теперь одна волна может быть существенно больше двух других (а не одной, как это было в пункте 4.2.2.). 
	\subsubsection{Преобразовании частоты вверх.}
	Рассмотрим типичный случай параметрического приближения при так называемом преобразовании частоты вверх 
	\begin{equation}
		m_{1}^{2}(0)>>m_{2}^{2}(0),
		\label{64}
	\end{equation}
	предполагая для простоты отсутствие гармоники на границе \eqref{63} и наличие точного синхронизма $\delta=0$  (см. условие \eqref{30}). Тогда из \eqref{58} и \eqref{24} – \eqref{26} найдём 
	\begin{equation}
		W_{c}\equiv(m_{3}^{2})_{c}=N_{1}>>W_{b}\equiv(m_{3}^{2})_{b}=N_{2}>W_{a}\equiv(m_{3}^{2})_{a}=0,
		\label{65}
	\end{equation}
	и решение \eqref{59} интегрального уравнения \eqref{57} преобразуется к виду 
	\begin{equation}
		m_{3}^{2}(\zeta)=N_{2}sn^{2}\left\{\sqrt{N_{1}}\zeta,k=\sqrt{\frac{N_{2}}{N_{1}}}\right\}\cong N_{2}sin^{2}(\sqrt{N_{1}}\zeta).
		\tag{66b}
	 	\label{66_3}
	\end{equation}
	Число фотонов на частоте $\omega_{3}$  не превосходит число фотонов на частоте $\omega_{2}$, поскольку большего числа фотонов  $\omega_{3}$ образоваться не может из-за отсутствия "материала". Число фотонов на частоте $\omega_{2}$ меняется по закону
	\begin{equation}
		m_{2}^{2}(\zeta)=N_{2}cos^{2}(\sqrt{N_{1}}\zeta), \tag{66a}
		\label{66_2}
	\end{equation}
	а резервуар фотонов  $\omega_{1}$  бесконечен ($m_{1}^{2}=N_{1}=m_{1}^{2}(0)$). 
	\setcounter{equation}{66}
	\subsubsection{Параметрическое преобразование частоты вниз при высокочастотной накачке}
	Рассмотрим типичный случай образования низкочастотной гармоники при высокочастотной накачке 
	\begin{equation}
		m_{3}^{2}(0)>>m_{1}^{2}(0),
		\label{67}
	\end{equation}
	предполагая для простоты её полное \textit{\textbf{отсутствие на границе}} 
	\begin{equation}
		m_{2}^{2}(0)=0
		\label{68}
	\end{equation}
	и наличие \textit{\textbf{точного синхронизма}} $\delta=0$ (см. условие \eqref{30}). Тогда из \eqref{24}-\eqref{26} находим
	\begin{equation}
		\begin{aligned}
			\Gamma=0,\quad m_{1}^{2}(0)=N_{1}-N_{2},\quad m_{3}^{2}(0)=N_{2}>>N_{1}-N_{2},\\
			m_{1}^{2}(0)+m_{3}^{2}(0)=N_{1}>N_{2}=m_{3}^{2}(0)>>m_{1}^{2}(0)=N_{1}-N_{2}.
		\end{aligned}\label{69}
	\end{equation}
	Параметры решения \eqref{59} интегрального уравнения \eqref{57} (являющиеся решениями кубичного уравнения \eqref{58}), будут иметь значения 
	\begin{equation}
		W_{c}\equiv(m_{3}^{2})_{c}=N_{1}\geq W_{b}\equiv(m_{3}^{2})_{b}=N_{2}>>W_ {a}\equiv(m_{3}^{2})_{a}=0.
		\label{70}
	\end{equation}
	Это позволяет представить решение \eqref{59} в виде эллиптических функций 
	\begin{equation}
		\begin{aligned}
			m_{2}^{2}(\zeta)=N_{2}sn^{2}(\sqrt{N_{1}}\zeta,k);\quad m_{3}^{2}(\zeta)=N_{2}cn^{2}(\sqrt{N_{1}}\zeta,k),\\
			m_{1}^{2}(\zeta)=N_{1}-N_{2}cn^{2}(\sqrt{N_{1}}\zeta,k),
		\end{aligned}\label{71}
	\end{equation}
	модуль (второй аргумент) которых близок к единице: 
	$$k_{2}=\frac{N_{2}}{N_{1}}=1-(k')^{2}\cong1-\frac{m_{1}^{2}(0)}{m_{3}^{2}(0)}\rightarrow1$$
	В этом случае (в отличие от предыдущего) полный эллиптический интеграл первого рода \eqref{62} представляет собой достаточно большую величину 
	$$K(k)\cong ln\frac{4}{k'}+\frac{1}{4}\left[\left(ln\frac{4}{k'}-1\right)\right]+\dots\cong ln\frac{4m_{3}(0)}{m_{1}(0)}$$
	Максимальное значение мощности $m_{2}$ достигается на длине нелинейной среды $(L_{\text{П}}/2)$, которая равна половине пространственного периода изменения решения \eqref{71}, определяемого в соответствии с \eqref{61'} по упрощённой формуле 
	\begin{multline}
		L_{\text{П}}=L_{0}\frac{2}{\sqrt{N_{1}}}ln\left(\sqrt{\frac{16N_{2}}{N_{1}-N_{2}}}\right)=\\
		=\frac{\sqrt{\hat{n}_{1}\hat{n}_{2}\hat{n}_{3}}}{\beta c\sqrt{\omega_{1}\omega_{2}\omega_{3}}}\cdot\frac{ln[16N_{2}/(N_{1}N_{2})]}{\sqrt{\dfrac{\hat{n}_{1}}{\omega_{1}}|\tilde{E}_{1}(0)|^{2}+\dfrac{\hat{n}_{3}}{\omega_{3}}|\tilde{E}_{3}(0)|^{2}}}\cong\\
		\cong\frac{1}{\beta c|\tilde{E}_{3}(0)|}\sqrt{\frac{\hat{n}_{1}\hat{n}_{2}}{\omega_{1}\omega_{2}}}ln\frac{16N_{2}}{N_{1}-N_{2}}.
		\label{72}
	\end{multline}
	Интересно сравнить расстояние $L_{\text{П}}/2$ и длину нелинейного взаимодействия $L_{0}$ в процессе удвоения частоты в предположении, что в результате обоих процессов рождается одна и та же частота $\overline{\omega}$ . При этом необходимо считать, что в формуле \eqref{11} для $L_{0}$ основная частота накачки $$\omega=\frac{\overline{\omega}}{2},$$	
	а в формуле \eqref{72} рождающиеся в процессе преобразования вниз частоты $$\omega_{1}=\omega_{2}=\overline{\omega}$$ вдвое меньше частоты накачки. Кроме того, будем считать, что значения параметров в обеих формулах совпадают $$(|\tilde{E}_{3}(0)|)\equiv(|\tilde{E}(0)|\equiv(|\tilde{E}_{1}(0)|)),\quad \hat{n}_{1S}=\hat{n}_{2S}=\hat{n},\quad\beta_{S}=\beta$$ и что задана (типичная) величина отношения чисел фотонов на входе слоя $$16\frac{N_{2}}{N_{1}-N_{2}}=16\frac{m_{3}^{2}(0)}{m_{1}^{2}(0)}10^{6}.$$ Тогда отношение длин нелинейного преобразования частоты (взаимодействия) $$\frac{L_\text{П}}{2\cdot L_{0}}\cong\frac{6ln10}{2\times4\sqrt{2}}\cong1.22$$ свидетельствует о достаточно высокой эффективности процесса преобразования частоты вниз.
	
	В предельном случае $k\rightarrow1,[m_{1}^{2}(0)]m_{3}^{2}(0)\rightarrow0$ решение \eqref{71} имеет вид 
	\begin{equation}
		m_{3}^{2}(\zeta)=N_{2}/ch^{2}(\sqrt{N_{2}}\zeta),\quad m_{2}^{2}(\zeta)=N_{2}th^{2}(\sqrt{N_{2}}\zeta), m_{1}^{2}(\zeta)=N_{2}th^{2}(\sqrt{N_{2}}\zeta).\label{73} 
	\end{equation}
	Это – решение уравнений \eqref{22.1}, в которых $\Phi=-\frac{\pi}{2},\quad \cos\Phi=0, \quad\sin\Phi=-1.$  В этом случае уравнения имеют вид уравнений \textbf{\textit{трехмерного трехволнового резонансного взаимодействия}}
	\begin{equation}
		m_{1}'=m_{2}m_{3};\quad m'_{2}=m_{1}m_{3};\quad m'_{3}=-m_{1}m_{2}		
		\label{74}
	\end{equation}
	Решение \eqref{73} называется солитонным. 
	
	Итак, \textbf{запомните главное}:
	
	В основе трехчастотных взаимодействий в квадратичной среде находится процесс слияния-распада квантов. Он происходит по схеме, которая изображена на Рис.1.8. 
	
	\begin{minipage}{0.5\textwidth}
		\begin{flushleft}
			Эта схема отражает законы сохранения энергии $$\hbar\omega_{3}=\hbar\omega_{1}+\hbar\omega_{2}$$ и импульса $$\hbar\vec{k}_{3}=\hbar\vec{k}_{1}\hbar\vec{k}_{2}$$ в каждом отдельном акте. 
			
			\textbf{Наличие расстройки $|\vec{k}_{3}-\vec{k}_{1}-\vec{k}_{2}|\neq0$ не означает нарушения законов сохранения. Оно означает, что часть импульса забирается или создается через посредство границ кристалла, закрепленного на оптической скамье. }
		\end{flushleft}		
	\end{minipage}
	\begin{minipage}{0.5\textwidth}
		\begin{flushright}
			содержимое...
		\end{flushright}
	\end{minipage}
	\setcounter{equation}{0}
	\newpage
	\begin{center}
		\section*{\textbf{\huge ГЛАВА II. ЧЕТЫРЕХЧАСТОТНЫЕ ВЗАИМОДЕЙСТВИЯ 
				В КУБИЧНОЙ СРЕДЕ}}		  		
	\end{center}

	\underline{Пункт 1.} \textbf{\textit{Условия четырехчастотного взаимодействия }}
	\begin{equation}
		\vec{P}^{NL}=(\hat{\chi}\cdot\vec{E}\cdot\vec{E}\cdot\vec{E}),\label{1.3}
	\end{equation}
	аналогом соотношений (1.3) и (1.14) будут условие синхронизма и закон преобразования частот либо в виде 
	\begin{equation}
		\vec{k}_{4}=\vec{k}_{1}+\vec{k}_{2}+\vec{k}_{3}+\Delta\vec{k};\omega_{4}=\omega_{1}+\omega_{2}+\omega_{3},
		\tag{2a}\label{2a}
	\end{equation}
	либо в виде (см. Рис. 2.1)
		\begin{equation}
		\vec{k}_{4}+\vec{k}_{1}=\vec{k}_{2}+\vec{k}_{3}+\Delta\vec{k};\omega_{4}+\omega_{1}=\omega_{2}+\omega_{3}.
		\tag{2б}\label{2b}
	\end{equation}
	\setcounter{equation}{2}
	В такой среде при выполнении условий (2) осуществляется \textit{четырехчастотное} (\textbf{четырехволновое}) взаимодействие.	
	\vspace{3cm}
	
	\underline{Пункт 2.} \textbf{\textit{Основные уравнения четырехволнового взаимодействия}}
	
	Подставляя выражения типа (1.19) в уравнения Максвелла типа (1.18) с нелинейной поляризацией \eqref{1.3} в правой части и оставляя после усреднения по частоте и по длине волны уравнения первого порядка   ($\Delta_{\perp}\tilde{E}_{S}=0$), получим для случая (б) $\omega_{\alpha}+\omega_{\beta}=\omega_{S}+\omega_{f}$, $\vec{k}_{\alpha}+\vec{k}_{\beta}=\vec{k}_{S}+\vec{k}_{f}+\Delta\vec{k}$  систему уравнений, аналогичную (1.20). Она будет иметь вид 

	\begin{equation*}
		\frac{\omega_{1,4}}{c}\hat{n}_{1,4}\frac{d\tilde{E}_{1,4}}{dz_{1,4}}+\overline{\gamma}_{1,4}\tilde{E}_{1,4}+i\omega^{2}_{1,4}\left\{\beta\tilde{E}_{3}\tilde{E}_{2}\tilde{E}_{4,1}^{*}\exp\left[i\left(\Delta\vec{k}\vec{r}\right)\right]+\tilde{E}_{1,4}\sum_{j=1}^{4}(\overline{\beta}_{1,4})\tilde{E}_{j}\tilde{E}_{j}^{*}\right\}=0
	\end{equation*}
	\begin{equation}
		\frac{\omega_{2,3}}{c}\hat{n}_{2,3}\frac{d\tilde{E}_{2,3}}{dz_{2,3}}+\overline{\gamma}_{2,3}\tilde{E}_{2,3}+i\omega^{2}_{2,3}\left\{\beta\tilde{E}_{1}\tilde{E}_{4}\tilde{E}_{3,2}^{*}\exp\left[-i\left(\Delta\vec{k}\vec{r}\right)\right]+\tilde{E}_{2,3}\sum_{j=1}^{4}(\overline{\beta}_{1,4})\tilde{E}_{j}\tilde{E}_{j}^{*}\right\}=0 	\label{3}
	\end{equation}
	где $$\beta=\frac{2\pi}{4c^{2}}\mu\left(\tilde{\vec{e}}_{1}^{0*}\cdot\hat{\chi}\cdot\tilde{\vec{e}}_{2}^{0}\cdot\tilde{\vec{e}}_{3}^{0}\cdot\tilde{\vec{e}}_{4}^{0*}\right),\quad\overline{\beta}_{KS}=\frac{2\pi}{4c^{2}}\mu\left(\tilde{\vec{e}}_{k}^{0*}\cdot\hat{\chi}\tilde{\vec{e}}_{k}^{0}\cdot\cdot\tilde{\vec{e}}_{S}^{0}\cdot\tilde{\vec{e}}_{S}^{0*}\right)$$

	Наиболее существенное отличие укороченных уравнений \eqref{3} от соответствующих уравнений квадратичной среды (1.20) состоит в том, что каждое из уравнений (3) содержит не по одному, а по пять нелинейных членов. Характер нелинейных взаимодействий, описываемых разными членами, различен. Имеется "когерентное взаимодействие", как в квадратичной среде. Оно зависит от отстройки $(\Delta\vec{k}\cdot\vec{r})$ и растёт при уменьшении  $|\Delta\vec{k}|$. Имеется "некогерентное" взаимодействие (а также самовоздействие), которое связано с поправками к диэлектрической проницаемости из-за эффекта Керра. 

	Вначале введем $\tilde{E}_{s}=E_{s}exp(i\varphi_{s})$ и получим уравнения для интенсивностей 
	\begin{equation}
		\frac{\omega_{s}}{c}\hat{n}_{s}\frac{dE_{s}^{2}}{dz_{s}}+2\overline{\gamma}_{s}E_{s}^{2}\mp2\beta\omega_{s}^{2}E_{1}E_{2}E_{3}E_{4}sin\theta=0,\label{4.2}
	\end{equation}
	где знак плюс  справедлив для волн 2, 3 и  
	\begin{equation}
		\theta=(\Delta\vec{k}\cdot\vec{r})+\varphi_{2}+\varphi_{3}-\varphi_{1}-\varphi_{4}.
		\label{4.2'}\tag{4'}
	\end{equation}
	Введем безразмерные переменные 
	\begin{equation}
		\begin{aligned}
		\omega_{s}m_{s}^{2}=\hat{n}_{s}E_{s}^{2}&\left/\left[\sum_{q=1}^{4}\left(\hat{n}_{q}E_{q}^{2}\right)\right]\right.,\quad\zeta_{s}=\frac{z_{s}}{L_{0}}\equiv z_{s}\beta c\left[\sum_{q=1}^{4}\left(\hat{n}_{q}E_{q}^{2}\right)\right]\sqrt{\frac{\omega_{1}\omega_{2}\omega_{3}\omega_{4}}{\hat{n}_{1}\hat{n}_{2}\hat{n}_{2}\hat{n}_{3}\hat{n}_{4}}},\\
		&\gamma_{s}=\overline{\gamma}_{s}\sqrt{\frac{\hat{n}_{1}\hat{n}_{2}\hat{n}_{2}\hat{n}_{3}\hat{n}_{4}}{\omega_{1}\omega_{2}\omega_{3}\omega_{4}}}\left\{\beta\omega_{s}\hat{n}_{s}\left[\sum_{q=1}^{4}\left(\hat{n}_{q}E_{q}^{2}\right)\right]\right\}^{-1}\equiv\overline{\gamma}_{s}\cdot L_{0}
		\end{aligned}\label{5.2}
	\end{equation}
	в которых длина нелинейного взаимодействия $L_{0}$ обратно пропорциональна интенсивности (а не корню из этой величины, как и $L_{0}$ в (1.21)), и представим уравнения (\eqref{4.2}) в безразмерной форме   
	\begin{equation}
		\frac{d}{d\zeta_{s}}m_{s}^{2}+2\gamma_{s}m_{s}^{2}\mp2m_{1}m_{2}m_{3}m_{4}sin\theta=0.
		\label{4.2''}\tag{4''}
	\end{equation}
	В частном случае совпадения направлений распространения всех волн 
	\begin{equation}
		\zeta_{s}=\zeta
		\label{6.2}
	\end{equation}
	уравнения \eqref{4.2''} преобразуются в систему 
	\begin{equation}
		\frac{d}{d\zeta}m_{s}^{2}+2\gamma_{s}m_{s}^{2}\mp2m_{1}m_{2}m_{3}m_{4}sin\theta=0.
		\label{7.2.2}
	\end{equation}
	Если считать, что диссипация отсутствует  
	\begin{equation}
		\overline{\gamma}_{1}=\overline{\gamma}_{2}=\overline{\gamma}_{3}=\overline{\gamma}_{4}=0,\label{8.2}
	\end{equation}
	то в этом случае можно написать уравнение для величины $\theta$(см. также (1.23))
	\begin{equation}
		\frac{d\theta}{d\zeta}=\delta+\frac{cos\theta}{sin\theta}\cdot\frac{d}{d\zeta}[ln(m_{1}m_{4}m_{2}m_{3})]+\sum_{q=1}^{4}(\beta_{1q}+\beta_{4q}-\beta_{2q}-\beta_{3q}).
		\label{9.2}
	\end{equation}
	в котором $$\delta=\Delta k_{z}\cdot L_{0},\quad(\beta_{sq})\frac{c\omega_{s}}{\hat{n}_{s}}L_{0}.$$
	
	\underline{Пункт 3.} \textbf{\textit{Первые интегралы уравнений в отсутствие диссипации}}
	
	Рассмотрим решения уравнений \eqref{7.2.2} в приближении отсутствия поглощения  \eqref{8.2}. Тогда для чисел квантов всех четырех частот получим \textit{\textbf{законы сохранения}}, которые называются \textit{\textbf{соотношениями Мэнли-Роу}}. Эти соотношения имеют вид  
	\begin{equation}
		m_{1}^{2}+m_{2}^{2}=N_{1},\quad m_{1}^{2}+m_{3}^{2}=N_{2},\quad m_{1}^{2}-m_{4}^{2}=N_{3},
		\label{10.2}
	\end{equation}
	аналогичный (1.24). Законы сохранения \eqref{10.2} имеют достаточно ясный физический смысл, который можно сформулировать одним предложением.
	\textbf{\textit{Если число квантов в волне с частотой  $\omega_{1}$, проходящих через  $\textbf{ 1} \text{ см}^{2}$ волнового фронта за $\textbf{ 1}\text{ сек}$, увеличивается на некоторое количество, то число квантов в волне с частотой $\omega_{4}$ также увеличивается на то же количество, а в волнах с частотами $\omega_{2}$ и $\omega_{3}$ уменьшается на такую же величину.}}
	
	Умножив первое соотношение \eqref{10.2} на $\hbar\omega_{2}$, второе на $\hbar\omega_{3}$,третье на $(-\hbar\omega_{4})$ и сложив все, получим закон сохранения энергии 
	\begin{multline}
		\hbar\left[(m_{1}^{2}+m_{2}^{2})\omega_{2}+(m_{1}^{2}+m_{3}^{2})\omega_{3}+(m_{4}^{2}-m_{1}^{2})\omega_{4}\right]=\hbar\left(m_{1}^{2}\omega_{1}+m_{2}^{2}\omega_{2}+m_{3}^{2}\omega_{3}+m_{4}^{2}\omega_{4}\right)\equiv\\
		\equiv\sum_{q=1}^{4}\overline{S_{qz}^{T}}\equiv\overline{S_{z}}=\hbar(N_{1}\omega_{2}+N_{2}\omega_{3}-N_{3}\omega_{4})=const\label{11.2}
	\end{multline}
	\newpage
	\underline{Пункт 4.} \textbf{\textit{Генерация третьей гармоники в отсутствие поглощения.}}
	
	Модифицируем уравнения (\eqref{3}) для взаимодействия типа (а), заменив в них $\omega_{1}\rightarrow-\omega_{1}$; $\tilde{E}_{1}^{*}\rightarrow\tilde{E}_{1}$;$\varphi_{1}\rightarrow-\varphi_{1}$. Применим модифицированные уравнения  для изучения генерации третьей гармоники. Рассмотрим взаимодействие
	\begin{equation}
		1^{o}+1^{o}+1^{o}=3^{e}.
		\label{12.2}
	\end{equation}
	В этом случае в соотношениях (2)  и уравнениях \eqref{4.2} нужно заменить 
	\begin{multline}
		\omega_{1}\rightarrow-\omega,\quad\omega_{2}=\omega_{3}\rightarrow\omega,\quad\omega_{4}\rightarrow3\omega,\quad,\vec{k}_{1}\rightarrow-\vec{k},\quad\vec{k}_{2}=\vec{k}_{3}\rightarrow\vec{k},\vec{k}_{4}\rightarrow\vec{k}_{3},\\
		\tilde{E}_{1}^{*}\rightarrow\tilde{E}_{1};\quad\varphi_{1}\rightarrow-\varphi;\quad\tilde{E}_{2}=\tilde{E}_{3}\rightarrow\tilde{E}_{1};\quad\tilde{E}_{4}\rightarrow\tilde{E}_{3},\label{13.2}
	\end{multline}
	а  в уравнениях \eqref{7.2.2} и \eqref{9.2} сделать замены:
	\begin{equation}
		m_{1}=m_{2}=m_{3}=m;\qquad m_{4}=v.
		\label{14.2}
	\end{equation}
	Как и в случае удвоения в квадратичной среде, амплитуда поля  $\tilde{E}_{1}$  есть амплитуда, в $\sqrt{3}$ раз меньшая амплитуды действующего в среде поля на частоте $\omega_{1}=\omega_{2}=\omega_{3}=\omega$. 
	
	Из \eqref{11.2} находим в нашем случае закон сохранения энергии в виде 
	\begin{multline}
		\overline{S_{z}}=3\overline{S_{1z}}^{T}+\overline{S_{3z}}^{T}=\frac{c}{8\pi\mu}\left(3\hat{n}_{1}E_{1}^{2}+\hat{n}_{3}E_{3}^{2}\right)=const=\\
		=\hbar(m_{2}^{2}\omega_{2}+m_{3}^{2}\omega_{3}+m_{4}^{2}\omega_{4}+m_{1}^{2}\omega_{1})=(3\hbar m^{2}\omega+\hbar v^{2}\cdot3\omega).
		\label{15.2}
	\end{multline}
	Уравнения \eqref{7.2.2} преобразуются к виду
\begin{align}
	\frac{dm}{d\zeta}=-m^{2}v\sin\theta;\label{16a.2}\tag{16a}\\
	\frac{dv}{d\zeta}=m^{3}sin\theta,\label{16b.2}\tag{16б}
\end{align}
а уравнение для разности фаз \eqref{9.2} примет вид
\begin{equation}
	\frac{d\theta}{d\zeta}=\delta+ctg\theta\frac{d}{d\zeta}ln(m^{3}v)+am^{2}+bv^{2},
	\tag{16в}\label{16v.2}
\end{equation}
где \setcounter{equation}{16}
		\begin{equation}
			\begin{aligned}
				a=(9\omega c)\left(\frac{\beta_{31}}{\hat{n}_{3}}-\frac{\beta_{11}}{\hat{n}_{1}}\right)L_{0}\frac{8\pi\mu\overline{S_{z}}}{3c\hat{n}_{1}},\\  
				b=(3\omega c)\left(\frac{\beta_{33}}{\hat{n}_{3}}-\frac{\beta_{13}}{\hat{n}_{1}}\right)L_{0}\frac{8\pi\mu\overline{S_{z}}}{3c\hat{n}_{3}}.
			\end{aligned}
			\label{17.2}
		\end{equation}
		По сравнению с аналогичными (по физическому смыслу) уравнениями (1.28), описывающими генерацию второй гармоники в квадратичной среде, в уравнениях (16) имеется кое-что новое. Новое состоит в том, что изменилась нелинейность в правых частях первых двух уравнений (16), и в особенности в том, что из-за высокочастотного эффекта Керра появилась нелинейная добавка к $\varepsilon$ и соответственно к постоянной распространения, которая дала свой вклад в уравнение \eqref{16b.2}. 
		
		Закон Менли-Роу сохранения чисел квантов (энергии) в переменных $m$ и $v$ имеет вид
		\begin{equation}
			m^{2}+v^{2}=1
			\label{18.2}
		\end{equation}
		Уравнение \eqref{16b.2} можно проинтегрировать методом вариации произвольной постоянной и получить первый интеграл в виде 
		\begin{equation}
			m^{3}v\cos\theta=\hat{\Gamma}_{\delta}-\left[\frac{\delta}{2}v^{2}-\frac{a}{4}m^{4}+\frac{b}{4}v^{4}\right]\equiv\Gamma_{\delta}-\left[\frac{\delta+b}{2}v^{2}+\frac{-a+b}{4}m^{4}\right].
			\label{19.2}
		\end{equation}
		\textbf{\textit{Оценим влияние эффекта Керра на коэффициент преобразования в третью гармонику.}}
	
	Разрешая \eqref{19.2} относительно $\cos\theta$ и выражая $\sin\theta=\pm\sqrt{1-cos^{2}\theta}$, можно с учетом \eqref{18.2} получить уравнение относительно  $v$ (или $m$) в виде 
	\begin{equation}
		\frac{dv^{2}}{d\zeta}=\pm2\cdot\sqrt{v^{2}(1-v^{2})^{3}-\left\{\hat{\Gamma}_{\delta}-\left[\frac{\delta}{2}v^{2}+\frac{b}{4}v^{4}-\frac{a}{4}(1-v^{2})^{2}\right]\right\}^{2}}.
		\label{20.2}
	\end{equation}
	Если заданы начальные условия в виде
	\begin{equation}
		m(0)=1,\qquad v(0)=0,\label{21.2}
	\end{equation}
	то
	\begin{equation}
		\hat{\Gamma}_{\delta}=-\frac{a}{4},\label{22.2}
	\end{equation}
	и уравнение \eqref{20.2} преобразуется к виду 
	\begin{equation}
		2\zeta=\pm\int\limits_{0}^{v^{2}}\dfrac{dv^{2}}{\sqrt{v^{2}\left\{(1-v^{2})^{3}-\dfrac{v^{2}}{4}\left[(a+\delta)+\dfrac{1}{2}(b-a)v^{2}\right]\right\}}}
		\label{23.2}
	\end{equation}
	Подкоренное выражение в этом случае в отличие от аналогичного выражения (1.38*) для квадратичной среды есть полином  4-го  порядка по $v^{2}$. Один из корней полинома -- 
	\begin{equation}
		v^{2}_{a}=0,\label{24.2}
	\end{equation}
	а следующий корень -- 
	\begin{equation}
		v^{2}_{b}<1,\label{25.2}
	\end{equation}
	если только не выполняется условие
	\begin{equation}
		a=b=-\delta
		\label{26.2}
	\end{equation}
	Это означает, что правая часть \eqref{23.2} может быть преобразована к  эллиптическому интегралу и, следовательно, решение уравнения \eqref{23.2} представимо в виде эллиптической функции, изменяющейся в пределах $v^{2}_{a}<v^{2}<v^{2}_{b}<1$. Физически это означает, что нельзя осуществить полную перекачку мощности волне $\tilde{E}_{3}$  от волны $\tilde{E}_{1}$  без точного согласования фазовых скоростей, когда
	\begin{equation}
		\theta=const=\frac{\pi}{2}\label{27.2}
	\end{equation}
	Для полной перекачки необходимо осуществить точное согласование фазовых скоростей волн с учетом изменения $\varepsilon$ из-за эффекта Керра. Именно этот факт и отражают условия \eqref{26.2}.
	
	Поскольку в кубичной среде условия \eqref{26.2} практически невыполнимы, то нельзя добиться полной перекачки энергии в третью гармонику. \textit{\textbf{Такой же эффект имеет место в любой среде}}, где участвуют в процессе более трех волн.
	
	Если же $\delta=0$, то расчеты показывают, что КПД волноводного утроителя частоты без потерь не превышает $60\%$.

	\underline{Заключение.}
	
	Главным отличием нелинейного взаимодействия волн в кубичной среде от аналогичных процессов в квадратичной среде является не когерентное взаимодействие волн, обусловленное дополнительной нелинейной диэлектрической проницаемостью $\varepsilon$ из-за эффекта Керра. Наличие нелинейной добавки к $\varepsilon$ изменяет условия согласования фазовых скоростей взаимодействующих волн, усложняет схемы оптимальных преобразователей частоты. В частности, иногда не оптимальными оказываются преобразователи частоты, в которых все волны (пучки) распространяются в одном направлении. В этих случаях приближение плоских волн оказывается весьма грубым приближением, малопригодным для оценок\footnote{Реально всегда имеем дело с пучком. Если $a_{\perp}$- его характерный поперечный размер, то $|\vec{k}_{\perp}|~(2\pi/a_{\perp})$ есть характерное поперечное отклонение вектора  $\vec{k}$  от направления синхронизма и соответственно $2\Delta\theta=2|k_{\perp}/k|$ есть характерный раствор конуса угла, в котором сосредоточено излучение. Далеко не для всех плоских волн внутри этого конуса выполняется соотношение синхронизма. Поэтому в целом взаимодействие волн оказывается менее эффективным.}. Но прежде, чем перейти к пучкам, познакомимся ещё с некоторыми нелинейными явлениями в оптике в приближении плоских волн. 
\newpage
\setcounter{section}{0}
\setcounter{equation}{0}
\section*{Глава III. Взаимодействие волн при вынужденном комбинационном (или \textit{Рамановском}) рассеянии (ВКР) лазерного излучения}
\section{Физическая природа комбинационного рассеяния (КР)}
\subsection{Феноменологическое описание КР. Стоксово излучение.}

Комбинационное рассеяние (КР) давно используется для изучения колебательных спектров молекул. Суть КР в том, что от вещества, облучённого светом частоты $\omega_{L}$ , отражается свет на частотах $\omega_{S,a}=\omega_{L}\mp\omega_{V}$, смещённых относительно несущей частоты $\omega_{L}$ на постоянную величину $\omega_{V}$ . Это следствие \textbf{двухквантовых} процессов (Рис. 3.1). Интенсивность \textbf{стоксова рассеяния}
\begin{equation}
	\hbar\omega_{S}=\hbar\omega_{L}-\hbar\omega_{V}\label{1.3.1}
\end{equation}
на несколько порядков больше интенсивности \textbf{антистоксова рассеяния}
\begin{equation}
	\hbar\omega_{a}=\hbar\omega_{L}+\hbar\omega_{\text{Г}}
	\tag{1'}\label{1.3.1'}
\end{equation}
\setcounter{footnote}{0}
Различие интенсивностей этих процессов объясняется тем, что молекул на уровне 2 находится в   $exp(+\hbar\omega_{V}/kT)$   раз меньше, чем на нижнем уровне \footnote{Если $\omega_{V}=3\cdot10^{13}c^{-1},\quad t=300\o K$, то $\left(h\omega_{V}/kT\right)\cong0,76$ и $e^{0,76}=2,14$. Для $\omega_{V}>>3\cdot10^{13}c_{-1}$ это различие много больше 2.}. 

Уровни 2 и 1 относятся к колебательному спектру молекулы и не обладают дипольным моментом $$\vec{d}_{1,2}=\iiint\tilde{\xi}_{2}^{*}\left(\vec{r}\right)\cdot\vec{r}e\tilde{xi}_{1}\left(\vec{r}\right)dV=0$$.
Поэтому они не взаимодействуют с полем частоты $\omega_{1,2}\equiv\omega_{V}$ и не проявляют себя в спектрах поглощения. Говорят, что переход $2 \rightarrow 1$ \textit{запрещен}, что обе функции $\tilde{xi}_{1,2}$   имеют \textbf{\underline{одинаковую четность}}. 

Какому же механическому колебанию зарядов в молекуле может отвечать этот энергетический уровень? Оказывается, что он отвечает, например, продольному колебанию навстречу друг другу двух одинаково заряженных ядер (Рис. 3.2). Это колебание с наибольшей простотой объясняет полуфеноменологическую интерпретацию явления ВКР, которую дал Плачек. Её суть в том, что электронное облако, обволакивающее ядра и ответственное за поляризуемость молекулы, зависит от расстояния между ядрами (от ядерной координаты $Q$). От координаты $Q$ зависит тензор электронной поляризуемости молекулы $\bar{\alpha}(Q)$ и, как следствие, дипольный момент 
\begin{equation}
	\vec{d}=\left(\bar{\alpha}(Q)\cdot\vec{E}\right),\label{1.3.2}
\end{equation}
который наводится под воздействием внешнего поля. Зависимость $\bar{\alpha}(Q)$ от параметра $Q$  должна быть достаточно слабой. Поэтому тензор $\bar{\alpha}(Q)$ можно представить в виде ряда 
\begin{equation}
	\bar{\alpha}(Q)=\bar{\alpha}(0)+\left(\frac{\partial\bar{\alpha}}{\partial Q}\right)_{0}Q+\dots;
	\label{1.3.3}
\end{equation}
по степеням малого параметра $Q$ , меняющегося во времени по гармоническому закону $Q=Q_{0}\cos\omega_{V}t,$ и затем найти дипольный момент молекулы
\begin{equation}
	\vec{d}=\left(\bar{\alpha}(0)\cdot\vec{E}\right)+\left(\left(\frac{\partial\bar{a}}{\partial Q}\right)_{0}\cdot\vec{E}\right)Q_{0}\cos\omega_{V}t.
	\label{1.3.4}
\end{equation}

Частота $\omega_{V}$ механического колебания определяется свойствами молекулы. 
Амплитуда и фаза колебания связаны с фазой и амплитудой внешней силы, которая действует со стороны поля. Некоторая часть энергии вследствие движения электронного облака передается ядрам, и амплитуда механического колебания от этого растет.

\subsection{Квантово-механический расчет поляризации на стоксовой частоте.}

В основе расчёта находятся уравнения для матрицы плотности трехуровневой системы и заданное электрическое поле в виде $$\vec{E}=\textbf{Re}\left\{\vec{\tilde{E}}_{L}exp(i\omega_{L}t)+\vec{\tilde{E}}_{S}exp(i\omega_{S}t)\right\}.$$

\begin{enumerate}
	\item В гамильтониане системы $$\bar{H}_{0}+\bar{H}=\bar{H}_{0}-\left(\bar{vec{d}}_{31}\cdot\vec{E}_{L}\right)-\left(\bar{vec{d}}_{32}\cdot\vec{E}_{S}\right)$$ имеются малые члены $\left(\left|\bar{H}\right|<<\left|\bar{H}_{0}\right|\right),$ ответственные за взаимодействие молекулы с полем.
	\item Предполагается, что в нулевом приближении $\left(\bar{H}=0\right)$ справедливы начальные условия $$\rho_{11}(0)=1,\quad\rho_{22}(0)=\rho_{33}(0)=0.$$
	\item В первом приближении в силу $\bar{H}_{13}=-\left(\bar{vec{d}}_{31}\cdot\vec{E}_{L}\right)\neq0$ определяется $$\tilde{\rho}_{13}^{(1)}exp\left(i\omega_{L}t\right)~\tilde{\vec{d}}_{13}T_{13}\vec{\tilde{E}}_{L}exp(i\omega_{L}t).$$
	\item Во втором приближении находится $$\rho_{33}^{(2)}\sim\tilde{\vec{d}}_{13}*T_{33}\vec{\tilde{E}}*_{L}exp(-i\omega_{L}t)\rho_{13}^{1}exp(i\omega_{L}t)=T_{33}T_{13}\left|\left(\tilde{\vec{d}}_{13}\cdot\vec{\tilde{E}}_{L}*\right)\right|^{2}.$$
	\item В третьем приближении теории возмущений определяется $$\rho_{32}^{(3)}\sim\tilde{\vec{d}}_{32}T_{32}\vec{\tilde{E}}_{S}exp(i\omega_{S}t)\rho_{33}^{(2)}=T_{33}T_{31}T_{32}\left|\tilde{\vec{d}}_{13}\right|^{2}\left|\vec{\tilde{E}}_{L}\right|^{2}\tilde{\vec{d}}_{32}\vec{\tilde{E}}_{S}exp(i\omega_{S}t).$$
	\item Компонент поляризации на стоксовой частоте найдётся как $$\vec{P}_{S}=N\left(\tilde{\rho}_{32}\tilde{\vec{d}}_{32}+\tilde{\rho}_{23}\tilde{\vec{d}}_{23}\right)\cong N\left|\tilde{\vec{d}}_{13}\right|^{2}\left|\tilde{\vec{d}}_{23}\right|^{2}T_{32}T_{31}T_{33}\left|\vec{\tilde{E}}_{L}\right|^{2}\vec{\tilde{E}}_{S}exp(i\omega_{S}t)+\text{\textit{к.с.}}$$
\end{enumerate}

Из квантово-механического расчёта следует, что амплитуда поляризации на стоксовой частоте, которая ответственна за изменение поля на $\omega_{S}$, пропорциональна интенсивности излучения на основной частоте $\omega_{L}$. 
\section{Основные уравнения процесса ВКР}

Процесс ВКР описывается уравнениями Максвелла в виде
\begin{equation}
	\left[\nabla\times\left[\nabla\times\vec{E}\right]\right]+\frac{\mu}{c^{2}}\frac{\partial^{2}\left(\hat{\varepsilon}_{0}\cdot\vec{E}\right)}{\partial t^{2}}+\frac{4\pi\mu\sigma_{0}}{c^{2}}\frac{\partial\vec{E}}{\partial t}=-\frac{4\pi\mu}{c}\frac{\partial^{2}\vec{P}^{NL}}{\partial t^{2}},
	\label{1.3.5}
\end{equation}
\begin{equation}
	\vec{P}^{NL}=N\left(\left(\frac{\partial\hat{\alpha}}{\partial Q}\right)\cdot Q\vec{E}\right),
	\label{1.3.6}
\end{equation}
где $Q=Q_{0}cos\omega_{V}t$ -- ядерное колебание. Система \eqref{1.3.5}, \eqref{1.3.6} должна быть замкнутой. Для этого необходимо написать уравнение для $Q$, что можно сделать на основе феноменологической теории. 

Феноменологический подход обусловлен тем, что уравнением для осцилляторного колебания $Q$ должно быть уравнение осциллятора с затуханием и силой в правой части. Чтобы получить выражение для силы, нужно воспользоваться \textbf{стандартным методом}, основанным на свойствах \textbf{свободной энергии} нелинейного диэлектрика в присутствии электрического поля. 

\textbf{Свободная энергия} среды в электрическом поле определяется  как функция
\begin{equation}
	F=\overline{W}-\left(\vec{P}^{L}\cdot\vec{E}\right)-\left(\vec{P}^{NL}\cdot\vec{E}\right)-\dots\equiv\overline{W}-F^{L}-F^{NL}-\dots,\tag{7'}
	\label{1.3.7'}
\end{equation}
которая связана со средней энергией $\overline{W}$ и \textbf{работой} $F^{L}+F^{NL}$ по созданию поля в среде. При изменении поля \textbf{работа}  имеет полный дифференциал 
\begin{equation}
	d(F^{L}+F^{NL})=-\left(\vec{P}\cdot d\vec{E}\right)\equiv-\left(\vec{P}^{L}\cdot d\vec{E}\right)-\left(\vec{P}^{NL}\cdot d\vec{E}\right)
	\label{1.3.7}
\end{equation}
и поэтому справедливо определение нелинейного дипольного момента
\begin{equation}
	\vec{P}^{NL}=-\frac{\partial F^{NL}}{\partial \vec{E}}.
	\tag{6*}
	\label{1.3.6*}
\end{equation}

Из \eqref{1.3.6} и \eqref{1.3.6*} получаются выражения для нелинейной части свободной энергии\footnote{См. также приложение к главе 3}
\begin{equation}
	F^{NL}=-\frac{N}{2}\left(\left(\frac{\partial\hat{\alpha}}{\partial Q}\right)_{0}\cdot Q\vec{E}\cdot\vec{E}\right),
	\tag{7*}
	\label{1.3.7*}
\end{equation}
обусловленной дипольным моментом единицы объёма среды $\vec{P}^{NL}$, а также силы 
\begin{equation}
	\frac{1}{N}f_{q}=-\frac{1}{N}\frac{\partial F}{\partial Q}
	\label{1.3.8}=\frac{1}{2}\left(\left(\frac{\partial\hat{\alpha}}{\partial Q}\right)\cdot\vec{E}\cdot\vec{E}\right)
\end{equation}
в правой части уравнения для одиночного осциллятора $Q$. С учетом \eqref{1.3.8} уравнение для $Q$ имеет вид 
\begin{equation}
	\ddot{Q}+\omega_{21}^{2}Q+2\gamma_{Q}\dot{Q}=\frac{1}{2}\left(\left(\frac{\partial\hat{\alpha}}{\partial Q}\right)\cdot\vec{E}\cdot\vec{E}\right)
	\label{1.3.9}
\end{equation}

Система уравнений \eqref{1.3.5}, \eqref{1.3.6}, \eqref{1.3.9} – полная и замкнутая. В случае, когда учитывается образование лишь стоксовой частоты, её решение следует искать в виде 
\begin{equation}
	\vec{E}=\frac{1}{2}\tilde{e}^{0}_{L}\tilde{\vec{E}}exp\left\{i\left[\omega_{L}t-\left(\vec{k}_{L}\cdot\vec{r}\right)\right]\right\}+\frac{1}{2}\tilde{\vec{e}}_{S}^{0}\tilde{E}_{S}exp\left\{i\left[\omega_{S}t-\left(\vec{k}_{S}\cdot\vec{r}\right)\right]\right\}+\text{\textit{к.с.}}
	\label{1.3.10}
\end{equation}
$$Q=\frac{1}{2}\tilde{Q}_{r}exp(i\omega_{V}t)+\text{\textit{к.с.}}$$

Для медленно меняющихся амплитуд $\tilde{E}_{L}(\vec{r})$, $\tilde{E}_{S}(\vec{r})$ получаются уравнения  
\begin{equation}
	\begin{cases}
		\left(\left[\tilde{e}_{L}^{0}\times\left[\vec{k}_{L}\times\tilde{e}_{L}^{0}\right]\right]\cdot\nabla\tilde{E}_{L}\right)+\bar{\gamma}_{L}\tilde{E}_{L}+i\beta\omega_{L}^{2}\tilde{Q}_{V}\tilde{E}_{S}exp(i\Phi)=0,\\
		\left(\left[\tilde{e}_{S}^{0}\times\left[\vec{k}_{S}\times\tilde{e}_{S}^{0}\right]\right]\cdot\nabla\tilde{E}_{S}\right)+\bar{\gamma}_{S}\tilde{E}_{S}+i\beta\omega_{S}^{2}\tilde{Q}_{V}\tilde{E}_{L}exp(i\Phi)=0,
	\end{cases}
\label{1.3.11}
\end{equation}
$$\text{где}\quad\beta=\frac{\pi}{c^{2}}\mu N\left(\tilde{\vec{e}}_{L}^{0}\cdot\left(\frac{\partial\hat{\alpha}}{\partial Q}\right)_{0}\cdot\tilde{\vec{e}}_{S}^{0*}\right),\quad\Phi=\left(\vec{k}_{L}\cdot\vec{r}\right)-\left(\vec{k}_{S}\cdot\vec{r}\right).$$
Амплитуда $\tilde{Q}_{r}$ колебания ядерной координаты $Q$ находится в виде вынужденного решения уравнения \eqref{1.3.9} на частоте  $\omega_{V}$
\begin{equation}
	\tilde{Q}_{V}(-\omega_{V}^{2}+\omega_{21}^{2}+2i\gamma_{Q}\omega_{V})=\frac{1}{4}\left(\tilde{\vec{e}}_{L}^{0}\cdot\left(\frac{\partial\hat{\alpha}}{\partial Q}\right)_{0}\cdot\tilde{\vec{e}}_{S}^{0}\right)\tilde{E}_{L}\tilde{E}_{S}^{*}
	\tag{11*}
	\label{1.3.11*}
\end{equation}
Поскольку $\tilde{Q}_{V}$ из \eqref{1.3.11*} можно выразить через $\tilde{E}_{L}$ и $\tilde{E}^{*}_{S}$, то \eqref{1.3.11} представляет собой два связанных уравнения
\begin{equation}
	\frac{d\tilde{E}_{L}}{dz_{L}}\left(\frac{\omega_{L}}{c}\hat{n}_{L}\right)+\bar{\gamma}_{L}\tilde{E}_{L}+\omega_{L}^{2}\tilde{g}\left|\tilde{E}_{S}\right|^{2}\tilde{E}_{L}
	\label{1.3.12}
\end{equation}
$$\frac{d\tilde{E}_{S}}{dz_{L}}\left(\frac{\omega_{S}}{c}\hat{n}_{S}\right)+\bar{\gamma}_{S}\tilde{E}_{S}+\omega_{S}^{2}\tilde{g}\left|\tilde{E}_{L}\right|^{2}\tilde{E}_{S}$$
в которых $z_{L,S}$ направления распространения волн $\tilde{E}_{L,S}$ , а коэффициент 
\begin{equation}
	\tilde{g}=\frac{i(\beta^{2}c^{2}/4\pi\mu N)}{(\omega_{21}^{2}-\omega_{V}^{2})+2i\gamma_{Q}\omega_{V}}\equiv\frac{i(\beta^{2}c^{2}/4\pi\mu N)}{(\omega_{21}^{2}-\omega_{V}^{2})^{2}+4(\gamma_{Q}\omega_{V})^{2}}\left\{(\omega_{21}^{2}-\omega_{V}^{2})-2i\gamma_{Q}\omega_{V}\right\}
	\label{1.3.13}
\end{equation}
определяет величину нелинейности среды. 

Уравнения \eqref{1.3.12} отличаются от уравнений (1.20) в двух аспектах.
\begin{enumerate}
	\item В \eqref{1.3.12} отсутствует когерентное взаимодействие, и потому отсутствует условие синхронного взаимодействия. Что это значит? Прежде всего, условие синхронного взаимодействия – это проявление закона сохранения импульса в каждом от-дельном акте рождения или распада фотонов: $$\hbar\vec{k}_{3}=\hbar\vec{k}_{2}+\hbar\vec{k}_{1}.$$
	При взаимодействии полей в квадратичной среде возможность некоторого рассинхронизма (или неполного синхронизма) означала, что некоторый избыток или недостаток импульса в каждом отдельном акте ("покрывался") компенсировался за счет среды в целом, за счет, быть может, импульса от границ среды и т.д. \textbf{В случае ВКР импульс сохраняется за счет той части среды, которая не взаимодействует с э.-м. полем}. Необходимый импульс поставляется средой подобно тому, как это имеет место в кристаллах, где импульс сохраняется с точностью до волнового вектора решетки.
	\item Первая особенность порождает вторую, которая проявляется в законах сохранения. В отсутствие поглощения $$\bar{\gamma}_{S}=\bar{\gamma}_{L}=0$$ из \eqref{1.3.12} можно получить уравнения для чисел фотонов $$m_{L}^{2}=\hat{n}_{L}\left|\tilde{E}_{L}\right|^{2}/\omega_{L}\quad\text{и}\quad(\hat{n}_{S}\left|\tilde{E}_{S}\right|^{2}/\omega_{S})=m_{S}^{2}$$
\end{enumerate}
в виде
\begin{equation}
	\begin{cases}
		\frac{d}{dz_{L}}m_{L}^{2}+\frac{\omega_{L}\omega_{S}}{\hat{n}_{L}\hat{n}_{S}}c(\tilde{g}+\tilde{g}^{*})m_{L}^{2}m_{S}^{2}=0,\\
		\frac{d}{dz_{s}}m_{s}^{2}+\frac{\omega_{L}\omega_{S}}{\hat{n}_{L}\hat{n}_{S}}c(\tilde{g}+\tilde{g}^{*})m_{L}^{2}m_{S}^{2}=0,
	\end{cases}\label{1.3.14}
\end{equation}
Из \eqref{1.3.14} легко получается закон сохранения числа квантов для любого элементарного объема (энергия сохраняется только интегрально с учётом тепловых потерь)
\begin{equation}
	\frac{d}{dz_{L}}m_{L}^{2}+\frac{d}{dz_{S}}m_{S}^{2}=0.\label{1.3.15}
\end{equation}
Его символически можно представить в виде
\begin{equation}
	div\vec{M}=0,\tag{15'}\label{1.3.15'}
\end{equation}
если ввести $\vec{M}=m_{L}^{2}\vec{z}_{L}^{0}+m_{S}^{2}\vec{z}_{S}^{0}$ вектор потока фотонов взаимодействующих волн. 

Особенность \eqref{1.3.14} в том, что \textbf{для развития поля $\tilde{\vec{E}}_{S}$ синхронизм с полем $\tilde{\vec{E}}_{L}$ не нужен: по всем направлениям условия будут одинаковы.}
\section{Порог генерации стоксовой частоты}

Будем считать, что на границе $z=0$ задано лазерное поле $\vec{\tilde{E}}_{L}$ и отсутствует стоксово поле $\vec{E}_{S}$ : 
\begin{equation}
	\tilde{E}_{L}(0)=\tilde{E}_{0},\quad\tilde{E}_{S}(0)=\tilde{E}_{0}.\label{1.3.16}
\end{equation}
Найдем решение уравнения \eqref{1.3.12} для $\tilde{E}_{S}$ в приближении заданного поля $\tilde{E}_{0}$. Из \eqref{1.3.12} следует, что $\tilde{E}_{S}$ будет расти, если
\begin{equation}
	Re(\omega_{S}^{2}E_{0}^{2}\tilde{g}^{*})=\frac{(\beta/4)2\gamma_{Q}\omega_{V}\omega_{S}^{2}}{(\omega_{21}^{2}-\omega_{V}^{2})^{2}+4(\gamma_{Q}\omega_{V})^{2}}\left(\tilde{\vec{e}}_{L}^{0}\cdot\left(\frac{\partial\hat{\alpha}}{\partial Q}\right)_{0}\cdot\tilde{\vec{e}}_{S}^{0}\right)E_{0}^{2}>\bar{\gamma}.
	\label{1.3.17}
\end{equation}
Соотношение \eqref{1.3.17} указывает минимальную величину накачки $(E_{0}^{2})_{\text{пор}}$, при превышении которой начинается усиление стоксовой компоненты $\vec{\tilde{E}}_{S}$.
\begin{enumerate}
	\item Особенность инкремента усиления поля $\vec{E}_{S}$ в том, что он одинаков по всем направлениям. 
	\item Вторая особенность инкремента усиления в том, что он имеет экстремум на частоте $(\omega_{S})_{max}$, которая смещена относительно частоты $(\omega_{L}-\omega_{21})$. Другими словами, частота ядерного колебания $\omega_{V}$, которая устанавливается, тоже не равна $\omega_{21}$. Расчеты показывают, что
	\begin{equation}
		(\omega_{S})_{max}\cong(\omega_{L}-\omega_{21})+\left\{\gamma_{Q}^{2}/(\omega_{L}-\omega_{21})\right\}
		\label{1.3.18}
	\end{equation}
	и поэтому именно на этой частоте быстрее всего начинает расти поле $\tilde{E}_{S}$.
	\item  Третья особенность инкремента в том, что он пропорционален $E_{0}^{2}$ и поэтому длина $L_{0}$ развития $\tilde{E}_{S}$ подобна длине утроения в кубичной среде и короче $L_{0}$ длины удвоения, которая пропорциональна $\left|E_{0}\right|$. 
\end{enumerate}
\section{Вынужденное комбинационное рассеяние}

Вынужденное комбинационное рассеяние (ВКР) наступает, когда $\left|\tilde{E}_{S}\right|^{2}$ становится сопоставимым с $\left|\tilde{E}_{L}\right|^{2}$. В этом случае нужно совместно решать уравнения \eqref{1.3.12}. Будем это делать, вводя очень существенное ограничение.

Поскольку с ростом $\left|\tilde{E}_{S}\right|^{2}$ растет $Q$ и увеличивается число возбужденных молекул на уровне 2, то $\rho_{22}\neq0$. Обычно структура верхнего уровня 3 такова, что возможны переходы  $\omega_{32}\sim\omega_{L}$ и $\omega_{31}\sim\omega_{S}$. В результате появляются поля на частотах  $$\omega_{a}=\omega_{L}+\omega_{V}\quad(\omega_{a})_{2}=\omega_{L}+2\omega_{V}$$ и в общем случае на частотах $(\omega_{S})_{q}=\omega_{L}-q\omega_{V};\quad(\omega_{a})_{q}=\omega_{L}+q\omega_{V}$
\begin{enumerate}
	\item Будем считать, что эти процессы запрещены и что, кроме поля на частоте $\omega_{S}$, никаких полей на других частотах не образуется. 
	\item Будем  считать, что обе волны направлены в одну сторону (в $+z$--направлении). Воспользуемся уравнениями \eqref{1.3.14}, в которых положим 
	\begin{equation}
		z_{S}=z_{L}=z\label{1.3.19}
	\end{equation}
и учтём \eqref{1.3.15}  форме
\begin{equation}
	m_{L}^{2}+m_{S}^{2}=m^{2}(0)\equiv m^{2}(0)\equiv m_{0}^{2}=const.
	\tag{15''}
	\label{1.3.15''}
\end{equation}
\end{enumerate}
Введем
\begin{equation}
	u^{2}=(m_{L}^{2}/m_{0}^{2});v^{2}=(m_{S}^{2}/m_{0}^{2}),\zeta=(z/L_{0})=z\cdot(\omega_{L}\omega_{S}/\hat{n}_{L}\hat{n}_{S})2c(Re\tilde{g})m_{0}^{2}.
	\label{1.3.20}
\end{equation}
Тогда уравнения \eqref{1.3.14} примут вид
\begin{equation}
	\frac{du^{2}}{d\zeta}+u^{2}v^{2}=0;\quad\frac{dv^{2}}{d\zeta}-u^{2}v^{2}=0\label{1.3.21}
\end{equation}
Решение \eqref{1.3.21} в виде
\begin{equation}
	\frac{dv^{2}}{v^{2}(1-v^{2})}=d\zeta=\frac{dv^{2}}{v^{2}}+\frac{dv^{2}}{1-v^{2}}\Rightarrow\left(\frac{v^{2}т}{v_{0}^{2}}\right)\frac{1-v_{0}^{2}}{1-v^{2}}=e^{\zeta}\Rightarrow\left[\frac{v_{0}^{2}e^{\zeta}}{1-v_{0}^{2}+v_{0}^{2}e^{\zeta}}\right]=v^{2}\label{1.3.22}
\end{equation}
позволяет сделать вывод о полной (асимптотической) перекачке $\omega_{L}$ в стоксову частоту $\omega_{S}$ на бесконечно большой длине $z$ при сколь угодно малом начальном значении $v^{@}_{0}$  (например, шумовом поле на $\omega_{S}$). \textit{Этот процесс очень похож на удвоение при полном синхронизме в квадратичной среде. }
\section{ВКР в направлении назад}

ВКР в направлении назад описывается уравнениями \eqref{1.3.12}, в которых 
\begin{equation}
	z_{S}=-z_{L}=-z.
	\label{1.3.23}
\end{equation}
Из этих уравнений с помощью нормировок 
\begin{equation}
	m_{L}^{2}-m_{S}^{2}=m^{2}(0)\equiv m_{0}^{2}=const=m_{L}^{2}(0)-m_{S}^{2}(0)\equiv\left[1-\eta(0)\right]m_{L}^{2}(0)
	\label{1.3.15'''}
	\tag{15'''}
\end{equation}
и \eqref{1.3.20} получаются уравнения
\begin{equation}
	\frac{du^{2}}{d\zeta}+u^{2}v^{2}=0,\quad\frac{dv^{2}}{d\zeta}+u^{2}v^{2}=0
	\label{1.3.24}
\end{equation}
имеющие первый интеграл
\begin{equation}
	u^{2}-v^{2}=1\label{1.3.25}
\end{equation}
Координата $\zeta=(\rfrac{z}{L_{0}})$ в этом случае  нормирована на число фотонов $m_{0}^{2}$ в сечении $z=0$ , которое является разностью чисел фотонов на частотах $\omega_{S}$ и $\omega_{L}$. В отсутствие иных источников числа фотонов  $m_{S}^{2}(0)$  и $m_{L}^{2}(0)$ определяют коэффициент преобразования лазерного излучения в излучение на сток-совой частоте как
\begin{equation}
	\eta(0)=\left\{m_{S}^{2}(0)/m_{L}^{2}(0)\right\}\equiv\left\{v^{2}(0)/\left[1+v^{2}(0)\right]\right\}.
	\label{1.3.26}
\end{equation}

Для корректного расчёта процесса преобразования частоты с помощью уравнений \eqref{1.3.24} следует задать условия для интенсивности стоксова поля на ка-кой-то границе $z=L$ (или где-то внутри слоя) в виде $m_{S}^{2}(z=L)=m_{S}^{2}(L)$. Решение уравнений \eqref{1.3.24} для $v^{2}$  в этом случае имеет вид 
\begin{equation}
	ln\left\{v^{2}/\left(1+v^{2}\right)\right\}=(1-\zeta)+ln\left\{v^{2}_{l}/\left(1+v_{l}^{2}\right)\right\},
	\label{1.3.27}
\end{equation}
где $v_{l}^{2}$ -- значение $v^{2}$  на границе $\zeta=l$ . Решение \eqref{1.3.27} можно представить как 
\begin{equation}
	v^{2}=\left\{v_{l}^{2}exp(1-\zeta)/\left[1+v_{l}^{2}-v_{l}^{2}exp(1-\zeta)\right]\right\}.
	\tag{27'}
	\label{1.3.27'}
\end{equation}
Поскольку величина $\left\{v^{2}/(1+v^{2})\right\}=\left\{v^{2}/u^{2}\right\}\eta(\zeta)$ есть отношение чисел фотонов с энергиями $ \omega_{S}\hbar $  и $ \omega_{L}\hbar $   в каждом сечении слоя, то в отсутствие дополнительного источника стоксова излучения в сечении $z=L$ для заданного значения интенсивности стоксова поля $v_{l}^{2}$ решение \eqref{1.3.27} внутри слоя имеет физический смысл, если выполняется очевидное условие 
\begin{equation}
	\eta(\zeta)\leq\eta(0)\leq1
	\tag{*}\label{1.3.*}
\end{equation}
Условие \eqref{1.3.*} ограничивает область существования решения \eqref{1.3.27}. Оно справедливо только в том случае, если сечение $\zeta=l$ , где задаётся $v_{l}^{2}$ , находится внутри некоторой области слоя среды 
\begin{equation}
	0\leq\zeta\equiv l\leq l_{bn}(v_{l}^{2})
	\label{1.3.28}
\end{equation}
граница $l_{bn}$  которой определяется правой частью \eqref{1.3.*} и находится из 
\begin{equation}
	ln\eta(0)\equiv l+ln\left\{v_{l}^{2}/(1+v_{l}^{2})\right\}\leq l_{bn}+ln\left\{v_{l}^{2}/(1+v_{l}^{2})\right\}=0.	
	\tag{28*}
	\label{1.3.28*}
\end{equation}
Используя нормировку \eqref{1.3.20}, преобразуем \eqref{1.3.28*} в соотношение  
\begin{equation}
	L\frac{\omega_{L}\omega_{S}}{\hat{n}_{L}\hat{n}_{S}}2c(Re\tilde{g})\left\{m_{L}^{2}(0)-m_{S}^{2}(0)\right\}+ln\frac{m_{S}^{2}(L)}{m_{S}^{2}(L)+\left\{m_{L}^{2}(0)-m_{S}^{2}(0)\right\}}\leq 0.
	\tag{28**}\label{1.3.28**}
\end{equation}
При $m_{L}^{2}(0)-m_{S}^{2}(0)\rightarrow0$  оно позволяет найти границу $L_{bn}$  области, в которой можно получить физически обоснованное (реализуемое) решение \eqref{1.3.27} для заданных значений интенсивности $m_{S}^{2}(L)$  и всех прочих величин. В результате преобразования \eqref{1.3.28**} путём разложения его в ряд по малому параметру $m_{L}^{2}(0)-m_{S}^{2}(0)$  получится соотношение 
\begin{equation}
	(L_{bn})\frac{\omega_{L}\omega_{S}}{\hat{n}_{L}\hat{n}_{S}}2c(Re\tilde{g})\cong\frac{1}{m_{S}^{2}(L)}
	\label{1.3.29}
\end{equation}
для определения границы области существования решения \eqref{1.3.27}. Размеры этой области растут при уменьшении $m_{S}^{2}(L)$. Если под $m_{S}^{2}(L)=\overline{m}^{2}$ понимать шумовой фон фотонов $\omega_{S}$ внутри среды, то размеры области определения решения \eqref{1.3.27} достигнут максимального значения 
\begin{equation}
	(L_{bn})_{max}\equiv\overline{L}_{cr}=\left\{\hat{n}_{L}\hat{n}_{S}/2\omega_{L}\omega_{S}c(Re\tilde{g})\cdot\overline{m}^{2}\right\}.
	\tag{29*}\label{1.3.29*}
\end{equation}
Соотношение \eqref{1.3.29*} нужно понимать так. Если на вход среды с ВКР–процессами падает лазерное излучение и для стоксова излучения образуется инкремент $Re\tilde{g}\sim E_{L}^{2}$ , то при толщине слоя, большей или равной $\overline{L}_{cr}$ , возможна генерация (из шумов) стоксова излучения назад с коэффициентом преобразования (полезного действия), достигающим даже предельного значения (равного единице)\footnote{В этом случае приближённое явное выражение для интенсивности стоксова поля имеет вид $m_{S}^{2}(z)\cong\overline{m}^{2}(\overline{L}_{cr}/z)=\left\{\overline{m}^{2}/\left[\overline{m}^{2}\frac{\omega_{L}\omega_{S}}{\hat{n}_{L}\hat{n}_{S}}2c(Re\tilde{g})(z-L_{cr})+1\right]\right\}.$}. 
\section{Антистоксово излучение}
При более полном теоретическом исследовании ВКР следует рассматривать систему связанных уравнений для волн, имеющих частоты $\omega_{Sq}=\omega_{L}-q\omega_{V};\omega_{aq}=\omega_{L}+q\omega_{V}$. Это -- практически неразрешимая задача. 

Поэтому вначале рассмотрим систему из трех связанных уравнений для $\tilde{E}_{a,L,S}$, используя (для простоты) практически нереализуемое условие. Будем считать, что  \textbf{на частотах $\omega_{L}\pm q\omega_{V}$ (где $q>1$) существует большое поглощение и соответствующие поля не возникают (не нарастают с уровня шумов)}.  В приближении плоских волн получим систему уравнений 
\begin{equation}
	\begin{cases}
		\frac{\omega_{L}}{c}\hat{n}_{L}\frac{d\tilde{E}_{L}}{dz_{L}}+\overline{\gamma}_{L}\tilde{E}_{L}+i\omega_{L}^{2}\tilde{Q}_{V}\tilde{E}_{S}exp(-i\Phi_{S})=0;\\
		
		\frac{\omega_{S}}{c}\hat{n}_{S}\frac{d\tilde{E}_{S}}{dz_{S}}+\overline{\gamma}_{S}\tilde{E}_{S}+i\omega_{S}^{2}\tilde{Q}_{V}\tilde{E}_{L}exp(+i\Phi_{S})=0;\\
		
		\frac{\omega_{a}}{c}\hat{n}_{a}\frac{d\tilde{E}_{a}}{dz_{a}}+\overline{\gamma}_{a}\tilde{E}_{a}+i\omega_{a}^{2}\tilde{Q}_{V}\tilde{E}_{L}exp(-i\Phi_{a})=0,
	\end{cases}\label{1.3.30}
\end{equation}
где

	$$\tilde{Q}_{V}=(k_{S}/\tilde{D})\cdot\left\{\tilde{E}_{L}\tilde{E}_{S}^{*}exp(i\Phi_{S})R\tilde{E}_{a}\tilde{E}_{L}^{*}exp(-\Phi_{a})\right\};$$
\begin{equation}
	\begin{cases}
		k_{S,a}=\frac{1}{4}\left(\tilde{\vec{e}}_{L}^{0}\cdot\left(\frac{\partial\hat{\alpha}}{\partial Q}\right)_{0}\cdot\vec{e}_{S,a}^{0*}\right);\quad\tilde{D}=\omega_{21}^{2}-\omega_{V}^{2}+2i\gamma_{Q}\omega_{V},\quad R=k_{a}/k_{s};\\
		\Phi_{S,a}=\left(\vec{k}_{S,a}\cdot\vec{r}\right)-\left(\vec{k}_{L}\cdot\vec{r}\right).
	\end{cases}
	\label{1.3.30'}\tag{30'}
\end{equation}
В традиционном случае $z_{L}=z_{S}=z_{a}=z$ уравнения \eqref{1.3.30} имеют вид 
\begin{equation}
	\begin{cases}
		\cfrac{d\tilde{E}_{L}}{dz}+\gamma_{L}\tilde{E}_{L}+c\cfrac{i\beta\omega_{L}}{\hat{n}_{L}\tilde{D}}\left\{E_{S}^{2}\tilde{E}_{L}+R\tilde{E}_{a}\tilde{E}_{S}\tilde{E}_{L}^{*}exp(i\theta)\right\}=0;\\
		
		\cfrac{d\tilde{E}_{a}}{dz}+\gamma_{a}\tilde{E}_{a}+c\cfrac{i\beta\omega_{a}}{\hat{n}_{a}\tilde{D}}\left\{R^{2}E_{L}^{2}\tilde{E}_{a}+R\tilde{E}_{L}^{2}\tilde{E}_{S}^{*}exp(-i\theta)\right\}=0;\\
		
		\cfrac{d\tilde{E}_{S}}{dz}+\gamma_{S}\tilde{E}_{S}+c\cfrac{i\beta\omega_{S}}{\hat{n}_{S}\tilde{D}^{*}}\left\{E_{L}^{2}\tilde{E}_{S}+R\tilde{E}_{L}^{2}\tilde{E}_{a}^{*}exp(-i\theta)\right\}=0,		
	\end{cases}
	\label{1.3.31}
\end{equation}
где $$\theta=\left(\left[-2\vec{k}_{L}+\vec{k}_{a}+\vec{k}_{S}\right]\cdot\vec{r}\right),\gamma_{L,S,a}=\bar{\gamma}_{L,S,a}(c/\omega_{L,S,a}\hat{n}_{L,S,a}).$$
Система \eqref{1.3.31} также весьма сложна, поэтому рассмотрим только инкременты роста полей $\tilde{E}_{a,S}$ в приближении фиксированного поля накачки 
\begin{equation}
	E_{L}^{2}=const=E_{0}^{2}.
	\label{1.3.32}
\end{equation}
В этом случае \eqref{1.3.31} превращается в систему двух связанных линейных уравнений первого порядка с переменными коэффициентами относительно неизвестных комплексных амплитуд $\tilde{E}_{a}$ и  $\tilde{E}_{S}^{*}$
\begin{equation}
	\begin{cases}
		\cfrac{d\tilde{E}_{a}}{dz}\gamma_{a}\tilde{E}_{a}+i\cfrac{\beta c\omega_{a}}{\hat{n}\tilde{D}}E_{0}^{2}\left\{R^{2}\tilde{E}_{a}+\tilde{R}exp(-i\Delta kz)\tilde{E}_{S}^{*}\right\}=0,\\
		
		\cfrac{d\tilde{E}_{S}^{*}}{dz}\gamma_{S}\tilde{E}_{S}^{*}-i\cfrac{\beta c\omega_{S}}{\hat{n}_{S}\tilde{D}}E_{0}^{2}\left\{\tilde{E}_{S}^{*}+\tilde{R}^{*}exp(-i\Delta kz)\tilde{E}_{a}\right\}=0,
	\end{cases}\label{1.3.33}
\end{equation}
где $$\Delta k=\left(\left[\vec{k}_{a}+\vec{k}_{S}-2\vec{k}_{L}\right]\cdot\vec{z}_{0}\right),\quad arg\tilde{R}=arg\left\{\tilde{E}_{L}^{2}\cdot exp(-i\theta+i\Delta kz)\right\}=\varphi_{R}.$$
Практически на границе в сечении $z=0$ на частотах $\omega_{a,S}$ существуют только шумовые поля $\tilde{E}_{a,S}$. 

В соответствии с имеющимся опытом решения таких уравнений наибольшего роста $\tilde{E}_{a,S}$ можно ожидать (казалось бы) при наличии полного синхронизма $(\Delta k=0)$. "Физическое объяснение" такого ожидания в том, что энергообмен между волнами $\omega_{L}$, $\omega_{a}$ и $\omega_{S}$ происходит через посредство одних и тех же колебаний $Q_{V}$ , имеющих частоту $\omega_{V}$ . Оба взаимодействия будут приводить к накапливающимся эффектам при оптимальной фазе молекулярных колебаний, и при этой фазе должен быть синхронизм фазовых скоростей по направлению $z$ .

Однако в самом простом частном случае 
\begin{equation}
	\gamma_{S}=\gamma_{a}=\gamma;\quad\left|\tilde{R}\right|\equiv R=1;\quad\hat{n}_{a}=\hat{n}_{S}=n;\quad\Delta k=0
	\label{1.3.34}
\end{equation}
нетрудно убедиться, что волны $\tilde{E}_{a,S}$ будут затухать в направлении $z$ . В самом деле, подставляя в этом случае в уравнения \eqref{1.3.33} решения в виде 
\begin{equation}
	\left(\tilde{E}_{a},\tilde{E}_{S}^{*}\right)=\left(\tilde{E}_{a},\tilde{E}_{S}^{*}\right)_{0}\cdot exp\tilde{\lambda}z
	\label{1.3.35}
\end{equation}
из условия нетривиального решения \eqref{1.3.33} получим дисперсионное уравнение
\begin{equation}
	\begin{vmatrix}
		\left(\tilde{\lambda}+\gamma\right)+i\omega_{a}\tilde{g}& i\omega_{a}\tilde{g}(-i\varphi_{R})\\
		-i\omega_{s}\tilde{g}exp(i\omega_{R})& \left(\tilde{\lambda}+\gamma\right)-i\omega_{s}\tilde{g}
	\end{vmatrix}=0=\left(\tilde{\lambda}+\gamma\right)^{2}+\left(\tilde{\lambda}+\gamma\right)(i\tilde{g})\left(\omega_{a}-\omega_{s}\right)=0
	\label{1.3.36}
\end{equation}
корни которого 
\begin{equation}
	\tilde{\lambda}_{1}=-\gamma;\tilde{\lambda}_{2}=-\gamma-i\tilde{g}(\omega_{a}-\omega_{s})=-\gamma-(\omega_{a}-\omega_{s})\left\{\cfrac{\beta cE_{0}^{2}}{\hat{n}_{s}\left|\tilde{D}\right|^{2}\left[i\left(\omega_{21}^{2-\omega_{V}^{2}}\right)+2\gamma_{Q}\omega_{V}\right]}\right\}
	\label{1.3.37}
\end{equation}
имеют отрицательные реальные части. Это значит, что в принятых «оптимальных» условиях поля $\tilde{E}_{a,s}$ не будут нарастать. 

В чем причина такого неожиданного результата? В том, что не учли изменений дисперсионных свойств среды, которые происходят из-за наличия накачки. Не учли изменения $\varepsilon$, которое возникает при наличии $E_{L}^{2}=E_{0}^{2}$. Дело в том, что из-за нелинейности среды на частотах $\omega_{a}$ и $\omega_{s}$   возникают добавки к $\varepsilon$  типа керровской. Поэтому почти так же, как в кубичной среде, где лучшие условия для образования третьей гармоники могут быть созданы при наличии расстройки, для возникновения стоксова и антистоксова излучения в среде должна быть введена \textit{некоторая рассинхронизация $\left(\left[2\vec{k}_{L}-\vec{k}_{S}-\vec{k}_{a}\right]\cdot\vec{z}_{0}\right)$  волновых векторов}. При наличии сильного поля $\varepsilon$  среды изменится, и возникнет необходимое для усиления $\tilde{E}_{a,S}$  согласование фазовых скоростей.

В результате оказывается, что антистоксова часть излучения хорошо усиливается (и, следовательно, становится заметной и регистрируется), если волновой вектор $\vec{k}_{a}$  лежит на поверхности некоторого конуса. Точнее говоря, хорошее усиление для частоты $\omega_{a}$  имеет место, если $\vec{k}_{a}$  находится внутри телесного угла, ограниченного двумя коническими поверхностями\footnote{Поскольку ВКР зависит от направления, то ясно, что реально речь идет о распространении пучков, а не плоских волн. А это, в свою очередь, означает, что в такой среде, где имеется почти керровская добавка к $\varepsilon$  (пропорциональная $E_{S}^{2}$) на частоте поля накачки $\omega_{L}$, накачка начинает самофокусироваться. Из-за самофокусировки поле на оси растет, и ВКР начинается при меньшем пороге, чем ожидалось по теории, основанной на приближении плоских волн.}. 

\section*{Приложение}

Полезно получить среднее значение дифференциала свободной энергии \eqref{1.3.*2}
\begin{multline}
	\overline{dF}^{\Delta t}=\\=-\sum_{s,k}^{N}\overline{\left[\frac{1}{2}\vec{\tilde{P}}(\omega_{s})e^{i\omega_{s}t}+\text{\textit{к.с.}}\right]\left[\frac{1}{2}d\vec{\tilde{E}}(\omega_{k})e^{i\omega_{k}t}+\text{\textit{к.с.}}\right]}^{\Delta t}=\\=-\frac{1}{4}\sum_{S=1}^{N}\vec{\tilde{P}}(\omega_{s})d\vec{\tilde{E}}^{*}(\omega_{s})+\text{\textit{к.с.}}\tag{*}
	\label{1.3.*2}
\end{multline}
за промежуток времени $\Delta t>>2\pi/\omega_{s}$ , а также некоторые другие выражения.

Из \eqref{1.3.*2} тривиальным образом определяется комплексная амплитуда поляризации на стоксовой частоте
\begin{equation}
	\vec{\tilde{P}}(\omega_{s})=-4\langle\partial\bar{F}/\partial\vec{\tilde{E}}^{*}(\omega_{s})\rangle
	\tag{**}
	\label{1.3.**}
\end{equation}
\newpage
С учетом \eqref{1.3.8} далее находим выражение \eqref{1.3.***} для средней энергии в виде суммы трёх членов 
\begin{multline}
	\bar{F}=\\=\frac{-N}{16}\left\{\left(\left(\frac{\partial\hat{\alpha}}{\partial Q}\right)_{0}\cdot\tilde{Q}_{0}\vec{\tilde{E}}_{s}\cdot\vec{\tilde{E}}^{*}_{L}\right)\right.+\\+\left.\left(\left(\frac{\partial\hat{\alpha}}{\partial Q}\right)_{0}\cdot\tilde{Q}_{0}^{*}\vec{\tilde{E}}_{L}\cdot\vec{\tilde{E}}^{*}_{s}\right)\right.+\\+\left.\left(\left(\frac{\partial\hat{\alpha}}{\partial Q}\right)_{0}\cdot\vec{\tilde{E}}_{L}\cdot\vec{\tilde{E}}_{s}^{*}\tilde{Q}_{0}^{*}\right)\right\}+\text{\textit{к.с.}},
	\tag{***}
	\label{1.3.***}
\end{multline}
каждый из которых представляет собой часть свободной энергии на одной из трёх частот $\omega_{L,S,V}$. Выражение для свободной энергии \eqref{1.3.***} позволяет найти комплексные амплитуды нелинейной поляризации на лазерной и стоксовой частотах 
$$\vec{\tilde{P}}_{L}^{NL}=\frac{N}{4}\left(\left(\frac{\partial\hat{\alpha}}{\partial Q}\right)_{0}\cdot\tilde{Q}_{0}\vec{\tilde{E}}_{S}\right);\quad
\vec{\tilde{P}}_{S}^{NL}=\frac{N}{4}\left(\left(\frac{\partial\hat{\alpha}}{\partial Q}\right)_{0}\cdot\tilde{Q}_{0}^{*}\vec{\tilde{E}}_{L}\right)$$
и комплексную амплитуду силы
$$\frac{1}{N}\tilde{f}_{Q}=-\frac{4}{N}\frac{\partial\bar{F}}{\partial\tilde{Q}_{0}^{*}}=-\frac{1}{4}\left(\left(\frac{\partial\hat{\alpha}}{\partial Q}\right)_{0}\cdot\vec{\tilde{E}}_{L}\cdot\vec{\tilde{E}}_{S}^{*}\right),$$
находящейся в правой части  уравнения для осциллятора $Q=Q_{0}\cos\omega_{V}t $. 

\newpage
\setcounter{section}{0}
\setcounter{equation}{0}
\section*{Глава IV. Взаимодействие волн лазерного излучения и звука при вынужденном рассеянии Мандельштама -- Бриллюэна (ВРМБ)}
\section{Физическая природа рассеяния Мандельштама-Бриллюэна (РМБ)}

РМБ  - это рассеяние света на акустических волнах. Волна плотности среды создает решетку диэлектрической проницаемости, на которой может дифрагировать световая волна. Поскольку решётка диэлектрической проницаемости (Рис. 4.1.) перемещается со скоростью $a$  в направлении распространения акустической волны плотности среды, то угол отражения и частота $\omega_{S}$  отражённой волны из-за эффекта Допплера зависят от угла падения лазерной волны $\omega_{L}$. Законы сохранения энергии и импульса в каждом отдельном акте требуют выполнения соотношений 
\begin{equation}
	\begin{split}
		\omega_{L}=\omega_{S}+\omega_{P},\\
		\vec{k}_{L}=\vec{k}_{S}+\vec{k}_{p}.
	\end{split}
	\label{1.4.1}
\end{equation}
Поскольку волновое число волны давления 
\begin{equation}
	\left|\vec{k}_{p}=(\omega_{p}/a)\right|
	\label{1.4.2}
\end{equation}
по порядку величины равно волновому числу $k_{L}$   волны лазерного излуче-ния, а отношение скорости звука к скорости света мало $(a/c)\cong10^{-5}$,  то мало и отношение частоты звуковой волны к частоте лазерного излучения $$(\omega_{p}/\omega_{L})\cong10^{-5},\quad(\omega_{L}/\omega_{S})\cong1.$$
Поэтому при РМБ геометрия волновых векторов $\vec{k}_{L,S,P}$ взаимодействующих волн света и звука может быть совершенно произвольной и, в частности, такой, какая изображена на Рис. 4.2.  Если ввести угол  $\theta$   между $\vec{k}_{L}$  и $\vec{k}_{S}$, то получаются формулы 
\begin{equation}
	k_{P}\cong2k_{L}\sin\frac{\theta}{2};\omega_{p}\cong\frac{2a\omega_{L}}{c}n_{L}\sin\frac{\theta}{2};\omega_{S}\cong\omega_{L}\left(1-\frac{2a}{c}n_{L}\sin\frac{\theta}{2}\right),\label{1.4.3}
\end{equation}
подтверждающие возможность трактовки сдвига частоты $\omega_{L}$ относительно $\omega_{S}$ как следствия эффекта Допплера. 


Акустическая энергия появляется в результате работы, производимой излуче-нием над акустической волной. Акустическая волна $\omega_{p}$ под воздействием  световой волны $\omega_{L}$ будет усиливаться, и при этом создадутся условия для усиления волны $\omega_{S}$. 

В целом ситуация при каждом отдельном акте взаимодействия двух фотонов и фонона точно такая же, какая была в нелинейной квадратичной среде, где во взаимодействии принимали участие три фотона. А это значит, что в основе описания всех процессов в такой нелинейной среде должно находиться описание трехчастотного трехволнового взаимодействия двух электромагнитных волн и одной акустической волны. 

\section{Описание ВРМБ}

Электромагнитное поле в среде описывается уравнениями Максвелла
\begin{equation}
	\left[\nabla\times\left[\nabla\times\vec{E}\right]\right]+\frac{\mu\varepsilon}{c^{2}}\frac{\partial^{2}\vec{E}}{\partial t^{2}}+\frac{4\pi\mu\sigma_{0}}{c^{2}}\frac{\partial\vec{E}}{\partial t}=-\frac{4\pi}{c^{2}}\mu\frac{\partial\vec{P}^{NL}}{\partial t^{2}},
	\label{1.4.4}
\end{equation}
где
\begin{equation}
	\vec{P}^{NL}=\left(\hat{\chi}^{\text{Б}}\cdot\vec{E}:\overleftrightarrow{p}^{0}_{\alpha\beta}p\right)\label{1.4.5}
\end{equation}
в общем виде представляет собой нелинейную поляризацию среды, возникаю-щую вследствие зависимости ее поляризованности от давления  $p$ (или что то же самое от плотности). Поскольку в общем случае в анизотропных твердых средах упругие волны могут иметь разную природу (продольные, поперечные, сдвиговые и т.д.), то каждой из них можно поставить в соответствие орт $\overleftrightarrow{p}^{0}_{\alpha\beta}$ с двойным содержанием (тип волны и направление ее распространения). Тогда  $\hat{\chi}^{\text{Б}}$  можно рассматривать как тензор четвертого ранга. Если же иметь в виду только изотропную среду и только продольные волны, то $\hat{\chi}^{\text{Б}}$ можно считать тензором второго ранга. Буквой  $p$  в \eqref{1.4.5} обозначен избыток давления в среде над его равновесной (атмосферной) величиной (далее – давление).

Чтобы сделать систему уравнений \eqref{1.4.4}-\eqref{1.4.5} замкнутой, нужно написать уравнения для давления $p$  в среде с силой в правой части, зависящей от электромагнитного поля. Левая часть уравнения для $p$  должна зависеть от свойств среды и типа распространяющейся волны давления. Правую же часть можно найти практически  аналогично тому, как это было сделано для ВКР. 

Свободная энергия $F$  при изменении $\vec{E}$  имеет полный дифференциал
\begin{equation}
	d(\Delta F^{NL})=-(\vec{P}^{NL}\cdot d\vec{E})
	\label{1.4.*}
	\tag{*}
\end{equation}
и находится как
\begin{equation}
	\Delta F^{NL}=-0.50\cdot(\hat{\chi}^{\text{Б}}:\overleftrightarrow{p}^{0}_{\alpha\beta}p\cdot\vec{E}\cdot\vec{E}).
	\tag{**}
	\label{1.4.**}
\end{equation}
Далее определяется сила
\begin{equation}
	f_{p}=-\frac{\partial(\Delta F^{NL})}{\partial p}=\frac{1}{2}(\overleftrightarrow{p}^{0}_{\alpha\beta}:\hat{\chi}^{\text{Б}}\cdot\vec{E}\cdot\vec{E})
	\label{1.4.6}
\end{equation}
в правой части уравнения для $p$. 

Для более глубокого понимания природы \textbf{ВРМБ} напишем уравнение для $p$  в изотропной газообразной (или жидкой) среде, где имеется электрострик-ция. 

Полная система уравнений гидродинамики в такой среде имеет вид
\begin{equation}
	\begin{split}
		&\frac{\partial\breve{\rho}}{\partial t}+div\breve{\rho}\breve{\vec{v}}=0;\\
		&\frac{\partial\breve{\vec{v}}}{\partial t}+\left(\breve{\vec{v}}\cdot\nabla\right)\breve{\vec{v}}=-\frac{1}{\breve{\rho}}\nabla\breve{p}+\frac{1}{\breve{\rho}}\nabla\left\{\frac{1}{8\pi}\left(\breve{\rho}\frac{\partial\varepsilon}{\partial\breve{\rho}}\right)_{S}\left|\vec{E}\right|^{2}\right\}+\frac{1}{\breve{\rho}}div\hat{\sigma}+\vec{g};\\
		&\breve{\rho}\breve{T}\left[\frac{\partial\breve{S}}{\partial t}+\left(\breve{\vec{v}}\cdot\nabla\breve{S}\right)\right]=(\hat{\sigma}\cdot\nabla)\breve{\vec{v}}+div(k\nabla\breve{T})+\breve{\rho}Q,
	\end{split}
	\tag{1'}\label{1.4.1'}
\end{equation}
где использованы плотность $(\rho)$,  скорость частиц $(\breve{\vec{v}})$, температура $(T)$, энтропия $(S)$, тензор вязкости $\hat{\sigma}$  среды $(div(\hat{\sigma})=\eta\Delta\breve{\vec{v}}+\left[(\eta/3)+\zeta\right]\nabla div\breve{\vec{v}})$; $Q\Rightarrow\sigma_{0}\left|\vec{E}\right|^{2}$ -- источник тепла. Система \eqref{1.4.1'} должна быть дополнена уравнением состояния. В качестве последнего для простоты возьмём уравнение состояния идеального газа в виде 
$$\breve{p}=f_{1}(\breve{\rho},\breve{T})=(R\breve{T}/\breve{V}^{0})=R\breve{\rho}\breve{T}\quad\text{или}\quad\breve{p}=f_{2}(\breve{\rho},\breve{S})\equiv(\gamma-1)\breve{\rho}^{\gamma}exp\left[(\breve{S}-S_{0})/C_{V}\right],$$
$\gamma=(C_{p}/C_{V})$ -- отношение теплоемкостей, $\breve{V}^{0}\equiv(1/\breve{\rho})$  -- объём, занимаемый единицей плотности вещества. 

Система \eqref{1.4.1'} должна быть линеаризована вблизи состояния равновесия 
\begin{equation}
	\breve{p}=p_{0}+p;\quad\breve{\rho}=\rho_{0}+\rho;\quad\breve{T}=T_{0}+T;\quad\breve{S}=S_{0}+S;\breve{\vec{v}}=0+\vec{v}.
	\tag{2'}
	\label{1.4.2'}
\end{equation}
\newpage
В этом случае для возмущений $\rho,p,S$ и $\vec{v}$  получим уравнения 
\begin{equation}
	(\partial\rho/\partial t)+\rho_{0}div\vec{v}=0;\label{1.4.3'}\tag{3'}
\end{equation}
\begin{equation}	
	\rho_{0}\frac{\partial\vec{v}}{\partial t}=-\nabla p+\left\{\frac{1}{8\pi}\left(\rho\frac{\partial\varepsilon}{\partial\rho}\right)_{S_{0}}\right\}\nabla\left|E\right|^{2}+\eta\Delta\vec{v}+(\rfrac{1}{3}\eta+\zeta)\nabla div\vec{v}\tag{4'}\label{1.4.4'}
\end{equation}
Из линеаризованного уравнения $\breve{p}=f_{1}(\breve{\rho},\breve{T})=R\breve{\rho}\breve{T}$  состояния идеального газа 
\begin{equation}
	R\breve{T}\equiv R(T_{0}+T)=\breve{p}\breve{V}^{0}=(\breve{p}/\breve{\rho})\equiv[(p_{0}+p)/(\rho_{0}+\rho)]\cong\frac{p_{0}}{\rho_{0}}+\frac{p}{\rho_{0}}-\frac{p_{0}}{\rho_{0}^{2}}\rho\tag{5'}\label{1.4.5'}
\end{equation}
с учётом $RT_{0}=(p_{0}/\rho_{0})$ находим 
\begin{equation}
	RT=(p/\rho_{0})-(p_{0}/\rho_{0}^{2})\rho
	\tag{5''}\label{1.4.5''}
\end{equation}
Линеаризуя второе уравнение $\breve{p}=f_{2}(\breve{\rho},\breve{S})$  состояния идеального газа, найдём также
\begin{equation}
	p=\rho(\partial p/\partial\rho)_{S_{0}}+S(\partial p/\partial S)_{\rho_{0}}=a^{2}\rho+(p_{0}/C_{V})S.\tag{6'}\label{1.4.6'}
\end{equation}
\setcounter{footnote}{0}
Далее из последнего уравнения гидродинамики в приближении отсутствия источника тепла $(Q\cong0)$ и с учётом \eqref{1.4.6'} получим\footnote{В уравнениях для ВРМБ (для простоы) пренебрегается теплом $(Q\cong0)$, которое играет основную роль в уравнениях для описания взаимодействий полей c участием тепловой нелинейности.}
\begin{equation}
	\rho_{0}T_{0}\frac{\partial S}{\partial t}=\frac{1}{R}k\Delta\left(\frac{p}{\rho_{0}}-\frac{p_{0}}{\rho_{0}^{2}}\rho\right)+(\hat{\sigma}\cdot\nabla)\vec{v}
	\tag{7'}\label{1.4.7'}
\end{equation} 

В простейшем одномерном случае, вводя линейное смещение $q$ , найдем
\begin{equation}
	\vec{v}=\frac{\partial q}{\partial t}\vec{z}_{0};\tag{8'}\label{1.4.8'}
\end{equation}
\begin{equation}
	\rho=-\rho_{0}\frac{\partial q}{\partial z}.
	\tag{9'}\label{1.4.9'}
\end{equation}
В адиабатическом приближении $(S=0)$ из \eqref{1.4.6'} получим 
\begin{equation}
	p\cong a^{2}\rho=\left(\frac{\partial p}{\partial \rho}\right)_{S_{0}}\cdot\rho\equiv-a^{2}\rho_{0}\frac{\partial q}{\partial z}.
	\label{1.4.10'}\tag{10'}
\end{equation}
Если применить операцию $div$  к уравнению \eqref{1.4.4'}, то из преобразованного \eqref{1.4.4'} с учётом \eqref{1.4.10'} и \eqref{1.4.3'} получится уравнение   
\begin{equation}
	-\frac{1}{a^{2}}\frac{\partial^{2}p}{\partial t^{2}}+\frac{\partial^{2}p}{\partial z^{2}}+\left(\frac{4}{3}\eta+\zeta\right)\frac{1}{a^{2}\rho_{0}}\frac{\partial^{3}p}{\partial t\partial z^{2}}=\frac{1}{8\pi}\left(\rho\frac{\partial\varepsilon}{\partial\rho}\right)_{S_{0}}\frac{\partial^{2}}{\partial z^{2}}\left|\vec{E}\right|^{2},
	\tag{11'}
	\label{1.4.11'}
\end{equation}
которое является наиболее простым уравнением-примером. 

Установим связь между правыми частями \eqref{1.4.4} и \eqref{1.4.11'} с помощью следующего соотношения
\begin{equation}
	4\pi\vec{P}^{NL}\equiv4\pi\chi^{\text{Б}}p\vec{E}\equiv\varepsilon^{NL}(p)\cdot\vec{E}=\left(\frac{\partial\varepsilon}{\partial p}\right)_{S_{0}}p\vec{E}\equiv Y\cdot\beta_{S}\cdot p\cdot\vec{E},
	\tag{12'}
	\label{1.4.12}
\end{equation}
в котором введены адиабатическая сжимаемость
\begin{equation}
	\beta_{S}=\left(\frac{1}{\rho}\frac{\partial\rho}{\partial p}\right)_{S_{0}}=\left(-\frac{1}{V^{0}}\frac{\partial V^{0}}{\partial p}\right)_{S_{0}}\equiv(1/T_{y,n,p})
	\tag{7*}\label{1.4.7*}
\end{equation}
и коэффициент электрострикции 
\begin{equation}
	Y=\left(\rho\frac{\partial\varepsilon}{\partial\rho}\right)_{S_{0}}\tag{7**}\label{1.4.7**}
\end{equation}
связанные с $\varepsilon^{NL}$ тождеством 
\setcounter{equation}{7}
\begin{multline}
	\frac{1}{p}\cdot\varepsilon^{NL}\equiv\\
	\equiv\left(\frac{\partial\varepsilon}{\partial p}\right)_{S_{0}}=\left(\frac{\partial\varepsilon}{\partial\rho}\right)_{S_{0}}\left(\frac{\partial\rho}{\partial V^{0}}\right)_{S_{0}}\left(\frac{\partial V^{0}}{\partial p}\right)_{S_{0}}=\left(\frac{\partial\varepsilon}{\partial\rho}\right)_{S_{0}}\left(-\frac{1}{V^{0}V^{0}}\right)_{S_{0}}\left(\frac{\partial V^{0}}{\partial p}\right)_{S_{0}}\equiv\\
	\equiv\left(\rho\frac{\partial\varepsilon}{\partial\rho}\right)_{S_{0}}\left(-\frac{1}{V^{0}}\frac{\partial V^{0}}{\partial p}\right)_{S_{0}}\equiv Y\cdot\beta_{S}\equiv(\chi^{\text{Б}}/4\pi)
	\label{1.4.8}
\end{multline}
В результате уравнение \eqref{1.4.11'} с использованием $\chi^{\text{Б}}$  и $\Gamma=\frac{1}{a^{2}\rho_{0}}\left(\frac{4}{3}\eta+\zeta\right)$ примет вид 
\begin{equation}
	-\frac{1}{a^{2}}\frac{\partial^{2}p}{\partial t^{2}}+\frac{\partial^{2}p}{\partial z}+\Gamma\frac{\partial^{3}p}{\partial t\partial z^{2}}=\frac{\chi^{\text{Б}}}{2\beta_{S}}\frac{\partial^{2}}{\partial z^{2}}\left|\vec{E}\right|^{2}.
	\label{1.4.9}
\end{equation}

Коэффициент $\Gamma$  в уравнении \eqref{1.4.9} получен в адиабатическом приближении. Если же учесть изменение $S$ , то $\Gamma$  получит добавку из-за теплопроводности
\begin{equation}
	\Gamma=\frac{1}{a^{2}\rho_{0}}\left[\frac{4}{3}\eta+\zeta+\frac{\kappa}{C_{p}}(\gamma-1)\right].
	\tag{*}\label{1.4.*2}
\end{equation}
В общем случае в уравнении \eqref{1.4.9} следует трансформировать ещё два члена, представив их в наиболее общем виде
\begin{equation}
	\frac{\partial^{2}p}{\partial z^{2}}\rightarrow\nabla^{2}p\equiv\Delta p;\qquad\chi^{\text{Б}}\frac{\partial^{2}}{\partial z^{2}}\left|\vec{E}\right|^{2}\rightarrow\Delta(\overleftrightarrow{p}^{0}_{\alpha\beta}:\hat{\chi}^{\text{Б}}\cdot\vec{E}\cdot\vec{E}).
	\tag{**}
	\label{1.4.**2}
\end{equation}
и, в конечном счёте, получить уравнение 
\begin{equation}
	-\frac{1}{a^{2}}\frac{\partial^{2}p}{\partial t^{2}}+\Delta p+\Gamma\frac{\partial}{\partial t}\Delta p=\frac{1}{2\beta_{S}}\Delta(\overleftrightarrow{p}^{0}_{\alpha\beta}:\hat{\chi}^{\text{Б}}\cdot\vec{E}\cdot\vec{E}).
	\label{1.4.10}
\end{equation}

Уравнения \eqref{1.4.4}, \eqref{1.4.5} и \eqref{1.4.10} образуют замкнутую систему.
\section{Трехчастотное взаимодействие}

Рассмотрим в рамках уравнений \eqref{1.4.4}, \eqref{1.4.5} и \eqref{1.4.10} описание взаимодействия одной звуковой и двух электромагнитных волн 
\begin{equation}
	\begin{split}
		&\vec{E}=Re\left\{\vec{\tilde{e}}_{L}^{0}\tilde{E}_{L}exp\left[i\omega_{L}t-i(\vec{k}_{L}\cdot\vec{r})\right]+\vec{\tilde{e}}_{s}^{0}\tilde{E}_{s}exp\left[i\omega_{s}t-i(\vec{k}_{s}\cdot\vec{r})\right]\right\}\\
		&p=Re\left\{\tilde{p}exp\left[i\omega_{p}t-i(\vec{k}_{p}\cdot\vec{r})\right]\right\}
	\end{split}	
	\label{1.4.11}
\end{equation}
Подставим \eqref{1.4.11} в систему \eqref{1.4.4}, \eqref{1.4.10} и получим систему 
\begin{equation}
	\begin{split}
		&\frac{\omega_{L}}{c}\hat{n}_{L}\frac{d\tilde{E}_{L}}{dz_{L}}+\bar{\gamma}_{L}\tilde{E}_{L}+i\beta\omega^{2}_{L}\tilde{p}\tilde{E}_{s}exp(-i\theta)=0;\\		
		&\frac{\omega_{s}}{c}\hat{n}_{s}\frac{d\tilde{E}_{s}}{dz_{s}}+\bar{\gamma}_{s}\tilde{E}_{s}+i\beta\omega^{2}_{s}\tilde{p}\tilde{E}_{L}exp(i\theta)=0;\\		
		&\frac{\omega_{p}}{a}\frac{d\tilde{p}}{dz_{p}}+\bar{\gamma}_{p}\tilde{p}+i\frac{\chi^{\text{Б}}\kappa^{2}_{p}}{4\beta_{s}}\tilde{E}_{S}^{*}\tilde{E}_{L}exp(i\theta)=0;
	\end{split}
\tag{12'}
\label{1.4.12'}
\end{equation}
в которой
\begin{equation}
	\beta=\frac{\pi}{c^{2}}\mu\left(\overleftrightarrow{p}^{0}_{\alpha\beta}:\hat{\chi}^{\text{Б}}\cdot\vec{\tilde{e}}_{s}^{0}\cdot\vec{\tilde{e}}_{L}^{0*}\right)\equiv\frac{\pi\mu}{c^{2}}\chi^{\text{Б}};\qquad\bar{\gamma}_{p}=\frac{\Gamma}{2}\omega_{p}l^{2}_{p};
	\tag{12''}
	\label{1.4.12''}
\end{equation}
\begin{equation}
	\frac{\chi^{\text{Б}}k_{p}^{2}}{4\beta_{s}}=\beta\omega_{p}^{2}\left(\frac{c^{2}/a^{2}}{4\pi\mu\beta_{s}}\right);\qquad\theta=\left(\left[\vec{k}_{p}+\vec{k}_{s}-\vec{k}_{L}\right]\cdot\vec{r}\right).
	\label{1.4.12'''}
	\tag{12'''}
\end{equation}
Эти уравнения по форме почти совпадают с уравнениями \eqref{20} и \eqref{20} трехчастотного трехволнового взаимодействия в квадратичной среде. Главное отличие в том, что в \eqref{1.4.12'} все волны распространяются в разных направлениях. 
\section{ВРМБ вблизи порога возбуждения}

Вблизи порога возбуждения интенсивности волн звука $p^{2}$  и стоксовой частоты $E_{s}^{2}$  малы по сравнению с $E_{L}^{2}$, и  можно считать $E_{L}^{2}=E_{0}^{2}=const$ . Тогда в системе \eqref{1.4.12'}  останутся два уравнения. Если предпоследнее уравнение \eqref{1.4.12'} записать для $\tilde{E}_{s}^{*}$ , то получим систему из двух связанных уравнений 
\setcounter{equation}{12}
\begin{equation}
	\begin{split}
		\frac{d\tilde{p}}{dz_{p}}+\gamma_{p}\tilde{p}+i\frac{\chi^{\text{Б}}k_{p}^{2}}{4\beta_{s}}\tilde{E}_{0}\tilde{E}_{s}^{*}exp(i\theta)=0;\\
		\frac{d\tilde{E}_{s}^{*}}{dz_{s}}+\gamma_{s}\tilde{E}_{s}^{*}-i\beta\frac{\omega_{s}}{\hat{n}_{s}}c\tilde{E}_{0}^{*}\tilde{p}exp(-i\theta)=0,
	\end{split}
	\label{1.4.13}
\end{equation}
в которой введены коэффициенты $\gamma_{s}=\bar{\gamma}_{s}(c/\hat{n}_{s}\omega_{s})$ ; $\gamma_{p}=\bar{\gamma}_{p}(a/\omega_{p})$ , а направления изменения комплексных амплитуд $\tilde{E}_{s}^{*}$  и $\tilde{p}$  образуют угол $\psi$  (Рис. 4.3)

Полагая, что по каждому направлению волны нарастают с одинаковым инкрементом 
\begin{equation}
	(\tilde{p})=(\tilde{p}^{0})exp(\tilde{\lambda}z_{p});\quad	(\tilde{E}_{s}^{*})=(\tilde{E}_{s}^{*0})exp(\tilde{\lambda}z_{s}),
	\label{1.4.14}
\end{equation}
из условия нетривиальности решения \eqref{1.4.13} найдём дисперсионное уравнение\footnote{Формально на Рис. 4.3 можно выбрать направление по биссектрисе угла $\psi$  и назвать его, например, $y$ . Тогда в уравнениях \eqref{1.4.13} можно перейти к этой новой координате и заменить $z_{s,p}$  на $y$  по правилам $z_{s,p}=y\cdot\cos(\psi/2)$. Если $z_{s,p}$  можно одинаковым образом выразить через $y$ , то их можно заменить на $\bar{z}=y\cdot\cos(\psi/2)$  и вместо \eqref{1.4.13} получить два уравнения, в которых производные берутся по одной координате $\bar{z}$. Решения этой пары уравнений следует искать в виде $(\tilde{p},\tilde{E}_{s}^{*})=(\tilde{p}^{0},\tilde{E}_{s}^{*0})\exp(\tilde{\lambda}\bar{z})$.} 
\begin{equation}
	\begin{vmatrix}
		&\tilde{\lambda}+\gamma_{p}  &i\frac{\chi^{\text{Б}}k_{p}}{4\beta_{s}}\tilde{E}_{0}e^{i\theta}\\
		&-i\beta\frac{\omega_{s}}{\hat{n}_{s}}c\tilde{E}_{0}^{*}e^{-i\theta}  &\tilde{\lambda}+\gamma_{s}
	\end{vmatrix}=0=\left(\tilde{\lambda+\gamma_{p}}\right)\left(\tilde{\lambda+\gamma_{s}}\right)-\beta\frac{\omega_{s}}{\hat{n}_{s}}c\frac{\chi^{\text{Б}}k_{p}}{4\beta_{s}}\left|\tilde{E}_{0}\right|^{2}=0.
	\label{1.4.15}
\end{equation}
Из анализа корней 
\begin{equation}
	\tilde{\lambda}_{1,2}=-\frac{1}{2}(\gamma_{p}+\gamma_{s})\pm\sqrt{\frac{1}{4}(\gamma_{p}+\gamma_{s})^{2}-\gamma_{p}\gamma_{s}+\beta\frac{\chi^{\text{Б}}k_{p}\omega_{s}c}{4\beta_{s}\hat{n}_{s}}E_{0}^{2}}
	\label{1.4.16}
\end{equation}
уравнения \eqref{1.4.15} находим, что $Re\tilde{\lambda}_{1}>0$ в случае 
\begin{equation}
	E_{0}^{2}>\frac{\gamma_{s}\gamma_{p}(4\beta_{s})\hat{n}_{s}}{\beta\chi^{\text{Б}}k_{p}\omega_{s}c}=\frac{\gamma_{s}\gamma_{p}(4\beta_{s})\hat{n}_{s}^{2}}{\pi\mu(\chi^{\text{Б}})^{2}k_{s}k_{p}}=(E_{0}^{2})_{cr}.
	\label{1.4.17}
\end{equation}

\textit{Вторая особенность ВРМБ}. Из \eqref{1.4.17} видна вторая особенность ВРМБ: пороговое значение  $\left|E_{0}^{2}\right|_{cr}$  убывает с ростом $k_{p}$ . Если считать, что величина $k_{L}$  задана (и, следовательно, $k_{s}\approx k_{L}$  с точностью до $10^{-5}$), то $\left|E_{0}^{2}\right|_{cr}$ минимально для $\vec{k}_{s}$ , направленного точно навстречу $\vec{k}_{L}$  (ибо в этом случае $k_{p}$ максимально велико). 

Формулой \eqref{1.4.17} можно пользоваться для оценок ВРМБ во всех средах, включая твердые и кристаллические тела. Воспользуемся \eqref{1.4.8} и свяжем $\chi^{\text{Б}}$ с модулем упругости $\chi^{\text{Б}}=(Y\beta_{s}/4\pi)=(Y/4\pi T_{\text{упр}})$. Используя эту связь, например, для сапфира и полагая $\lambda_{L}=1\mu\text{м}=10^{-4}cm\approx\lambda_{s}$, где $T_{\text{упр}}=5\cdot10^{11}\text{г}\cdot cm^{-1}\cdot\text{сек}^{-2}$;$\gamma_{p}=10cm^{-1}$; $\gamma_{s}\approx\gamma_{L}=10^{-2}cm^{-1}$; $Y\approx0,30\div0,10$, получим пороговую мощность вектора Пойнтинга $\overline{\vec{S}_{L}}^{T}_{cr}=\frac{c}{8\pi}\sqrt{\frac{\varepsilon_{L}}{\mu}}\left|E_{0}^{2}\right|_{cr}=(10^{7}\div10^{8})\frac{\text{Вт}}{\text{см}^{2}}$. 
\section{Свойства уравнений ВРМБ. Законы сохранения }

Вначале преобразуем уравнения \eqref{1.4.12} в уравнения для чисел квантов. Первые два уравнения нужно умножить на $(c^{2}/8\pi\mu\hbar\omega_{L,s}^{2})\tilde{E}_{L,s}^{*}$ и сложить с комплексно сопряженными выражениями. Тогда получим уравнения для чисел квантов $m_{L,s}^{2}=\left\{\left(\vec{S}_{L,s}\cdot\vec{z}_{L,s}^{0}\right)\hbar\omega_{L,s}\right\}$ , где $\vec{S}_{L,s}$ -- векторы Пойнтинга и $\vec{z}_{L,s}^{0}$ указывают направления распространения волновых фронтов этих двух электромагнитных волн. 

Чтобы преобразовать к такому же виду третье уравнение \eqref{1.4.12}, нужно ввести вектор Умова-Пойнтинга для звуковой волны. Такой величиной является интенсивность звука
\begin{equation}
	\overline{S_{p}}{2\pi/\omega_{p}}=\overline{pv}^{2\pi/\omega_{p}}=(1/2)\tilde{p}\tilde{v}^{*},
	\label{1.4.18}
\end{equation}
где  $\tilde{v}$ -- амплитуда скорости частиц среды в звуковой волне. Размерность  вектора Умова $\left[\overline{S_{p}}\right]=MT^{-3}=\left[\overline{\left|\vec{S}_{L,S}\right|}\right]$  совпадает с размерностью \textit{(г/см$^{3}$)}  век-тора Пойнтинга.

Обращаясь к формулам \eqref{1.4.8'}–\eqref{1.4.10'}, найдем 
\begin{equation}
	\tilde{p}=ik_{p}a^{2}\rho_{0}\tilde{q}=i\omega_{p}a\rho_{0}\tilde{q},\quad\tilde{v}=i\omega_{p}\tilde{q}=(\tilde{p}/a\rho_{0})
	\label{1.4.19}
\end{equation}
и далее получим 
\begin{equation}
	\overline{S_{p}}=\left[(\tilde{p}\tilde{p}^{*})/2a\rho_{0}\right]
	\label{1.4.20}
\end{equation}
Таким образом, третье уравнение \eqref{1.4.12} нужно умножить на $(\tilde{p}^{*}/2a\rho_{0}\omega_{p}\hbar)$ и сложить с комплексно сопряженным. Преобразуя коэффициенты в этих уравнениях, нужно учесть соотношение 
\begin{equation}
	\beta_{s}\cdot a^{2}\cdot\rho_{0}\equiv\left(\frac{\partial\rho}{\rho\partial p}\right)_{S_{0}}\cdot\left(\frac{\partial p}{\partial\rho}\right)_{S_{0}}\cdot\rho_{0}\equiv1,
	\label{1.4.21}
\end{equation}
которое является следствием очевидных термодинамических связей. 

В результате преобразований получим систему уравнений 
\begin{equation}
	\begin{split}
		&(dm^{2}_{L}/dz_{L})+2\gamma_{L}m_{L}^{2}+2gm_{L}m_{s}m_{p}\sin\overline{\theta}=0;\\
		&(dm^{2}_{s}/dz_{s})+2\gamma_{s}m_{s}^{2}+2gm_{L}m_{s}m_{p}\sin\overline{\theta}=0;\\
		&(dm^{2}_{p}/dz_{p})+2\gamma_{p}m_{p}^{2}+2gm_{L}m_{s}m_{p}\sin\overline{\theta}=0,
	\end{split}
\label{1.4.22}
\end{equation}
где	$$\bar{\theta}=\theta+\varphi_{L}-\varphi_{s}-\varphi_{p},\qquad g=\beta\sqrt{\left(2\rho_{0}\hbar\omega_{L}\omega_{s}\omega_{p}c^{2}a/\hat{n}_{L}\hat{n}_{s}\right)}$$

Система \eqref{1.4.22} напоминает одновременно систему \eqref{22.1} для трехволновых процессов в квадратичной среде и систему \eqref{14} для взаимодействующих стоксовой $(\omega_{S})$ и лазерной $(\omega_{L})$ волн в среде с ВКР. Как и в случае ВКР, здесь волны распространяются по разным направлениям.

В отсутствие потерь
\begin{equation}
	\gamma_{L}=\gamma_{s}=\gamma_{p}=0
	\label{1.4.23}
\end{equation}
уравнения \eqref{1.4.22} имеют законы сохранения чисел квантов в дифференциальной форме, как в случае ВКР, и с такой физической интерпретацией, как в случае квадратичной среды:
\begin{equation}
	\frac{dm_{L}^{2}}{dz_{L}}+\frac{dm_{s}^{2}}{dz_{s}}=0; \quad \frac{dm_{L}^{2}}{dz_{L}}+\frac{dm_{p}^{2}}{dz_{p}}=0; \quad, \frac{dm_{s}^{2}}{dz_{s}}-\frac{dm_{p}^{2}}{dz_{p}}=0.
	\label{1.4.24}
\end{equation}
Если каждое из уравнений \eqref{1.4.22} умножить на $\overline{\theta}$ и сложить, то с учетом \eqref{1.4.1} получим закон сохранения энергии в дифференциальной форме 
\begin{equation}
	\frac{d}{dz_{L}}(\hbar\omega_{L}m_{L}^{2})+\frac{d}{dz_{s}}(\hbar\omega_{s}m_{s}^{2})+\frac{d}{dz_{p}}(\hbar\omega_{p}m_{p}^{2})=0.
	\tag{24'}
	\label{1.4.24'}
\end{equation}
В отличие от случая квадратичной среды система \eqref{1.4.22} не совсем полная. Отсутствует уравнение для $\overline{\theta}$. Система будет полной и похожей на \eqref{22.2} в некоторых частных случаях.
\section{Стоксово рассеяние вперед}
В некоторых анизотропных кристаллических средах из-за разницы показателей преломления обыкновенной и необыкновенной волн возможно ВРМБ при геометрии, представленной на Рис. 4.4. При этом возможен предельный случай, когда волновые векторы $\vec{k}_{L,s,p}$ всех волн параллельны: 
\begin{equation}
	z_{L}=z_{s}=z_{p}=z=(\zeta/g).
	\label{1.4.25}
\end{equation}
В этом предельном случае уравнения \eqref{1.4.22} просто полностью переходят в три первых уравнения \eqref{22}. При отсутствии поглощения  к ним  добавляется четвертое для $\overline{\theta}$ в виде 
\begin{equation}
	\frac{d\overline{\theta}}{d\zeta}=\frac{1}{g}\Delta k_{z}+\ctg\overline{\theta}\cdot\frac{d}{d\zeta}ln(m_{L}m_{s}m_{p}),
	\label{1.4.26}
\end{equation}
где	$$\Delta k_{z}=(\left[\vec{k}_{p}+\vec{k}_{s}-\vec{k}_{L}\right]\cdot\vec{z}_{0}).$$
Очевидно, что оказываются справедливыми все законы сохранения \eqref{24}–\eqref{26} для квадратичной среды и, конечно, все решения главы I.\footnote{Если ввести относительное число фотонов с помощью нормировки $c\hat{n}_{L}E_{L}^{2}/8\pi\mu\omega_{L}\bar{\Pi}_{z}$, где $\bar{\Pi}_{z}$ -- общий вектор Пойнтинга-Умова, то в \eqref{1.4.22} вместо $g$  появится $L_{0}^{-1}=\beta\sqrt{(2\rho_{0}\bar{\Pi}_{z}\omega_{L}\omega_{s}\omega_{p}c^{2}a/\hat{n}_{L}\hat{n}_{s})}$ и после перехода к безразмерному переменному $\zeta=(z/L_{0})$  они будут полностью совпадать с \eqref{22} как по форме, так и по смыслу.} 
\section{Стоксово рассеяние назад}

Наиболее важным в практическом использовании и экспериментально наиболее просто осуществимым является рассеяние назад 
\begin{equation}
	z_{L}=z_{p}=-z_{s}=z=(\zeta/g)
	\label{1.4.27}
\end{equation}
При условии \eqref{1.4.23} уравнения \eqref{1.4.22} и \eqref{1.4.26} переходят в систему
\begin{equation}
	\begin{split}
		&(dm_{L}/d\zeta)=-m_{s}m_{p}\sin\bar{\theta};\quad(dm_{s}/d\zeta)=-m_{L}m_{p}\sin\bar{\theta};\\
		&(dm_{p}/d\zeta)=-m_{L}m_{s}\sin\bar{\theta};\frac{d\bar{\theta}}{d\zeta}=\delta+\ctg\bar{\theta}\cdot\frac{d}{d\zeta}(\textbf{ln}m_{L}m_{s}m_{p})
	\end{split}\label{1.4.28}
\end{equation}
которая имеет первые интегралы 
\begin{equation}
	\begin{split}
		m_{L}^{2}-m_{s}^{2}=N_{1},\quad m_{s}^{2}+m_{p}^{2}=N_{2},\quad m_{L}^{2}+m_{p}^{2}=N_{3},\quad m_{L}m_{p}m_{s}\cos\bar{\theta}+\frac{\delta}{2}m_{L}^{2}=\Gamma\\
		\hbar\left(\omega_{L}m_{L}^{2}+\omega_{p}m_{p}^{2}-\omega_{s}m_{s}^{2}\right)=\bar{\Pi}_{z}=\left(\left[\bar{\vec{S}}_{L}+\bar{\vec{S}}_{s}+\bar{\vec{S}}_{p}\right]\cdot\vec{z}_{0}\right)\equiv\omega_{L}N_{1}=\omega_{p}N_{2}.
	\end{split}\label{1.4.29}
\end{equation}
В наиболее простой и реальной ситуации, когда на границу слоя нелинейной среды $\zeta=0$  падает излучение $\omega_{L}$  с заданной интенсивностью $E_{L}^{2}(0)$ , а источники звука и стоксова излучения отсутствуют, вполне определенное значение может иметь лишь одна постоянная $N_{3}=m_{L}^{2}(0)$  из всех трёх независимых постоянных первых интегралов \eqref{1.4.29}. Две оставшиеся постоянные определяются, как и в случае ВКР, заданным коэффициентом преобразования $\eta=m_{s}^{2}(0)/m_{L}^{2}(0)$  излучения $\omega_{L}$   в излучение $\omega_{s}$ , а также тем, что на границе $\zeta=1$  задано значение  $m_{s}^{2}(l)$  (даже если оно равно нулю).

В общем случае решение зависит от граничных условий и, в частности, от наличия $m_{p}^{2}(0)$, от расстройки $\delta$  и, конечно, от $\eta,m_{s}^{2}(l)$ . 
\subsection{Точный синхронизм}
\begin{center}
	$\delta=0$	
\end{center}
Рассмотрим решения \eqref{1.4.28} при точном синхронизме $(\delta=0)$, считая, что в сечении $\zeta=l$  поле стоксова излучения отсутствует: 
\begin{equation}
	m_{s}^{2}(l)=0
	\label{1.4.30}
\end{equation}
В этом случае получаем $\Gamma=0$, и поэтому $\cos\bar{\theta}=0$   или 
\begin{equation}
	\bar{\theta}=(\pi/2)+2q\pi.
	\label{1.4.31}
\end{equation}
Выражая $m_{p}=\sqrt{N_{2}-m_{s}^{2}}, m_{L}=\sqrt{N_{2}+m_{s}^{2}}$ и учитывая \eqref{1.3.31}, получим уравнение
\begin{equation}
	(dm_{s}/d\zeta)=-\sqrt{(N_{2}+m_{s}^{2})(N_{2}-m_{s}^{2})},
	\tag{32'}\label{1.4.32'}
\end{equation}
которое с помощью замены переменного $V^{2}=(m_{s}^{2}/N_{2})$  и введения новых обозначений  $k_{2}=\left[N_{2}/(N_{1}+N_{2})\right]$  и $k'^{2}=\left[N_{1}/(N_{1}+N_{2})\right]$ приводится к стандартному виду 
\begin{equation}
	\int\limits_{V^{2}(l)}^{V^{2}}\frac{dV}{\sqrt{(1-V^{2})\cdot(k'^{2}+k^{2}v^{2})}}=-\int\limits_{l}^{\zeta}\sqrt{N_{1}+N_{2}}d\zeta'.
	\label{1.4.32}
\end{equation}
Решение \eqref{1.4.32} для $V^{2}(l)=0$  находится в виде 
\begin{equation}
	V^{2}(\zeta)=\frac{m_{s}^{2}(\zeta)}{N_{2}}=cn^{2}\left\{K(k)+\sqrt{N_{1}+N_{2}}(l-\zeta),k\right\},
	\label{1.4.33}
\end{equation}
где $K(k)$ -- полный эллиптический интеграл первого рода в нормальной форме \eqref{62}. В этом случае величина $m_{s}^{2}(0)$ определена полностью, если определено $l$. Если же задать $m_{s}^{2}(0)$  (или $\eta$), то при прочих фиксированных условиях нужно будет определить $l$. По полученному решению для $m_{s}^{2}$  и по первым интегралам \eqref{1.4.29} можно найти аналитические выражения  для $m_{p}^{2}$  и $m_{L}^{2}$. 

\subsection{Параметрическое приближение}

Приближение заданного поля $m_{L}^{2}=\bar{m}_{0}^{2}$, пожалуй, является наиболее важным с практической точки зрения. Исключая из рассмотрения первое уравнение \eqref{1.4.28} и подставляя $m_{L}=\bar{m}_{0}$  во все остальные, можно получить корректную систему уравнений для этого частного случая. Преобразуя эти уравнения, можно использовать первые интегралы \eqref{1.4.29} в форме 
\begin{equation}
	m_{s}^{2}+m_{p}^{2}=N_{2};\quad \bar{m}_{0}m_{p}m_{s}\cos\bar{\theta}+\frac{\delta}{2}m_{s}^{2}=\Gamma-\frac{\delta}{2}N_{1}\equiv\Gamma_{1}
	\label{1.4.34}
\end{equation}
или 
\begin{equation}
	\bar{m}_{0}m_{p}m_{s}\cos\bar{\theta}-\frac{\delta}{2}m_{p}^{2}=\Gamma-\frac{\delta}{2}N_{3}\equiv\Gamma_{2}.
	\tag{34'}\label{1.4.34'}
\end{equation}
Два первых интеграла \eqref{1.4.34} обычно дополняются граничными условиями в виде
\begin{itemize}
	\item[а)]$m_{s}^{2}(l)=0$ в плоскости $\zeta=l$
\end{itemize}
\begin{itemize}
	\item[б)] $m_{p}^{2}(0)=0$ в плоскости $\zeta=0$
\end{itemize}
В случае а) находим 
\begin{equation}
	\Gamma_{1}=0;\quad\cos\bar{\theta}=-(\delta/2\bar{m}_{0})\cdot(m_{s}/m_{p})\equiv-\bar{\delta}\cdot(m_{s}/m_{p})
	\tag{35a}\label{1.4.35a}
\end{equation}
и далее получаем из второго уравнения (28) уравнение для стоксова поля 
\begin{equation}
	-(dm_{s}/d\zeta)=\bar{m}_{0}m_{p}\sin\bar{\theta}=\sqrt{\bar{m}_{0}^{2}m_{p}^{2}-(\delta^{2}/4)m_{s}^{2}}=\bar{m}_{0}\sqrt{N_{2}-m_{s}^{2}(1+\bar{\delta}^{2})}.
	\tag{36a}\label{1.4.36a}
\end{equation}
Если ввести $V^{2}=(m_{s}^{2}/N_{2})\cdot(1+\bar{\delta}^{2})$ , то \eqref{1.4.36a} преобразуется в 
$$\int\limits_{0}^{V}\frac{dV}{\sqrt{1-V^{2}}}=-\int\limits_{l}^{\zeta}\bar{m}_{0}d\zeta'\sqrt{1+\bar{\delta}^{2}}$$
что в конечном итоге дает решение $V=\sin\left\{\bar{m}_{0}\sqrt{1+\bar{\delta}^{2}}(l-\zeta)\right\}\Rightarrow$
\begin{equation}
	m_{s}^{2}=\left\{N_{2}/(1+\bar{\delta}^{2})\right\}\sin^{2}\left\{\bar{m}_{0}\sqrt{1+\bar{\delta}^{2}}(l-\zeta)\right\}.
	\tag{37a}\label{1.4.37a}
\end{equation} 
Заметим, что \eqref{1.4.37a} совместно с \eqref{1.4.34} полностью определяет и решение для 
\begin{equation}
	m_{p}^{2}=N_{2}-m_{s}^{2}\equiv N_{2}\frac{\bar{\delta}^{2}+\cos^{2}\left\{\bar{m}_{0}\sqrt{1+\bar{\delta}^{2}}(l-\zeta)\right\}}{1+\bar{\delta}^{2}}
	\label{1.4.38a}\tag{38a}
\end{equation}
При этом $m_{p}^{2}$ нигде не обращается в нуль, если $\delta\neq0$, и поэтому $m_{p}(0)\neq0$.
\setcounter{equation}{38}
Если в случае $\bar{\delta}=0$  внутри слоя на расстоянии 
\begin{equation}
	\bar{m}_{0}l_{cr}=\beta c\sqrt{\frac{ca\omega_{s}\omega_{p}}{4\pi\mu\hat{n}_{s}}\bar{\rho}_{0}}L_{cr}=\frac{\pi}{2}
	\label{1.4.39}
\end{equation}
стоксова компонента поля обращается в нуль, то при этом в нуль обращается звуковое поле на границе $z=0$. Такое решение говорит о возможности описания реально происходящего процесса образования ВМБР назад от бесконечно длинного (т.е. без второй резкой границы) слоя нелинейной среды при нулевых граничных условиях для звука и стоксова поля и при одном непременном условии строгого резонанса
\begin{equation}
	\vec{k}_{p}+\vec{k}_{s}-\vec{k}_{L}=0.
	\label{1.4.40}
\end{equation}
Оно играет такую же роль, как положительная обратная связь в любом генераторе электромагнитного поля. 
Для случая б) в случае  граничных условий $m_{p}(0)=0$  находим 
\begin{equation}
	\Gamma_{2}=0;\qquad\cos\bar{\theta}=\bar{\delta}(m_{p}/m_{s})
	\tag{35б}\label{1.4.35б}
\end{equation}
и далее
\begin{equation}
	(dm_{p}/d\zeta)=\sqrt{\bar{m}_{0}^{2}m_{s}^{2}-(\delta^{2}/4)m_{p}^{2}}=\bar{m}_{0}\sqrt{N_{2}-m_{p}^{2}(1+\bar{\delta}^{2})}
	\tag{36б}\label{1.4.36б}
\end{equation}
Отсюда по аналогии находим решение 
\begin{equation}
	m_{p}^{2}=\left\{N_{2}/(1+\bar{\delta}^{2})\right\}\sin^{2}\left\{\bar{m}_{0}\sqrt{1+\bar{\delta}^{2}}\zeta\right\}.
	\tag{37б}\label{1.4.37б}
\end{equation}
При этом	
\begin{equation}
m_{p}^{2}=N_{2}-m_{s}^{2}=N_{2}\frac{\bar{\delta}^{2}+\cos^{2}\left\{\bar{m}_{0}\sqrt{1+\bar{\delta}^{2}}(l-\zeta)\right\}}{1+\bar{\delta}^{2}}
	\tag{38б}\label{1.4.38б}
\end{equation}
нигде не обращается в нуль, если $\delta\neq0$, и максимально при $\zeta=0$, \\где $m_{s}^{2}(0)=N_{2}$. 

\newpage
\setcounter{section}{0}
\setcounter{equation}{0}
\section*{Глава V. ПУЧКИ В НЕЛИНЕЙНОЙ ОПТИКЕ. ПАРАМЕТРИЧЕСКИЕ ЭФФЕКТЫ В ОПТИКЕ}
\textbf{Введение}

Параметрические процессы в электродинамике -- электромагнитные процессы в средах, параметры которых меняются во времени. \textit{Параметрические явления} в нелинейной среде – все явления, которые могут быть описаны в приближении \textit{фиксированной накачки}. 

В нелинейной оптике реально все процессы преобразования частот происходят в условиях, когда излучение имеет вид распространяющегося пучка. Рассмотрим взаимодействие пучков трех частот в квадратичной среде 
$$\vec{E}=Re\left\{\sum_{q=1}^{3}\vec{\tilde{e}^{0}}_{q}\tilde{E}(\vec{r},t)exp\left[i\omega_{q}t-i\left(\vec{k}_{q}\cdot\vec{r}\right)\right]\right\}.$$
Практически часто реализуются случаи, когда пучок излучения на одной из частот существенно превышает по мощности излучение на других частотах. В этих случаях оправдано применение приближения заданного поля (фиксированной накачки).

\section{Преобразование частот в волновых пучках в квадратичной среде}
\subsection{Основные уравнения}
В уравнениях, которые описывают взаимодействие пучков, нужно учесть зависимость полей $\tilde{E}_{q}$  от $\vec{r}_{\perp}$  и от времени $t$. Проведём обобщение системы  \eqref{20} на этот случай и с этой целью преобразуем первый член каждого уравнения 
\begin{equation}
	\left(\left[\vec{\tilde{e}^{0}}_{q}\times\left[\vec{k}_{q}\times\vec{\tilde{e}^{0}}_{q}\right]\right]\cdot\nabla\tilde{E}_{q}\right)=k_{q}\cos\alpha_{q}\left(\vec{s}_{q}^{0}\cdot\nabla\tilde{E}_{q}\right)\label{1.5.1}
\end{equation}
где $\vec{s}_{q}^{0}$ -- орта в направлении групповой скорости  $\vec{\textbf{v}}_{q}$ , которая определяется из соотношения 
\begin{equation}
	\overline{\vec{S}_{q}}^{T}=Re\frac{c}{8\pi}\left[\tilde{\vec{E}}_{q}\times\tilde{\vec{H}^{*}}_{q}\right]=\left(\overline{w_{em}}^{T}\right)_{q}\vec{v}_{q}	\label{1.5.2}
\end{equation}
где $\left(\overline{w_{em}}^{T}\right)_{q}$ -- плотность энергии э.-м. поля. Нужно учитывать также связи
$$\left[\vec{k}_{q}\times\tilde{\vec{e}^{0}}_{q}\tilde{E}_{q}\right]=\frac{\omega_{q}}{c}\mu\tilde{H}_{q}\tilde{\vec{h}^{0}}_{q};\vec{k}_{q}=\frac{\omega_{q}}{c}\vec{n}_{q}$$
$$\overline{\vec{S}}_{q}^{T}=\frac{c/\mu}{8\pi}\left\{\vec{n}_{q}\left(\vec{\tilde{E}}_{q}\cdot\vec{\tilde{E}^{*}}_{q}\right)-\tilde{\vec{e}^{0}}^{*}_{q}\tilde{E}_{q}\left(\vec{n}_{q}\tilde{\vec{e}^{0}}_{q}\right)\tilde{E}_{q}^{*}\right\}$$
Из Рис. 5.1 видно, что 
\begin{equation}
	\begin{split}
		&\vec{n}_{q}\parallel\vec{k}_{q}\Uparrow\vec{z}_{q0}\Uparrow\vec{z}_{0},\quad\vec{\tilde{D}}_{q}\parallel\vec{x}_{0};\quad\left(\vec{n}_{q}\cdot\vec{e}_{q}^{0}\right)=-n_{q}\sin\alpha_{q},\quad\vec{e}_{q}^{0}=\vec{x}_{0}\cos\alpha_{q}-\vec{z}_{0}\sin\alpha_{q};\\
		&\vec{s}_{q}^{0}=\left(\vec{z}_{0}\cos\alpha_{q}+\vec{x}_{0}\sin\alpha_{q}\right)\text{ и }\overline{\vec{S}_{q}}^{T}=\left(c/8\pi\mu\right)E_{q}^{2}n_{q}\cos\alpha_{q}\vec{s}_{q}^{0}.
	\end{split}
	\tag{2'}\label{1.5.2'}
\end{equation}
С другой стороны, воспользовавшись связями $$\tilde{\vec{D}}_{q}=-\left[\vec{n}_{q}\times\tilde{\vec{H}}_{q}\right],\qquad\mu\tilde{\vec{H}}_{q}=\left[\vec{n}_{q}\times\tilde{\vec{E}}_{q}\right]$$
и получив $$\tilde{\vec{D}}_{q}=-\left(1/\mu\right)\left[\vec{n}_{q}\times\left[\vec{n}_{q}\times\tilde{\vec{E}}_{q}\right]\right]=\left(1/\mu\right)\left\{n^{2}_{q}\tilde{\vec{E}}-\vec{n}_{q}\left(\vec{n}_{q}\cdot\tilde{\vec{E}}_{q}\right)\right\},$$
можно найти
\begin{equation}
	\left(\overline{w_{em}}\right)_{q}=\left(1/8\pi\right)\left(\tilde{\vec{E}}_{q}\cdot\tilde{\vec{D}^{*}}_{q}\right)=\left(1/8\pi\mu\right)n^{2}_{q}E_{q}^{2}\cdot\cos^{2}\alpha_{q}.
	\label{1.5.3}
\end{equation}
Из \eqref{1.5.2} и \eqref{1.5.3} найдем 
\begin{equation}
	\vec{\textbf{v}}_{q}=\left(c/n_{q}\cos\alpha_{q}\right)\left(\vec{z}_{0}\cos\alpha_{q}+\vec{x}_{0}\sin\alpha_{q}\right)\equiv\left(c/n_{q}\cos\alpha_{q}\right)\vec{s}_{q}^{0}
	\label{1.5.4}
\end{equation}
и представим операторы уравнений \eqref{20} в виде 
\begin{equation}
	k_{q}\cos\alpha_{q}\cdot\left(\vec{s}_{q}^{0}\cdot\nabla\tilde{E}_{q}\right)=\left(\omega_{q}/c^{2}\right)n_{q}^{2}\cos^{2}\alpha_{q}\cdot\left(\vec{v}_{q}\cdot\nabla\tilde{E}_{q}\right).
	\label{1.5.5}
\end{equation}

Вначале обобщим \eqref{1.5.5} на нестационарные процессы в приближении, которое позволяет пренебречь второй производной по времени 
\begin{equation}
	\left(\omega_{q}/c^{2}\right)n_{q}^{2}\cos^{2}\alpha_{q}\cdot\left(\vec{v}_{q}\cdot\nabla\tilde{E}_{q}\right)\rightarrow\left(\omega_{q}/c^{2}\right)n_{q}^{2}\cos^{2}\alpha_{q}\cdot\left(\vec{v}_{q}\cdot\nabla\tilde{E}_{q}\right)\cdot\left\{\left(\partial/\partial t\right)+\left(\vec{v}_{q}\cdot\nabla\right)\right\}\tilde{E}_{q}
	\label{1.5.6}
\end{equation} 

Переходя от приближения плоских волн (амплитуды волн предполагаются зависящими от  $\vec{r}_{\perp}$   как от параметра), в котором были получены уравнения \eqref{20}, к "диффузионному" приближению, которое описывает распространение пучков, вместо оператора \eqref{1.5.6} получим оператор 
\begin{multline}
	\left(\omega_{q}/c^{2}\right)n_{q}^{2}\cos^{2}\alpha_{q}\cdot\hat{L}_{q}\tilde{E}_{q}=\left(\omega_{q}/c^{2}\right)n_{q}^{2}\cos^{2}\alpha_{q}\times\\
	\times\left\{\left(\frac{\partial}{\partial t}\right)+\left(\vec{v}_{q}\cdot\nabla\right)+\frac{i}{2}\sum_{\alpha,\beta=1}^{2}\left(\frac{\partial^{2}\omega_{q}}{\partial k_{\alpha}\partial k_{\beta}}\right)\frac{\partial^{2}}{\partial x_{\alpha}\partial x_{\beta}}-\frac{i}{2}\left(\frac{d^{2}k_{q}}{d\omega_{q}^{2}}\right)\frac{\partial^{2}}{\partial t^{2}}\right\}\tilde{E}_{q}
	\label{1.5.7}
\end{multline}
в каждом уравнении типа \eqref{20}. В этом случае система уравнений примет вид 
\begin{equation}
	\begin{split}
		\left(\omega_{1,2}/c^{2}\right)n_{1,2}^{2}\cos^{2}\alpha_{1,2}\hat{L}_{1,2}\tilde{E}_{1,2}+&\bar{\gamma}_{1,2}\tilde{E}_{1,2}+i\beta\omega_{1,2}^{2}\tilde{E}_{3}\tilde{E}_{2,1}^{*}\exp\left\{-i\left(\Delta\vec{k}\cdot\vec{r}\right)\right\}=0;\\
		\left(\omega_{3}/c^{2}\right)n_{3}^{2}\cos^{2}\alpha_{3}\hat{L}_{3}\tilde{E}_{3}+&\bar{\gamma}_{3}\tilde{E}_{3}+i\beta\omega_{3}^{2}\tilde{E}_{2}\tilde{E}_{1}\exp\left\{+i\left(\Delta\vec{k}\cdot\vec{r}\right)\right\}=0.
	\end{split}
	\label{1.5.8}
\end{equation}
Уравнения \eqref{1.5.8} трудно решить даже в параметрическом случае.
\subsection{Уравнения одноволнового приближения при параметрическом преобразовании частот в условиях высокочастотной накачки}

Сделаем некоторые упрощения в постановке задачи.
\begin{enumerate}
	\item Рассмотрим стационарный трехчастотный случай, исключив зависимость амплитуд полей от времени  $t$.
	\item Будем считать, что все волны распространяются в одном направлении: $z_{1}=z_{2}=z_{3}=z$.
	\item Будем считать, что поле $\tilde{E}_{3}$  неизменно и не зависит от $\vec{r}_{\perp}$  и от $z$  (т.е. $\tilde{E}_{3} = const$). Это значит, что \eqref{1.5.8} превратится в систему из двух линейных уравнений, которые будут иметь постоянные коэффициенты. Такая система будет описывать параметрическое преобразование волновых пучков $\tilde{E}_{1,2}\left(\vec{r}_{\perp},z\right)$, имеющих частоты $\omega_{1,2}$, с помощью высокочастотной накачки $\tilde{E}_{3}$  на частоте  $\omega_{3}=\omega_{1}$+$\omega_{2}$.
	\item  Для простоты рассмотрим двухмерную задачу $\left(\partial\tilde{E}_{q}/\partial y\right)=0$ .
	\item Допустим, что имеет место полный синхронизм $(\Delta\vec{k}=0)$. 
	\item В качестве граничных зададим  условия
	\begin{equation}
		\tilde{E}_{1}\left(\vec{r}_{\perp},z=0\right)=\tilde{E}_{1}\left(\vec{r}_{\perp},0\right),\qquad\tilde{E}_{2}\left(\vec{r}_{\perp},z=0\right),
		\label{1.5.9}
	\end{equation}
	которые означают, что на границе отсутствует волновой пучок на частоте $\omega_{2}$.
	
	 При этих предположениях параметрическое усиление частот $\omega_{1}$  и $\omega_{2}$   будет описываться системой из двух линейных уравнений 
	 \begin{equation}
	 	\begin{split}
	 		&\left(\frac{\partial}{\partial z}+b_{1}\frac{\partial}{\partial x}+\frac{i}{2}a_{1}\frac{\partial^{2}}{\partial x^{2}}+\gamma_{1}\right)\tilde{E}_{1}+i\sigma_{1}\tilde{E}_{3}\tilde{E}_{2}^{*}=0\\
	 		-i\sigma_{2}\tilde{E}_{3}^{*}\tilde{E}_{1}+&\left(\frac{\partial}{\partial z}+b_{2}\frac{\partial}{\partial x}-\frac{i}{2}a_{2}\frac{\partial^{2}}{\partial x^{2}}+\gamma_{2}\right)\tilde{E}_{2}^{*}=0
	 	\end{split}
	 	\label{1.5.10}
	 \end{equation}
 	относительно полей $\tilde{E}_{1}$  и $\tilde{E}_{2}^{*}$ . В этих уравнениях использованы следующие обозначения: $b_{j}=\left(v_{jx}/v_{jz}\right)$  -- отношение компонент групповой скорости $\vec{v}_{j}$, 
 	\begin{equation}
 		\begin{split}
 			&\sigma_{j}=\left(\beta\omega_{j}c^{2}/v_{j}n_{j}^{2}\cos^{2}\alpha_{j}\right);\quad\gamma_{j}=\left(\bar{\gamma}_{j}c^{2}/v_{j}\omega_{j}n_{j}^{2}\cos^{2}\alpha_{j}\right);\\
 			&a_{j}=\left(1/v_{jz}\right)\cdot\left(\partial^{2}\omega_{j}/\partial k_{x}^{2}\right)=\left(1/v_{jz}\right)\cdot\left(\partial v_{jx}/\partial k_{x}\right)\cong\left(v_{j}/v_{jz}k_{j}\right)\cong\left(1/k_{j}\right).
 		\end{split}
 		\tag{10'}\label{1.5.10'}
 	\end{equation}
 	\item Для простоты положим, что линейные поглощения на $\omega_{1,2}$ одинаковы: 
 	\begin{equation}
 		\gamma_{1}=\gamma_{2}=\gamma.
 		\label{1.5.11}
 	\end{equation}
 	Линейная система уравнений \eqref{1.5.10} с постоянными коэффициентами допускает решение в виде суперпозиции плоских волн с амплитудами 
 	\begin{equation}
 		\tilde{E}_{1}(x,z)=\int\tilde{E}_{1}(k,z)\exp(-ikx)dk,\qquad\tilde{E}_{2}^{*}(x,z)=\int\tilde{E}_{2}^{*}(k,z)\exp(-ikx)dk.
 		\label{1.5.12}
 	\end{equation}
 	Используя преобразование \eqref{1.5.12}, нетрудно получить из уравнений \eqref{1.5.10} систему двух линейных уравнений 
 	\begin{equation}
	\begin{split}
		\left(\frac{\partial}{\partial z}-ib_{1}k-\frac{i}{2}a_{1}k^{2}+\gamma\right)\tilde{E}_{1}(k,z)+i\sigma_{1}\tilde{E}_{3}\tilde{E}_{2}^{*}(k,z)=0;\\
			-i\sigma_{2}\tilde{E}_{3}^{*}\tilde{E}_{1}(k,z)+\left(\frac{\partial}{\partial z}-ib_{2}k+\frac{i}{2}a_{2}k^{2}+\gamma\right)\tilde{E}_{2}^{*}(k,z)=0
	\end{split}
	\label{1.5.13}
	\end{equation}
	относительно комплексных амплитуд двух связанных друг с другом плоских волн, имеющих разные частоты $\omega_{1,2}$. Решение уравнений \eqref{1.5.13} нужно искать в виде 	

 	\begin{equation}
		\tilde{E}_{1}(k,z)=\tilde{E}_{10}(k)\exp\left(\tilde{\Lambda} z\right),\qquad\tilde{E}_{2}^{*}=\tilde{E}_{20}^{*}(k)\exp\left(\tilde{\Lambda} z\right).
	\label{1.5.14}
	\end{equation}
	В каждом частном решении \eqref{1.5.14} амплитуды $\tilde{E}_{10}(k)$, $\tilde{E}_{20}^{*}(k)$ и комплексная постоянная $\tilde{\Lambda}$  изменения (вдоль оси  $z$) медленной амплитуды $\tilde{E}_{1,2}(k,z)$ каждой плоской волны зависят от проекции $\kappa$  каждого комплексного волнового числа $$\tilde{\vec{k}}_{1,2}=(k_{1,2})\cdot\vec{z}_{0}\pm k\vec{x}_{0}\mp Im\tilde{\Lambda}\cdot\vec{z}_{0}+iRe\tilde{\Lambda}\cdot\vec{z}_{0}$$ на ось $x$ , а также от всех иных параметров среды и поля. 
	
	Вначале найдём это частное решение уравнений \eqref{1.5.13}. С этой целью подставим \eqref{1.5.14} в \eqref{1.5.13} и из условия нетривиальности решения \eqref{1.5.13} получим дисперсионное уравнение относительно $\tilde{\Lambda}$  в виде 
	\begin{multline}
		\left(\tilde{\Lambda}+\gamma\right)^{2}-i\left(\tilde{\Lambda}+\gamma\right)\left[(b_{1}+b_{2})k+\frac{k^{2}}{2}(a_{1}-a_{2})\right]-\\-\sigma_{1}\sigma_{2}E_{3}^{2}-k^{2}\left(b_{1}+\frac{a_{1}}{2}k\right)\left(b_{2}-\frac{a_{2}}{2}k\right)=0.		
		\label{1.5.15}
	\end{multline}
	Уравнение \eqref{1.5.15} выглядит громоздким, но если ввести
	\begin{multline}
		\tilde{\Gamma}=\cfrac{i}{2}\left[(b_{1}+b_{2})k+\cfrac{k^{2}}{2}(a_{1}-a_{2})\right],\\\quad\tilde{D}=\cfrac{i}{2}k(b_{1}-b_{2})+\cfrac{i}{4}k^{2}(a_{1}+a_{2}),\quad g^{2}=\sigma_{1}\sigma_{2}E_{3}^{2},
		\label{1.5.16}
	\end{multline}
	то корни дисперсионного уравнения \eqref{1.5.15} можно представить в компактном виде
	\begin{equation}
		\left(\tilde{\Lambda}+\gamma\right)_{1,2}=\tilde{\Gamma}\pm\sqrt{g^{2}-\tilde{D}^{2}}.
		\label{1.5.17}
	\end{equation}
	\item Упростим \eqref{1.5.17} считая, что поля $\tilde{E}_{1,2}\left(\vec{r},z\right)$  медленно изменяются по поперечной координате и потому $k_{x}\equiv \kappa<<\left|\tilde{k}_{z}\right|\cong\left|\tilde{k}_{1,2}\right|_{z}$. Тогда при \textbf{достаточно сильной анизотропии} (определяющей направления групповых скоростей) могут быть справедливы условия $(\kappa/k_{1,2})\cong(\kappa a_{1,2})<<b_{1,2}$. 
	\item Предположим также, что усиление достаточно велико: $g^{2}>>\left|\tilde{D}\right|^{2}$ . 
	
	При этих предположениях корни характеристического уравнения \eqref{1.5.15} можно представить в виде 
	\begin{equation}
		\left(\tilde{\lambda}+\gamma\right)_{1,2}\cong\frac{i}{2}\kappa(b_{1}+b_{2})+\frac{i}{4}\kappa^{2}(a_{1}-a_{2})\pm g\mp\frac{(b_{1}-b_{2})^{2}}{8g}\kappa^{2}+0\left\{\kappa^{3}\right\}.
		\label{1.5.17*}\tag{17*}
	\end{equation}
	Каждому из корней \eqref{1.5.17*} отвечает линейно независимое решение. Эти два линейно независимых решения описывают два процесса \textbf{(происходящих в параметрическом приближении при трехчастотном взаимодействии)}: процесс \textbf{\textit{усиления}} из-за распада фотонов $\omega_{3}$  и процесс \textbf{\textit{ослабления}} из-за слияния фотонов $\omega_{1}$ и $\omega_{2}$  в фотоны $\omega_{3}$. Основную роль играет нарастающее решение, описывающее процесс распада фотонов накачки $\omega_{3}$. Отдельные спектральные компоненты этих полей 
	\begin{equation}
		\begin{split}
			\vec{E}_{11}=\vec{e}_{1}^{0}Re\left\{\left(\tilde{E}_{10}(\kappa)\right)_{1}exp\left[i\omega_{1}t-i\left(\vec{k}_{1}\cdot\vec{r}\right)-i\kappa x+\tilde{\Lambda}_{1}(\kappa)z\right]\right\};\\
			\vec{E}_{21}=\vec{e}_{2}^{0}Re\left\{\left(\tilde{E}_{20}^{*}(\kappa)\right)_{1}exp\left[-i\omega_{2}t-+i\left(\vec{k}_{2}\cdot\vec{r}\right)-i\kappa x+\tilde{\Lambda}_{1}(\kappa)z\right]\right\};
		\end{split}
		\label{1.5.18}
	\end{equation}
	при условии
	\begin{equation}
		g>\gamma+\left\{(b_{1}-b_{2})^{2}\kappa^{2}/8g\right\}
		\label{1.5.19}
	\end{equation}
	являются нарастающими в  $+z$–направлении плоскими волнами. В конечном счёте, соответствующее этому решению двухчастотное поле на достаточно большом расстоянии $z$  от границы $z=0$  будет определять пространственные структуры обоих пучков $ \left\{\tilde{E}_{1,2}\left(\vec{r}_{\perp},z\right)\right\}_{1}$.
\end{enumerate}

	Второе линейно независимое решение ($\vec{E}_{12}$ и $\vec{E}_{22}$ соответственно) при достаточно большом значении  $g$  быстро затухает и на практически интересных расстояниях $z$  оказывается пренебрежимо малым по сравнению с \eqref{1.5.18}, так что им можно пренебречь. Такое  \underline{приближение} называется \underline{\textbf{\textit{одноволновым}}}. 
	
	В этом приближении, игнорирующем $\tilde{\Lambda}_{2}$, изменение каждого из двух полей \eqref{1.5.18} будет описываться одним единственным параболическим уравнением 
	\begin{equation}
		\left\{\frac{\partial}{\partial z}+\frac{1}{2}(b_{1}+b_{2})\frac{\partial}{\partial x}+\gamma-g\pm\frac{i}{4}\cdot(a_{1}-a_{2})\frac{\partial^{2}}{\partial x^{2}}-\frac{(b_{1}-b_{2})^{2}}{8g}\cdot\frac{\partial^{2}}{\partial x^{2}}\right\}\tilde{E}_{1,2}(x,z)=0.
		\label{1.5.20}
	\end{equation}
	Это уже два несвязанных друг с другом параболических уравнения относительно $\tilde{E}_{1,2}(x,z)$. В них в отличие от обычных уравнений квазиоптики для линейной среды есть вторые производные с мнимыми и реальными коэффициентами. Вторые производные с мнимыми коэффициентами описывают процессы, связанные с дифракцией усиливаемых волн. Вторые производные с действительными коэффициентами ответственны за процессы расплывания пучков, которые носят название параметрического расплывания или \underline{параметрической диффузии}
	
	Учитывая роль линейного усиления и ослабления с помощью решений 
	\begin{equation}
		\tilde{E}_{1,2}(x,z)=\tilde{\bar{E}}_{1,2}(x,z)\exp\left\{(g-\gamma)z\right\},
		\label{1.5.21}
	\end{equation}
	получим из \eqref{1.5.20} \textbf{консервативный} вариант уравнений \textbf{\textit{одноволнового}} приближения 
	\begin{equation}
		\left\{\frac{\partial}{\partial z}+\frac{1}{2}(b_{1}+b_{2})\frac{\partial}{\partial x}\pm\frac{i}{4}\cdot(a_{1}-a_{2})\frac{\partial^{2}}{\partial x^{2}}-\frac{(b_{1}-b_{2})^{2}}{8g}\cdot\frac{\partial^{2}}{\partial x^{2}}\right\}\tilde{\bar{E}}_{1,2}(x,z)=0.
		\label{1.5.22}
	\end{equation}
	\subsection{Дифракция усиливаемых волн и эффект аномальной фокусировки}
Рассмотрим случай распространения волн вдоль направления так называемого "касательного синхронизма", когда их групповые скорости колинеарны
\begin{equation}
	b_{1}=b_{2}=\bar{b}\equiv b,\qquad b_{1}-b_{2}=0.
	\label{1.5.23}
\end{equation}
Тогда в уравнениях \eqref{1.5.22} остаются только члены $\pm(i/4)(a_{1}-a_{2})$, ответственные за дифракцию. Особенность этих дифракционных явлений заключается в том, что эффективное волновое число 
\begin{equation}
	\overline{\bar{k}}=\frac{2}{a_{1}-a_{2}}\cong\frac{2k_{1}k_{2}}{k_{2}-k_{1}}=\frac{4\pi}{\lambda_{1}-\lambda_{2}}\equiv\bar{k}_{1}=-\bar{k}_{2}
	\label{1.5.24}
\end{equation}
может быть как положительно, если  $\lambda_{1}>\lambda_{2}$, так и отрицательно, если  $\lambda_{1}<\lambda_{2}$.

Будем считать, что $\overline{\bar{k}}>0$. Тогда в уравнении для поля \textbf{\textit{длинноволнового излучения}} знак коэффициента $\left(i/2\bar{k}_{1}\right)$ при члене $\left(i/2\bar{k}_{1}\right)\cdot\left(\partial^{2}\tilde{\bar{E}}_{1}/\partial x^{2}\right)$, учитывающем дифракционные эффекты, совпадает с тем, который имеет место в обычном диффузионном уравнении $$\left\{\frac{\partial}{\partial z}+\frac{i}{2k_{q}}\cdot\frac{\partial^{2}}{\partial x^{2}}\right\}\tilde{E}_{q}(x,z)=0$$
Особенность уравнения состоит только в том, что при  $\lambda_{1}\rightarrow\lambda_{2}$ <<дифракционный член>> становится исчезающе малым. Это означает, что \textbf{при  $\lambda_{1}\rightarrow\lambda_{2}$ поперечная структура поля не меняется вдоль} направления распространения пучка.

Вспомним решения параболического уравнения диффузии лучевой амплитуды поля.  Если на входе в среду задано поле в виде пучка с комплексной амплитудой 
\begin{equation}
	\tilde{E}_{q}=\tilde{E^{0}}_{q}\exp\left\{i\frac{\bar{k}_{q}x^{2}}{2R_{q}(0)}-\frac{x^{2}}{2\rho_{q}^{2}(0)}\right\}\otimes\exp\left[+i\left(\omega_{q}t-\bar{k}_{q}z\right)\right].
	\label{1.5.25}
\end{equation}
и при этом определен знак отношения, например, в виде  
\begin{equation}
	\left[\bar{k}_{q}/R_{q}(0)\right]<0,
	\label{1.5.26}
\end{equation}
то в соответствии с общей теорией решения уравнений параболического типа с заданными граничными условиями комплексная амплитуда поля в произвольном сечении (плоскости $z=const>0$) находится по формуле Френеля 
\begin{equation}
	\tilde{\bar{E}}_{q}(x,z)=\sqrt{\frac{i\bar{k}_{q}}{2\pi z}}\int\limits_{-\infty}^{\infty}\tilde{\bar{E}}_{q}(\xi,0)\exp\left\{-i\frac{\bar{k}_{q}(\xi-x)^{2}}{2z}\right\}d\xi.
	\label{1.5.27}
\end{equation}
Такое поле в области $z>0$  в среде с $\bar{k}_{1}>0$  (и соответственно  $R_{1}(0)<0$) будет расходящимся пучком: его радиус $\rho_{1}(z)$  будет монотонно расти в направлении $z>0$  в соответствии с известной квазиоптической формулой 
\begin{equation}
	\rho_{1}(z)=\rho_{1}(0)\sqrt{\left\{1-\frac{z}{2R_{1}(0)}\right\}^{2}+\left\{\frac{z}{\bar{k}_{1}\rho_{1}^{2}(0)}\right\}^{2}}.
	\label{1.5.28.1}\tag{$28_{1}$}
\end{equation}
\textit{В области $z<0$  радиус этого пучка $\rho_{1}(z)$  будет меньше $\rho_{1}(0)$  внутри некоторого интервала с центром в <<перетяжке>> $z_{min}<0$.}

В уравнении, описывающем распространение коротковолнового излучения $\tilde{\bar{E}}_{2}$, коэффициент при члене, учитывающем дифракционные эффекты, имеет противоположный знак. Диффузионный член этого уравнения имеет такой вид $$-\left(i/2\bar{\bar{k}}\right)(\partial^{2}\tilde{\bar{E}}_{2}/\partial x^{2})\equiv(i/2\bar{k}_{2})(\partial^{2}\tilde{\bar{E}}_{2}/\partial x^{2}),$$

как будто поле $\tilde{E}_{2}$  распространяется в <<среде с отрицательным волновым числом>>
\begin{equation}
	\bar{k}_{2}\equiv\bar{k}_{2\text{ эфф}}=-\overline{\bar{k}}
	\label{1.5.24.1}\tag{$24_{1}$}
\end{equation}
или в среде с положительным волновым числом в направлении  
\begin{equation}	
	z'\equiv-z.
	\label{1.5.24.2}\tag{$24_{2}$}
\end{equation}

В среде с $\bar{k}_{2}<0$  условие \eqref{1.5.26} справедливо, если $R_{2}(0)>0$. При этом комплексная амплитуда поля $\tilde{\bar{E}}_{2}(x,z)$ в $+Z$-направлении трансформируется точно так же, как в среде с $\bar{k}_{1}>0$ поле $\tilde{\bar{E}}_{1}(x,z)$ изменяется в $–Z$-направлении. Поэтому фактически получается, что при распространения поля $\tilde{\bar{E}}_{2}(x,z)$ в $+Z$-направлении ширина пучка $\rho_{2}(z)$  будет изменяться по закону $\rho_{2}(z')$ или 
\begin{equation}
	\rho_{2}(z)=\rho_{2}(0)\sqrt{\left\{1-\frac{-z}{2R_{2}(0)}\right\}^{2}+\left\{\frac{-z}{\bar{k}_{2}\rho_{2}^{2}(0)}\right\}^{2}}
	\label{1.5.28.2}\tag{$28_{2}$}
\end{equation}
и при $R_{2}(0)>0$ ширина пучка $\rho_{2}(z)$ будет монотонно расти. Поле $\tilde{\bar{E}}_{2}(x,z)$ в виде \eqref{1.5.25} в случае $R_{2}(0)<0$ (т.е. со сферически выпуклой поверхностью волнового фронта) будет фокусироваться и иметь минимальный поперечный размер (перетяжку) на расстоянии $z_{min}\cong\left|R_{2}(0)\right|$ от границы. 

Физическая причина такой аномальной фокусировки состоит в следующем. Поле $\tilde{\textbf{E}}_{1}(x,z)$  при взаимодействии с накачкой $\tilde{E}_{3}$   возбуждает такое коротковолновое излучение $\tilde{E}_{2}(x,z)$, что у комплексно сопряженной амплитуды $\tilde{E}_{2}^{*}$ зависимость фазы от поперечных координат будет точно такой же, как у амплитуды $\tilde{E}_{1}$. Если пучок усиливаемого длинноволнового излучения $\tilde{E}_{1}$ имеет выпуклый фазовый фронт (и должен расплываться в $+Z$–направлении), то у поля $\tilde{E}_{2}(x,z)\exp\left\{i\omega_{2}t-\left(\vec{k}_{2}\cdot\vec{r}\right)\right\}$   фазовый фронт оказывается уже вогнутым. 

При параметрическом взаимодействии в квадратичной среде характер изменения поперечной структуры пучков определяется длинноволновым излучением, дифракционные эффекты для которого играют более существенную роль, развиваясь на более коротких дифракционных длинах $l_{d1}=k_{1}\rho_{1}^{2}$. Другими словами, если поле $\tilde{E}_{1}(x,z)\exp\left\{i\omega_{1}t-\left(\vec{k}_{1}\cdot\vec{r}\right)\right\}$ расплывается в направлении распространения, то расплывается в этом направлении и поле \\ $\tilde{E}_{2}^{*}(x,z)\exp\left\{-i\omega_{2}t+\left(\vec{k}_{2}\cdot\vec{r}\right)\right\}$. 

\subsection{Параметрическая диффузия}

При распространении пучков с разными направлениями групповых скоростей становится существенной \textit{\textbf{\underline{параметрическая диффузия}}} (ПД). Для оценки влияния ПД рассмотрим важный частный случай 
\begin{equation}
	b_{1}=-b_{2}=b,\qquad b_{1}+b_{2}=0.
	\label{1.5.28}
\end{equation}
Кроме того, пренебрежем эффектом дифракции, считая
\begin{equation}
	k_{2}=k_{1}\Rightarrow\Rightarrow(\bar{k}^{-1})=0.
	\label{1.5.29}
\end{equation}
В этом случае уравнения \eqref{1.5.22} преобразуются к виду 
\begin{equation}
	\left(\frac{\partial}{\partial z}-\frac{b^{2}}{2g}\cdot\frac{\partial^{2}}{\partial x^{2}}\right)\tilde{\bar{E}}_{1,2}=0.
	\label{1.5.30}
\end{equation}
Следовательно, расчет поля в сечении $z$  по известному полю $\tilde{\bar{E}}(\xi,0)$  можно осуществить с помощью известного преобразования \eqref{1.5.27} в виде 
\begin{equation}
	\tilde{\bar{E}}(x,z)=\sqrt{\frac{g/b^{2}}{2\pi z}}\int\limits_{-\infty}^{-\infty}\tilde{\bar{E}}(\xi,0)\exp\left\{-\frac{(g/b^{2})(\xi-x)^{2}}{2z}\right\}d\xi,
	\label{1.5.31}
\end{equation}
в котором функция Грина (или оператор Грина $\hat{G}$) содержит комплексное волновое число 
\begin{equation}
	\bar{k}_{q}\equiv\tilde{k}=-i(g/b^{2})
\end{equation}

Если на входе слоя  $z=0$ имеется двухмерный Гауссов пучок с цилиндрическим волновым фронтом 
\begin{equation}
	\tilde{\bar{E}}(x,0)=\tilde{\bar{E}}_{0}\exp\left\{-(x^{2}/2\rho_{0}^{2})-i(kx^{2}/2R_{0})\right\},
	\label{1.5.33}
\end{equation}
где $\rho_{0}$  и $R_{0}$ -- ширина и радиус кривизны волнового фронта пучка, то в процессе распространения этот пучок из-за ПД будет деформироваться. В произвольном сечении $z=const$  ширина пучка будет определяться по формуле 
\begin{equation}
	\rho^{2}(z)\equiv\rho_{0}^{2}\frac{(1+\zeta)^{2}+\Lambda\zeta^{2}}{1+\zeta+\Lambda\zeta},
	\label{1.5.34}
\end{equation}
в которой используется безразмерная координата $\zeta=(zb^{2}/g\rho_{0}^{2})$  и безразмерный параметр $\Lambda=(k\rho_{0}^{2}/R_{0})^{2}$ . Цилиндрический фазовый фронт поля пучка будет иметь радиус кривизны 
\begin{equation}
	R(z)=R_{0}\left\{(1+\zeta)^{2}+\Lambda\zeta^{2}\right\}
	\label{1.5.35}
\end{equation}

Из \eqref{1.5.34} следует, что изменения $\rho^{2}(z)$  не зависят от знака радиуса кривизны $R_{0}$  фазового фронта поля пучка. В случае плоского фазового фронта $(R_{0}=\infty)$ ширина пучка $\rho^{2}(z)$ монотонно растет и на расстояниях $z\gg(g\rho_{0}^{2}/b^{2})$  приближается к \textbf{\textit{диффузионной ширине}} $\rho_{\infty}^{2}=z\cdot b^{2}/g$. 
Если $\Lambda=(k\rho_{0}^{2}/R_{0})^{2}\gg1$, то вначале на малых расстояниях $\zeta$  происходит уменьшение $\rho^{2}(z)$  до размера $$\rho_{min}^{2}=\rho_{0}^{2}\frac{2\sqrt{\Lambda}}{\Lambda+1}\cong\rho_{0}^{2}\frac{2}{\sqrt{\Lambda}}\equiv2\left(\left|{R_{0}}\right|/k\right)$$
при	$\zeta_{m}=\frac{\sqrt{\Lambda}-1}{\Lambda+1}\cong\frac{1}{\sqrt{\Lambda}}$. На этом расстоянии радиус кривизны фазового фронта пучка увеличивается примерно вдвое $$\left|R(\zeta_{m})\right|\cong\left|R_{0}\right|\left\{\left[1+\left(1/\sqrt{\Lambda}\right)\right]^{2}+\Lambda(1/\Lambda)\right\}\approx2\left|R_{0}\right|,$$
так что почти сравнивается с дифракционной длиной $k\rho_{min}^{2}\cong2\left|R_{0}\right|$. Затем ширина пучка $\rho^{2}(z)$  начинает монотонно увеличиваться и стремится к $\rho_{0\infty}$.

Начальное уменьшение $\rho^{2}(z)$ при малых значениях величины $\left|R_{0}\right|$ объясняется тем, что из-за сильной фокусировки поля заметно увеличивается локальное отклонение волнового вектора усиливаемого излучения от направления синхронизма. Именно из-за этого поле на краях пучка усиливается слабее, чем в центре. После того, как эта причина будет устранена, пучок при своём дальнейшем распространении в $+z$-направлении будет монотонно увеличиваться (по ширине) из-за ПД. 

Радиус кривизны фазового фронта согласно \eqref{1.5.35} монотонно и неограниченно растёт по абсолютной величине $\left(\lim\limits_{z\rightarrow}\left|R(z)\right|\rightarrow\infty\right)$, и при этом знак кривизны остаётся неизменным. 




\end{document}